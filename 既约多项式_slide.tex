\documentclass[aspectratio=169]{ctexbeamer}
%\usetheme{Madrid}
\usetheme{Boadilla}
%\usecolortheme{beaver}
\usepackage{amsmath} 
\usepackage{amssymb} 
\usepackage{amsfonts} 
\usepackage{multicol}
\usepackage{graphicx}
\usepackage{pgfplots}
\pgfplotsset{compat=1.18}
\usefonttheme[onlymath]{serif} % 衬线数学字体

%\setbeamertemplate{theorem}[ams style]
\setbeamertemplate{theorems}[numbered]

\theoremstyle{definition}
\newtheorem{question}{问题}[section]
\newtheorem{exercise}{练习}[section]
\newtheorem{formula}{公式}[section]
\newtheorem{proposition}{命题}[section]
\newtheorem{property}{性质}[section]
	
\title[既约多项式]{既约多项式}
\subtitle{}
\author[珠海一中创美营]{珠海一中创美营(数学)}
\date[\today]{\today}
% \AtBeginSection[]
% {
% 	\begin{frame}
% 		\frametitle{目录}
% 		\tableofcontents[currentsection]
% 	\end{frame}
% }
\begin{document}
\frame{\titlepage}
%\begin{frame}
%	\frametitle{目录}
%	\tableofcontents
%\end{frame}

\section{既约多项式}

\subsection{艾氏判别法}
\begin{frame}
	\begin{theorem}[艾森斯坦判别法]
		设 $f(x)=a_{n} x^{n}+a_{n-1} x^{n-1}+\cdots+a_{1} x+a_{0}$ 是整系数多项式.

		如果存在一个质数 $p$ 满足以下条件:
		\begin{enumerate}
			\item $p$ 不整除 $a_{n}$;
			\item $p$ 整除其余的系数 $\left(a_{0}, a_{1}, \cdots, a_{n-1}\right)$;
			\item $p^{2}$ 不整除 $a_{0}$ .
		\end{enumerate}
		那么, $f(x)$ 在有理数集内不可约.
	\end{theorem}
\end{frame}

\setcounter{theorem}{0}
% 例 1
\begin{frame}[t]
	\begin{example}
		证明:对于任意的自然数 $n, x^{n}-2$ 在有理数集内不可约.
	\end{example}
\end{frame}


% 例 2
\begin{frame}[t]
	\begin{example}
		证明: $x^{4}+x^{3}+x^{2}+x+1$ 在有理数集内不可约.
	\end{example}
\end{frame}


% 例 3
\begin{frame}[t]
	\begin{example}
		$x^{6}+x^{3}+1$ 在有理数集内不可约.
	\end{example}
\end{frame}

\subsection{奇与偶}
\begin{frame}{奇与偶}
	如果把所有的奇数用 1 表示, 偶数用 0 表示, 那么就得到一种奇怪的算术:
	\begin{align*}
		0+0=0,0+1=1+0=1,1+1=0
	\end{align*}
	它们表示两个偶数的和是偶数;一个偶数与一个奇数的和是奇数;两个奇数的和是偶数. (在数论中, 这是以 2 为模的算术)

	采用这种算术, 可以使问题大为简化, 不但整数只有两个( 0 与 1 ), 而且多项式的个数也大大减少. 一次多项式只有两个, 即
	\begin{align*}
		x, x+1
	\end{align*}

	实际上, 如 $3 x+4$ 可以归为第一种,  $3 x+5$ 可以归为第 2 种, 而 $2 x+4=0$ 不是一次多项式. 二次多项式只有 4 个, 即
	\begin{align*}
		x^{2}, x^{2}+x, x^{2}+1, x^{2}+x+1,
	\end{align*}
	其中,  $x^{2}=x \cdot x, x^{2}+x=x(x+1), x^{2}+1=(x+1)^{2}$ , 都不是既约多项式, 只有 $x^{2}+x+1$ 是既约多项式.
\end{frame}

% 例 4
\begin{frame}[t]
	\begin{example}
		证明:当 $(b+c) d$ 为奇数时, 整系数的三次多项式 $x^{3}+b x^{2}+c x+$ $d$ 在有理数集内不可约.
	\end{example}
\end{frame}

% 例 5
\begin{frame}[t]
	\begin{example}
		证明: $x^{5}+x^{2}-1$ 在有理数集内不可约.
	\end{example}
\end{frame}

% 例 6
\begin{frame}[t]
	\begin{example}
		证明 $x^{6}+x^{3}-1$ 在有理数集内不可约.
	\end{example}
\end{frame}

% 例 7
\begin{frame}[t]
\begin{example}
	证明 $x^{4}+3 x^{3}+3 x^{2}-5$ 在有理数集内不可约.
\end{example}
\end{frame}

\section*{未完待续}
\begin{frame}
	\Huge
	$$
		\mathcal{TO}	\quad
		\mathcal{BE}  	\quad
		\mathcal{CONTINUED}
		\ldots
	$$
\end{frame}
\end{document}