\documentclass{March}
\title{平面几何选讲}
\author{LeyuDame}
\date{\today}
%\excludecomment{analysis}
%\excludecomment{proof}
%\excludecomment{solution}
%\excludecomment{note}

\begin{document}
\maketitle
\tableofcontents
\newpage
\section{面积公式及其应用}
\begin{itemize}
	\item 利用图形的面积公式, 可以解决许多与面积相关的问题.
	\item 对于常见的特殊图形面积的计算, 一般直接使用公式或等积变换, 对于非常规图形面积的计算, 可通过图形的割补, 以及图形的运动(平移, 旋转, 翻折)来转换成特殊图形面积问题.
	\item 有时题目中并没有直接涉及面积, 但可以通过对同一图形面积的不同算法, 推出需要的代数或几何关系, 从而使问题获解.
\end{itemize}
\begin{formula}[海伦公式]
	若已知三角形三边长$a, b, c$, 则三角形面积
	$$
		S=\sqrt{p(p-a)(p-b)(p-c)},
	$$其中$p=\frac{a+b+c}{2}$.
\end{formula}

\subsection{例题}
% 第一题
\begin{question}
	如图,  长方形 $A B C D$ 的面积是 2012 平方厘米, 梯形 $A E G F$ 的顶点 $F$ 在 $B C$上, $D$ 是腰 $E G$ 的中点, 试求梯形 $AEGF$​的面积.


	\begin{centertikzpicture}[x=0.75pt,y=0.75pt,yscale=-1,xscale=1]
		%uncomment if require: \path (0,300); %set diagram left start at 0, and has height of 300

		%Shape: Rectangle [id:dp555795369835903] 
		\draw   (256.24,151.98) -- (415.65,151.98) -- (415.65,214.5) -- (256.24,214.5) -- cycle ;
		%Straight Lines [id:da8497557282410093] 
		\draw    (256.24,151.98) -- (365.91,127.34) ;
		%Straight Lines [id:da975966892252139] 
		\draw    (334.96,214.27) -- (474.23,178.81) ;
		%Straight Lines [id:da45109299717958984] 
		\draw    (365.91,127.34) -- (474.23,178.81) ;
		%Straight Lines [id:da6456969007522482] 
		\draw    (256.24,151.98) -- (334.96,214.27) ;

		% Text Node
		\draw (242.5,135.63) node [anchor=north west][inner sep=0.75pt]    {$A$};
		% Text Node
		\draw (244.06,214.18) node [anchor=north west][inner sep=0.75pt]    {$B$};
		% Text Node
		\draw (415.15,214.4) node [anchor=north west][inner sep=0.75pt]    {$C$};
		% Text Node
		\draw (414.49,133.65) node [anchor=north west][inner sep=0.75pt]    {$D$};
		% Text Node
		\draw (328.35,216.63) node [anchor=north west][inner sep=0.75pt]    {$F$};
		% Text Node
		\draw (364.06,109.32) node [anchor=north west][inner sep=0.75pt]    {$E$};
		% Text Node
		\draw (473.15,176.82) node [anchor=north west][inner sep=0.75pt]    {$G$};


	\end{centertikzpicture}\captionof{figure}{}


\end{question}
\begin{analysis}
	重点关注矩形 $A B C D ,  \triangle D F A$ 和梯形 $A E G F$ 的面积关系.
\end{analysis}
\begin{solution}
	取 $A F$ 的中点 $K$, 连结 $D K ,  D F$.

	由题意, $D K$ 是梯形 $A E G F$ 的中位线,因此,

	$$
		D K\pll A E \pll F G, D K=\frac{1}{2}(A E+F G),
	$$

	所以
	$$
		\frac{S_{\triangle A E D}}{S_{\triangle A K D}}=\frac{A E}{D K}, \frac{S_{\triangle D F G}}{S_{\triangle D F K}}=\frac{F G}{D K} .
	$$

	由 $K$ 为 $A F$ 中点, 知
	$$
		S_{\triangle A D K}=S_{\triangle F D K}, S_{\triangle A E D}+S_{\triangle D F G}=S_{\triangle A F D},
	$$

	所以 $S_{\text {梯形 } A E G F}=2 S_{\triangle A F D}$. 结合 $S_{\text {矩形 } A B C D}=2 S_{\triangle A F D}$, 可得
	$$
		S_{\text {梯形 } A E G F}=S_{\text {矩形 } A B C D}=2012 .
	$$
\end{solution}
\begin{note}
	通过本题可了解几种常见的面积变换. 也可延长 $A E$ 和 $F D$ 交于一点, 通过全等方法证明 $S_{\text {梯形 } A B G F}=2 S_{\triangle A D F}$.
\end{note}

\begin{question}
	如图,  在梯形 $A B C D$ 中, $A D\pll B C, A D: B C=1: 2, F$ 为线段 $A B$ 上的点, $E$ 为线段 $F C$ 上的点, 且 $S_{\triangle A O F}: S_{\triangle D O E}=1: 3, S_{\triangle B E F}=24$,求 $\triangle A O F $ 的面积.

	\tikzset{every picture/.style={line width=0.75pt}} %set default line width to 0.75pt        
	\begin{centertikzpicture}[x=0.75pt,y=0.75pt,yscale=-1,xscale=1]
		%uncomment if require: \path (0,300); %set diagram left start at 0, and has height of 300

		%Shape: Trapezoid [id:dp7581601526669886] 
		\draw   (265.43,215.64) -- (289.43,134.64) -- (399.4,134.64) -- (493.64,215.64) -- cycle ;
		%Straight Lines [id:da06654590240440239] 
		\draw    (282.29,160.86) -- (493.64,215.64) ;
		%Straight Lines [id:da08516216273632549] 
		\draw    (399.4,134.64) -- (352.29,178.86) ;
		%Straight Lines [id:da5968764151812478] 
		\draw    (289.43,134.64) -- (352.29,178.86) ;
		%Straight Lines [id:da3050793013431803] 
		\draw    (282.29,160.86) -- (399.4,134.64) ;
		%Straight Lines [id:da6366873373294237] 
		\draw    (352.29,178.86) -- (265.43,215.64) ;

		% Text Node
		\draw (286.67,115.26) node [anchor=north west][inner sep=0.75pt]    {$A$};
		% Text Node
		\draw (250.67,217.26) node [anchor=north west][inner sep=0.75pt]    {$B$};
		% Text Node
		\draw (495.64,219.04) node [anchor=north west][inner sep=0.75pt]    {$C$};
		% Text Node
		\draw (400.67,116.59) node [anchor=north west][inner sep=0.75pt]    {$D$};
		% Text Node
		\draw (312.67,135.26) node [anchor=north west][inner sep=0.75pt]    {$O$};
		% Text Node
		\draw (346.29,182.26) node [anchor=north west][inner sep=0.75pt]    {$E$};
		% Text Node
		\draw (264.67,151.92) node [anchor=north west][inner sep=0.75pt]    {$F$};
	\end{centertikzpicture}\captionof{figure}{}


\end{question}
\begin{analysis}
	由于平行线间的距离处处相等,可将其作为一些三角形的高之和, 进行面积的变换.
\end{analysis}
\begin{solution}
	过点 $E$ 作 $A D$ 的垂线,分别交 $A D$ ,  $B C$ 于点 $N ,  M$. 过点 $F$ 作 $A D$ 的垂线, 分别交 $D A$ 延长线及 $B C$ 于点 $Q ,  P$.
	由于 $A D\pll B C$, 故 $P Q=M N$.
	所以, $Q F+F P=N E+E M$, 即 $F P-E M=N E-Q F$. 从而
	$$
		\begin{aligned}
			S_{\triangle A E D}-S_{\triangle A F D} & =\frac{1}{2}(E N-F Q) \cdot A D                                  \\
			                                        & =\frac{1}{2}(P F-M E) \cdot \frac{1}{2} B C                      \\
			                                        & =\frac{1}{2}\left(S_{\triangle B F C}-S_{\triangle B E C}\right) \\
			                                        & =\frac{1}{2} S_{\triangle B F E}=12 .
		\end{aligned}
	$$

	由 $S_{\triangle A O F}: S_{\triangle D O E}=1: 3$, 知
	$$
		S_{\triangle A E D}-S_{\triangle A F D}=S_{\triangle D O E}-S_{\triangle A O F}=2 S_{\triangle A O F}=12,
	$$

	所以 $S_{\triangle A O F}=6$.
\end{solution}
\begin{note}
	本题虽然只用了最基本的面积公式 $S=\frac{1}{2} a h$, 但发现了三角形高的差不变的性质,巧妙地转换成面积的差的关系. 在解题中要善于发现题中隐藏的不变量.
\end{note}

\begin{question}
	如图,  $P$ 是 $\triangle A B C$ 内的一点, 连结 $A P ,  B P ,  C P$ 并延长, 分别与 $B C ,  A C ,  A B$ 交于点 $D$ ,  $E ,  F$. 已知 $A P=6, B P=9, D P=6, E P=3, C F=20$.求 $\triangle A B C$ 的面积.


	\tikzset{every picture/.style={line width=0.75pt}} %set default line width to 0.75pt        

	\begin{centertikzpicture}[x=0.75pt,y=0.75pt,yscale=-1,xscale=1]
		%uncomment if require: \path (0,300); %set diagram left start at 0, and has height of 300

		%Shape: Triangle [id:dp5793965072731759] 
		\draw   (350.39,76.04) -- (384.07,233.57) -- (293,233.57) -- cycle ;
		%Straight Lines [id:da6002888595411584] 
		\draw    (350.39,76.04) -- (329,233.54) ;
		%Straight Lines [id:da022731049041765816] 
		\draw    (312.5,181.04) -- (384.07,233.57) ;
		%Straight Lines [id:da14173178559276556] 
		\draw    (369.5,165.04) -- (293,233.57) ;

		% Text Node
		\draw (275.43,230.83) node [anchor=north west][inner sep=0.75pt]    {$A$};
		% Text Node
		\draw (384.93,230.28) node [anchor=north west][inner sep=0.75pt]    {$B$};
		% Text Node
		\draw (342.93,57.28) node [anchor=north west][inner sep=0.75pt]    {$C$};
		% Text Node
		\draw (371.93,148.28) node [anchor=north west][inner sep=0.75pt]    {$D$};
		% Text Node
		\draw (295.43,165.28) node [anchor=north west][inner sep=0.75pt]    {$E$};
		% Text Node
		\draw (321.93,235.28) node [anchor=north west][inner sep=0.75pt]    {$F$};
		% Text Node
		\draw (330.25,202.7) node [anchor=north west][inner sep=0.75pt]    {$P$};


	\end{centertikzpicture}\captionof{figure}{}


\end{question}
\begin{analysis}
	可围绕 $P$ 为 $A D$ 中点这个条件构造平行线, 求出其余未知线段的长度. 再利用海伦公式及面积关系求解.
\end{analysis}
\begin{solution}
	过点 $D$ 作 $A E$ 的平行线交 $B P$ 于点 $M$, 过点 $D$ 作 $A B$ 的平行线交 $C P$ 于点 $N$.

	因为 $A P=P D=6$ 且 $D M\pll A E$, 得
	$$
		EP=M P=3,
	$$

	进而, $E M=6, B M=6$, 即 $M$ 为 $B E$ 的中点, 所以 $D$ 为 $B C$ 的中点, 即 $C D=B D$.

	因为 $D N\pll A B$, 所以 $N$ 为 $C F$ 的中点, $P$ 为 $N F$ 的中点.

	从而 $C N=F N=10, N P=P F=5, C P=15$.

	由中线长公式: $P D^2=\frac{1}{2} C P^2+\frac{1}{2} B P^2-\frac{1}{4} B C^2$ 可得, $B C=6 \sqrt{13}$, $C D=3 \sqrt{13}$.

	由海伦公式: $S_{\triangle C P D}=27$, 故 $S_{\triangle A B C}=4 S_{\triangle C P D}=108$.
\end{solution}
\begin{note}
	常见的三角形面积公式有:
	$$
		\begin{aligned}
			S_{\triangle A B C} & =\frac{1}{2} a h_a=\frac{1}{2} b h_b=\frac{1}{2} c h_c                \\
			                    & =\frac{1}{2} a b \sin C=\frac{1}{2} b c \sin A=\frac{1}{2} a c \sin B \\
			                    & =\frac{a b c}{4 R}                                                    \\
			                    & =2 R^2 \sin A \sin B \sin C                                           \\
			                    & =p r                                                                  \\
			                    & =\sqrt{p(p-a)(p-b)(p-c) .}
		\end{aligned}
	$$

	其中 $\triangle A B C$ 三边为 $a ,  b ,  c$, 对应边上的高为 $h_a ,  h_b ,  h_c$, 外接圆, 内切圆半径分别为 $R ,  r$, 半周长 $p=\frac{1}{2}(a+b+c)$.
\end{note}

% 第四题
\begin{question}
	如图,  设凸四边形 $A B C D$ 内接于以 $O$ 为中心的圆, 且两条对角线相互垂直. 求证:折线 $A O C$ 分该四边形面积相等的两部分.


	\tikzset{every picture/.style={line width=0.75pt}} %set default line width to 0.75pt        

	\begin{centertikzpicture}[x=0.75pt,y=0.75pt,yscale=-1,xscale=1]
		%uncomment if require: \path (0,300); %set diagram left start at 0, and has height of 300

		%Shape: Ellipse [id:dp37604538191852277] 
		\draw   (273.22,178.2) .. controls (273.09,136.33) and (306.93,102.28) .. (348.8,102.15) .. controls (390.67,102.01) and (424.73,135.85) .. (424.86,177.72) .. controls (424.99,219.6) and (391.15,253.65) .. (349.28,253.78) .. controls (307.41,253.91) and (273.36,220.07) .. (273.22,178.2) -- cycle ;
		%Straight Lines [id:da5750849334361499] 
		\draw    (286.65,134.79) -- (410.86,134.79) ;
		%Straight Lines [id:da8015473179076149] 
		\draw    (320.29,108.14) -- (320.29,248.4) ;
		%Straight Lines [id:da7244779949500968] 
		\draw    (320.29,108.14) -- (349.04,177.96) ;
		%Straight Lines [id:da7342277540359712] 
		\draw    (349.04,177.96) -- (320.29,248.4) ;
		%Straight Lines [id:da7314181000675453] 
		\draw    (286.65,134.79) -- (320.29,248.4) ;
		%Straight Lines [id:da8054004678976356] 
		\draw    (286.65,134.79) -- (320.29,108.14) ;
		%Straight Lines [id:da8200100236510444] 
		\draw    (320.29,108.14) -- (410.86,134.79) ;
		%Straight Lines [id:da4698153024206404] 
		\draw    (410.86,134.79) -- (320.29,248.4) ;

		% Text Node
		\draw (309.92,249.3) node [anchor=north west][inner sep=0.75pt]    {$A$};
		% Text Node
		\draw (349.03,174.23) node [anchor=north west][inner sep=0.75pt]    {$O$};
		% Text Node
		\draw (414.4,121.04) node [anchor=north west][inner sep=0.75pt]    {$B$};
		% Text Node
		\draw (309.08,90.28) node [anchor=north west][inner sep=0.75pt]    {$C$};
		% Text Node
		\draw (272.07,122) node [anchor=north west][inner sep=0.75pt]    {$D$};


	\end{centertikzpicture}\captionof{figure}{}


\end{question}
\begin{analysis}
	对角线互相垂直的四边形面积等于对角线乘积的一半.
\end{analysis}
\begin{proof}
	过点 $O$ 作 $O P \perp B D$, 垂足为点 $P$.

	故 $S_{\text {四这形 } A O C D}=S_{\triangle A O C}+S_{\triangle A C D}=\frac{1}{2} A C \cdot P D$.

	因为 $O P \perp B D$, 故 $P D=\frac{1}{2} B D$, 所以
	$$
		S_{\text {四边形 } A O C D}=\frac{1}{4} A C \cdot B D .
	$$

	因为 $A C \perp B D$, 所以 $S_{\text {四边形 } A B C D}=\frac{1}{2} A C \cdot B D$.

	所以 $S_{\text {四边形 } A B C D}=2 S_{\text {四边形 } A O C D}$, 即折线 $A O C$ 分该四边形面积相等的两部分.
\end{proof}

\begin{question}
	设 $P$ 是 $\triangle A B C$ 内一点, 延长 $A P ,  B P ,  C P$ 与对边相交于点 $D ,  E$ ,  $F$. 设 $A P=a, B P=b, C P=c$, 且 $a+b+c=43$, $P D=P E=P F=d=3$, 求 $a b c$ 的值.


	\tikzset{every picture/.style={line width=0.75pt}} %set default line width to 0.75pt        

	\begin{centertikzpicture}[x=0.75pt,y=0.75pt,yscale=-1,xscale=1]
		%uncomment if require: \path (0,300); %set diagram left start at 0, and has height of 300

		%Shape: Triangle [id:dp5363504387490252] 
		\draw   (334.99,99.82) -- (445.01,206.8) -- (275.31,206.8) -- cycle ;
		%Straight Lines [id:da2874789810342133] 
		\draw    (335.01,99.82) -- (342.36,206.78) ;
		%Straight Lines [id:da18274628283403493] 
		\draw    (295.86,171.13) -- (445.01,206.8) ;
		%Straight Lines [id:da10060626201660083] 
		\draw    (397.3,160.26) -- (275.27,206.8) ;

		% Text Node
		\draw (258.5,203.83) node [anchor=north west][inner sep=0.75pt]    {$A$};
		% Text Node
		\draw (446.89,203.46) node [anchor=north west][inner sep=0.75pt]    {$B$};
		% Text Node
		\draw (329.85,82.65) node [anchor=north west][inner sep=0.75pt]    {$C$};
		% Text Node
		\draw (400.94,145.11) node [anchor=north west][inner sep=0.75pt]    {$D$};
		% Text Node
		\draw (280.59,157.99) node [anchor=north west][inner sep=0.75pt]    {$E$};
		% Text Node
		\draw (334.3,206.86) node [anchor=north west][inner sep=0.75pt]    {$F$};
		% Text Node
		\draw (342.02,160.7) node [anchor=north west][inner sep=0.75pt]    {$P$};


	\end{centertikzpicture}\captionof{figure}{}


\end{question}
\begin{analysis}
	可将条件中相应线段比转换成面积比,通过面积相等的关系入手解题.
\end{analysis}
\begin{solution}
	由共边比例定理可得: $\frac{S_{\triangle P A B}}{S_{\triangle A B C}}=\frac{d}{c+d}$.

	同理: $\frac{S_{\triangle P B C}}{S_{\triangle A B C}}=\frac{d}{a+d}, \frac{S_{\triangle P C A}}{S_{\triangle A B C}}=\frac{d}{b+d}$.

	故 $\frac{d}{a+d}+\frac{d}{b+d}+\frac{d}{c+d}=\frac{S_{\triangle P A B}+S_{\triangle P B C}+S_{\triangle P C A}}{S_{\triangle A B C}}=\frac{S_{\triangle A B C}}{S_{\triangle A B C}}=1$.

	从而 $2 d^3+(a+b+c) d^2-a b c=0$, 所以 $a b c=441$.
\end{solution}

\begin{question}
	如图,  设 $\triangle A B C$ 的三条中线 $A D ,  B E ,  C F$ 交于点 $G$, 且 $\triangle A G F, \triangle C G D$ 和 $\triangle B G D$ 的内切圆半径都相同. 证明: $\triangle A B C$ 是正三角形.

	\tikzset{every picture/.style={line width=0.75pt}} %set default line width to 0.75pt        

	\begin{centertikzpicture}[x=0.75pt,y=0.75pt,yscale=-1,xscale=1]
		%uncomment if require: \path (0,300); %set diagram left start at 0, and has height of 300

		%Shape: Triangle [id:dp47034667075097025] 
		\draw   (349.48,96.34) -- (426.95,232.4) -- (272,232.4) -- cycle ;
		%Straight Lines [id:da7570166699481862] 
		\draw    (349.48,96.34) -- (350.79,232.11) ;
		%Straight Lines [id:da12462221947186469] 
		\draw    (272,232.4) -- (387.29,163.61) ;
		%Straight Lines [id:da8938121913300356] 
		\draw    (311.79,162.61) -- (426.95,232.4) ;

		% Text Node
		\draw (342.71,79.4) node [anchor=north west][inner sep=0.75pt]    {$A$};
		% Text Node
		\draw (258.71,229.4) node [anchor=north west][inner sep=0.75pt]    {$B$};
		% Text Node
		\draw (426.71,228.9) node [anchor=north west][inner sep=0.75pt]    {$C$};
		% Text Node
		\draw (343.71,233.9) node [anchor=north west][inner sep=0.75pt]    {$D$};
		% Text Node
		\draw (388.21,149.4) node [anchor=north west][inner sep=0.75pt]    {$E$};
		% Text Node
		\draw (302.21,148.9) node [anchor=north west][inner sep=0.75pt]    {$F$};
		% Text Node
		\draw (352.21,160.4) node [anchor=north west][inner sep=0.75pt]    {$G$};


	\end{centertikzpicture}\captionof{figure}{}


\end{question}
\begin{analysis}
	利用 $S=p \cdot r$, 及面积和内切圆半径相等的条件,得到相应的边的关系,再进行论证.
\end{analysis}
\begin{proof}
	设 $\triangle A F G$ 和 $\triangle C D G$ 的内心分别为 $I_1$和 $I_2$, 过 $I_1$ 作 $I_1 M_1 \perp F G$ 于点 $M_1$, 过 $I_2$ 作 $I_2 M_2 \perp$ $G D$ 于点 $M_2$, 连结 $I_1 G$ 和 $I_2 G$.

	因为 $D$ 为 $B C$ 的中点, 所以 $B D=C D$, 从而 $S_{\triangle B D G}=S_{\triangle C D G}$.

	故$\frac{1}{2}(B D+D G+B G) \cdot r=\frac{1}{2}(C D+D G+G C) \cdot r$.

	故 $B G=C G, G D \perp B C$.

	进而 $A D$ 垂直平分 $B C$, 所以 $A B=A C$.

	在 Rt $\triangle I_1 M_1 G$ 和 Rt $\triangle I_2 M_2 G$ 中有: $I_1 M_1=I_2 M_2$, 且有
	$$
		\angle I_1 G M_1=\frac{1}{2} \angle A G F=\frac{1}{2} \angle D G C=\angle I_2 G M_2 .
	$$

	从而 Rt $\triangle I_1 M_1 G \cong$ Rt $\triangle I_2 M_2 G$, 所以 $G M_1=G M_2$.

	因为 $G M_1=\frac{1}{2}(A G+G F-A F), G M_2=\frac{1}{2}(G D+G C-C D)$,故 $A G+G F-A F=G D+G C-C D$.

	因为 $A F+F G+G A=G D+D C+C G$, 所以 $A F=C D$.

	因而 $A B=B C$. 所以 $A B=A C=B C$, 即 $\triangle A B C$ 为正三角形.
\end{proof}
\begin{note}
	利用三角形面积的不同表达式, 可以建立各种等量关系, 帮助我们从中发现规律, 得到解题思路.
\end{note}

\begin{question}
	如图,  $\triangle A B C$ 的三边上 $B C=a, C A=b, A B=c, a, b$, $c$ 都是整数, 且 $a, b$ 的最大公约数为2. 点 $G$ 和点 $I$ 分别为 $\triangle A B C$ 的重心和内心, 且 $\angle G I C=90^{\circ}$. 求 $\triangle A B C$ 的周长.

	\tikzset{every picture/.style={line width=0.75pt}} %set default line width to 0.75pt        

	\begin{centertikzpicture}[x=0.75pt,y=0.75pt,yscale=-1,xscale=1]
		%uncomment if require: \path (0,300); %set diagram left start at 0, and has height of 300

		%Shape: Triangle [id:dp6067991823078811] 
		\draw   (366.19,130.38) -- (434.07,241.82) -- (257.92,241.82) -- cycle ;
		%Straight Lines [id:da31023092987383727] 
		\draw    (363.17,199.64) -- (434.07,241.82) ;
		%Straight Lines [id:da2190692332387303] 
		\draw    (363.17,199.64) -- (356.27,211.67) ;

		% Text Node
		\draw (358.15,114.63) node [anchor=north west][inner sep=0.75pt]    {$A$};
		% Text Node
		\draw (244.94,236) node [anchor=north west][inner sep=0.75pt]    {$B$};
		% Text Node
		\draw (435.24,237.48) node [anchor=north west][inner sep=0.75pt]    {$C$};
		% Text Node
		\draw (355.47,184.16) node [anchor=north west][inner sep=0.75pt]    {$I$};
		% Text Node
		\draw (340.98,199.99) node [anchor=north west][inner sep=0.75pt]    {$G$};


	\end{centertikzpicture}\captionof{figure}{}


\end{question}
\begin{analysis}
	过 $G I$ 的直线截得等腰 $\triangle P Q C$,利用重心和内心的性质将 $\triangle P Q C$ 的面积算两次,再.进行分析.
\end{analysis}
\begin{solution}
	过 $G I$ 的直线与边 $B C ,  C A$ 分别交于点 $P ,  Q$. 过 $G$ 作 $G E \perp B C, G F \perp A C$, 垂足分别为 $E ,  F$, 连结 $G C$.

	设 $\triangle A B C$ 的内切圆的半径为 $r, B C ,  C A$ 边上的高的长分别为 $h_a, h_b$.

	因为 $I$ 为 $\triangle A B C$ 的内心, 所以 $\angle P C I=\angle Q C I$.

	由题意: $\angle Q I C=\angle P I C=90^{\circ}$, 故 $\triangle P I C \cong \triangle Q I C$, 所以 $P C=Q C$.

	一方面: $S_{\triangle P Q C}=S_{\triangle P C C}+S_{\triangle Q I C}=\frac{1}{2} \cdot P C \cdot r+\frac{1}{2} \cdot Q C \cdot r$,

	另一方面: $S_{\triangle P Q C}=S_{\triangle G P C}+S_{\triangle G Q C}=\frac{1}{2} \cdot P C \cdot(G E+G F)$,

	故 $2 r=G E+G F=\frac{1}{3}\left(h_a+h_b\right)$.

	故 $\frac{4 S_{\triangle A B C}}{a+b+c}=\frac{1}{3} \cdot\left(\frac{2 S_{\triangle A B C}}{a}+\frac{2 S_{\triangle A B C}}{b}\right)$, 所以 $a+b+c=\frac{6 a b}{a+b}$.

	因为 $\triangle A B C$ 的重心 $G$ 和内心 $I$ 不重合, 故 $\triangle A B C$ 不是正三角形.

	若 $a=b$, 则 $a=b=2, c=2$, 矛盾!

	从而 $a \neq b$, 不妨设 $a>b, a=2 a_1, b=2 b_1,\left(a_1, b_1\right)=1$. 由 $\frac{6 a b}{a+b}=$ $\frac{12 a_1 b_1}{a_1+b_1}$ 为整数知, $a_1+b_1 \mid 12$, 即 $a+b \mid 24$. 因此 $a+b=6,8,12,24$.

	经检验, 只有当 $a=14, b=10, c=11$ 时满足条件, 故 $a+b+c=35$,即 $\triangle A B C$ 的周长为 35 .
\end{solution}
\begin{note}
	本题虽然没有直接涉及面积, 但借助于图形面积知识解答, 往往事半功倍.
\end{note}

\subsection{练习题}
\begin{exercise}
	如图,  已知正方形 $A B C D$ 的面积为 35 平方厘米, $E ,  F$ 分别为边 $A B ,  B C$上的点, $A F$ 与 $C E$ 相交于点 $G$, 并且 $\triangle A B F$ 的面积为 5 平方厘米, $\triangle B C E$ 的面积为 14 平方厘米. 求四边形 $BEGF$ 的面积.


	\tikzset{every picture/.style={line width=0.75pt}} %set default line width to 0.75pt        


	\tikzset{every picture/.style={line width=0.75pt}} %set default line width to 0.75pt        

	\begin{centertikzpicture}[x=0.75pt,y=0.75pt,yscale=-1,xscale=1]
		%uncomment if require: \path (0,300); %set diagram left start at 0, and has height of 300

		%Shape: Square [id:dp8170208926679419] 
		\draw   (285.93,107) -- (407,107) -- (407,228.07) -- (285.93,228.07) -- cycle ;
		%Straight Lines [id:da8823531898627133] 
		\draw    (407,107) -- (316.57,227.61) ;
		%Straight Lines [id:da8679275246403186] 
		\draw    (407.07,190.11) -- (285.93,228.07) ;

		% Text Node
		\draw (272,225.4) node [anchor=north west][inner sep=0.75pt]    {$A$};
		% Text Node
		\draw (406.5,227.43) node [anchor=north west][inner sep=0.75pt]    {$B$};
		% Text Node
		\draw (311,229.43) node [anchor=north west][inner sep=0.75pt]    {$E$};
		% Text Node
		\draw (271,93.43) node [anchor=north west][inner sep=0.75pt]    {$D$};
		% Text Node
		\draw (406.5,90.93) node [anchor=north west][inner sep=0.75pt]    {$C$};
		% Text Node
		\draw (312,198.43) node [anchor=north west][inner sep=0.75pt]    {$G$};
		% Text Node
		\draw (409,178.4) node [anchor=north west][inner sep=0.75pt]    {$F$};


	\end{centertikzpicture}\captionof{figure}{}




\end{exercise}
\begin{solution}
	连结 $A C ,  B G$.

	故 $\frac{B F}{B C}=\frac{S_{\triangle A B F}}{S_{\triangle A B C}}=\frac{5}{\frac{35}{2}}=\frac{2}{7}$.

	同理: $\frac{B E}{B A}=\frac{4}{5}$.

	设 $S_{\triangle A G E}=a, S_{\triangle E B G}=b, S_{\triangle B G F}=c, S_{\triangle F G C}=d$.

	则 $a+b+c=5, b+c+d=14$.

	$\frac{c}{c+d}=\frac{B F}{B C}=\frac{2}{7}, \frac{b}{a+b}=\frac{B E}{A B}=\frac{4}{5}$.

	解得: $a=\frac{7}{27}, b=\frac{28}{27}, c=\frac{100}{27}, d=\frac{250}{27}$.

	所以 $S_{B E G F}=b+c=4 \frac{20}{27}$ (平方厘米).
\end{solution}

\begin{exercise}
	如图,  点 $M$ 和 $N$ 三等分 $A C$, 点 $X$ 和 $Y$ 三等分 $B C, A Y$ 与 $B M ,  B N$ 分别交于点 $S ,  R$. 求四边形 $S R N M$ 的面积与 $\triangle A B C$ 的面积之比.

	\tikzset{every picture/.style={line width=0.75pt}} %set default line width to 0.75pt        

	\begin{centertikzpicture}[x=0.75pt,y=0.75pt,yscale=-1,xscale=1]
		%uncomment if require: \path (0,300); %set diagram left start at 0, and has height of 300

		%Shape: Triangle [id:dp6830656203714423] 
		\draw   (364.12,89.5) -- (413.1,222.4) -- (260,222.4) -- cycle ;
		%Straight Lines [id:da46389229387602593] 
		\draw    (364.12,89.5) -- (308.29,221.82) ;
		%Straight Lines [id:da8404558057185993] 
		\draw    (364.12,89.5) -- (358.79,221.82) ;
		%Straight Lines [id:da9445802612457177] 
		\draw    (379.79,130.82) -- (260,222.4) ;
		%Straight Lines [id:da9931991026526179] 
		\draw    (260,222.4) -- (395.79,174.82) ;

		% Text Node
		\draw (356.71,73.11) node [anchor=north west][inner sep=0.75pt]    {$A$};
		% Text Node
		\draw (245.71,217.61) node [anchor=north west][inner sep=0.75pt]    {$B$};
		% Text Node
		\draw (414.71,216.11) node [anchor=north west][inner sep=0.75pt]    {$C$};
		% Text Node
		\draw (380.71,117.61) node [anchor=north west][inner sep=0.75pt]    {$M$};
		% Text Node
		\draw (398.21,164.11) node [anchor=north west][inner sep=0.75pt]    {$N$};
		% Text Node
		\draw (299.21,223.61) node [anchor=north west][inner sep=0.75pt]    {$X$};
		% Text Node
		\draw (352.21,224.61) node [anchor=north west][inner sep=0.75pt]    {$Y$};
		% Text Node
		\draw (346.69,172.46) node [anchor=north west][inner sep=0.75pt]    {$R$};
		% Text Node
		\draw (350.69,131.06) node [anchor=north west][inner sep=0.75pt]    {$S$};


	\end{centertikzpicture}\captionof{figure}{}


\end{exercise}
\begin{solution}
	连结 $R C ,  R M ,  R X$.

	设
	$$
		S_{\triangle R C N}=a, S_{\triangle R C Y}=b .
	$$

	则
	\begin{align*}
		S_{\triangle R A M} & =S_{\triangle R M N}=S_{\triangle R C N}=a,  \\
		S_{\triangle R B X} & =S_{\triangle R X Y}=S_{\triangle R Y C}=b .
	\end{align*}

	故
	\begin{align*}
		S_{\triangle A B C} & =3 S_{\triangle A Y C}=3(3 a+b),  \\
		S_{\triangle A B C} & =3 S_{\triangle B N C}=3(3 b+a) .
	\end{align*}

	所以 $a=b, S_{\triangle A B C}=12 a$.

	设 $S_{\triangle S M R}=x, S_{\triangle A S M}=y, S_{\triangle B S R}=x_1, S_{\triangle B S A}=y_1$.

	故 $\frac{y_1}{x_1}=\frac{y}{x}, \frac{x_1+y_1}{x_1}=\frac{x+y}{x}$, 故 $\frac{6 a}{x_1}=\frac{a}{x}$, 故 $x_1=6 x$, 故 $6 x+x+a=$ $4 a$, 故 $x=\frac{3}{7} a$.

	所以 $S_{S R N M}=x+a=\frac{10}{7} a$. 所以 $S_{S R N M}: S_{\triangle A B C}=\frac{10}{7} a: 12 a=5: 42$.
\end{solution}

\begin{exercise}
	设 $\triangle A B C$ 三边上的三个内接正方形(有两个顶点在三角形的一边上, 另两个顶点分别在三角形另两边上) 的面积都相等. 证明: $\triangle A B C$ 为正三角形.
\end{exercise}
\begin{proof}
	如图, 设$ \triangle A B C $的面积为 $ S, B C $边上的高为 $ h_a $, 一边在 $ B C $边上的内接正方形的边长为 $x$.

	\tikzset{every picture/.style={line width=0.75pt}} %set default line width to 0.75pt        

	\begin{centertikzpicture}[x=0.75pt,y=0.75pt,yscale=-1,xscale=1]
		%uncomment if require: \path (0,300); %set diagram left start at 0, and has height of 300

		%Shape: Triangle [id:dp14761293695152755] 
		\draw   (315.59,60.16) -- (389.16,187.74) -- (242.02,187.74) -- cycle ;
		%Shape: Rectangle [id:dp7184348836137224] 
		\draw   (282.17,119.98) -- (349.77,119.98) -- (349.77,187.58) -- (282.17,187.58) -- cycle ;
		%Straight Lines [id:da07831174514074868] 
		\draw  [dash pattern={on 4.5pt off 4.5pt}]  (315.59,60.16) -- (315.82,187.22) ;

		% Text Node
		\draw (224.08,181.41) node [anchor=north west][inner sep=0.75pt]    {$B$};
		% Text Node
		\draw (263.77,106.19) node [anchor=north west][inner sep=0.75pt]    {$B'$};
		% Text Node
		\draw (323.38,191.72) node [anchor=north west][inner sep=0.75pt]    {$a$};
		% Text Node
		\draw (307.73,40.13) node [anchor=north west][inner sep=0.75pt]    {$A$};
		% Text Node
		\draw (389.9,185.77) node [anchor=north west][inner sep=0.75pt]    {$C$};
		% Text Node
		\draw (352.45,105.84) node [anchor=north west][inner sep=0.75pt]    {$C'$};
		% Text Node
		\draw (298.89,102.74) node [anchor=north west][inner sep=0.75pt]    {$x$};
		% Text Node
		\draw (318.1,134.23) node [anchor=north west][inner sep=0.75pt]    {$h_{a}$};
	\end{centertikzpicture}\captionof{figure}{}

	由题意: $B^{\prime} C^{\prime}\pll B C$, 则 $\triangle A B^{\prime} C^{\prime} \backsim \triangle A B C$.

	故 $\frac{x}{a}=\frac{h_a-x}{h_a}$, 所以 $x=\frac{a h_a}{a+h_a}=\frac{2 S}{a+h_a}$.

	同理可得另两种放法的内接正方形的边长.由于三个内接正方形面积相等,

	则 $a+h_a=b+h_b=c+h_c$.

	则 $a+\frac{2 S}{a}=b+\frac{2 S}{b}=c+\frac{2 S}{c}$, 记此值为 $l$.

	所以正数 $a ,  b ,  c$ 适合方程 $y+\frac{2 S}{y}=l$.

	当 $y \neq 0$ 时, 有 $y^2-l y+2 S=0$, 但任何二次方程至多有两个相异的根.

	所以 $a ,  b ,  c$ 中必有两数相同, 不妨设 $a=b$.

	若 $a \neq c$, 则 $a-c=\frac{2 S}{a c}(a-c)$.

	所以 $a c=2 S=a h_a$, 所以 $c=h_a$, 所以 $\triangle A B C$ 是以 $\angle B$ 为直角的直角三角形,且 $b$ 为斜边.

	因此 $b>a$, 矛盾!

	故 $b=a=c$, 所以 $\triangle A B C$ 为正三角形.
\end{proof}

\newpage
\section{平移, 旋转与翻折}
与代数变换的重要性一样, 几何变换同样在几何问题的解决中也起着非常重要的作用. 通过几何变换, 可以把分散的线段, 角相对集中起来, 从而使已知条件集中在一个我们所熟知的基本图形之中, 然后利用新的图形的性质对原图形进行研究, 从而使问题得以转化.

\subsection{例题}
\begin{question}
	如图, 设 $I$ 是 $\triangle A B C$ 的垂心. 求证: $A I^2+B C^2=B I^2+A C^2=C I^2+A B^2$.


	\tikzset{every picture/.style={line width=0.75pt}} %set default line width to 0.75pt        

	\begin{centertikzpicture}[x=0.75pt,y=0.75pt,yscale=-1,xscale=1]
		%uncomment if require: \path (0,300); %set diagram left start at 0, and has height of 300

		%Shape: Triangle [id:dp965134050283331] 
		\draw   (292.18,88.14) -- (418.93,206.14) -- (238.43,206.14) -- cycle ;
		%Straight Lines [id:da10632431868047854] 
		\draw    (292.18,88.14) -- (293,206.39) ;
		%Straight Lines [id:da9245616532823671] 
		\draw    (238.43,206.14) -- (322,116.39) ;
		%Straight Lines [id:da28870816990987525] 
		\draw    (270,136.89) -- (418.93,206.14) ;

		% Text Node
		\draw (283.43,70.19) node [anchor=north west][inner sep=0.75pt]    {$A$};
		% Text Node
		\draw (224.43,204.47) node [anchor=north west][inner sep=0.75pt]    {$B$};
		% Text Node
		\draw (419.43,203.47) node [anchor=north west][inner sep=0.75pt]    {$C$};
		% Text Node
		\draw (284.43,208.61) node [anchor=north west][inner sep=0.75pt]    {$M$};
		% Text Node
		\draw (322.93,100.61) node [anchor=north west][inner sep=0.75pt]    {$L$};
		% Text Node
		\draw (253.43,122.61) node [anchor=north west][inner sep=0.75pt]    {$N$};
		% Text Node
		\draw (294.09,123.67) node [anchor=north west][inner sep=0.75pt]    {$I$};


	\end{centertikzpicture}\captionof{figure}{}

\end{question}
\begin{analysis}
	对于 $\triangle A B C$ 各边来说,结论是轮换式,于是只需证得某一个等式即可. 显然等式每边都是两线段的平方和,故可考虑构造相应的直角三角形.
\end{analysis}
\begin{proof}
	分别过点 $B ,  I$ 作 $A I ,  A B$ 的平行线,两线交于 $P$ 点,连结 $P C$.

	由条件可知: 四边形 $A I P B$ 为平行四边形, 从而 $B P=A I, \angle P B C=\angle A L B=90^{\circ}$, 所以
	$$
		P C^2=P B^2+B C^2=A I^2+B C^2 ;
	$$

	同理: $P I=A B, \angle P I C=\angle B N C=90^{\circ}$.

	故 $P C^2=C I^2+P I^2=C I^2+A B^2$, 所以 $A I^2+B C^2=C I^2+A B^2$;

	同理: $A I^2+B C^2=B I^2+A C^2$.

	所以 $A I^2+B C^2=B I^2+A C^2=C I^2+A B^2$.
\end{proof}

\begin{question}
	已知 $\triangle A B C$ 的三条中线的长为 $3 ,  4 ,  5$. 求 $\triangle A B C$ 的面积.


	\tikzset{every picture/.style={line width=0.75pt}} %set default line width to 0.75pt        

	\begin{centertikzpicture}[x=0.75pt,y=0.75pt,yscale=-1,xscale=1]
		%uncomment if require: \path (0,300); %set diagram left start at 0, and has height of 300

		%Shape: Triangle [id:dp11486446469283806] 
		\draw   (284.64,90.55) -- (431.11,208.02) -- (240.57,208.02) -- cycle ;
		%Straight Lines [id:da08764497625308065] 
		\draw    (284.63,90.55) -- (337.35,207.76) ;
		%Straight Lines [id:da3654952452511284] 
		\draw    (240.6,208.02) -- (356.44,148.62) ;
		%Straight Lines [id:da5369841766242902] 
		\draw    (263.24,147.76) -- (431.11,208.02) ;

		% Text Node
		\draw (223.27,204.89) node [anchor=north west][inner sep=0.75pt]    {$A$};
		% Text Node
		\draw (275.23,73.38) node [anchor=north west][inner sep=0.75pt]    {$B$};
		% Text Node
		\draw (435.71,201.95) node [anchor=north west][inner sep=0.75pt]    {$C$};
		% Text Node
		\draw (357.66,129.27) node [anchor=north west][inner sep=0.75pt]    {$D$};
		% Text Node
		\draw (329.33,211.32) node [anchor=north west][inner sep=0.75pt]    {$E$};
		% Text Node
		\draw (246.41,134.89) node [anchor=north west][inner sep=0.75pt]    {$F$};
		% Text Node
		\draw (315.32,142.82) node [anchor=north west][inner sep=0.75pt]    {$G$};


	\end{centertikzpicture}\captionof{figure}{}

\end{question}
\begin{analysis}
	设 $\triangle A B C$ 的中线 $A D=3$, $B E=4, C F=5$. 现考虑通过平移使二条中线集中在一起, 构成一个确定的三角形,再分析面积间的数量关系.
\end{analysis}
\begin{solution}[法一]
	过 $F$ 作 $F K \pxqdy E C$, 连结 $K A$ ,  $K E ,  K B ,  E F$.

	从而 $K F \pxqdy E C$, 所以四边形 $K F C E$ 为平行四边形, 进而 $K F=E C$.

	又因为 $E C=A E$, 所以 $K F \pxqdy A E$, 从而四边形 $K A E F$ 为平行四边形, 所以 $K A \pxqdy E F$.

	因此四边形 $B K A D$ 为平行四边形. 从而 $B K=A D$.

	所以 $\triangle B K E$ 是以 $\triangle A B C$ 三边中线长为边长的三角形.

	因为 $5^2=3^2+4^2$, 所以 $\triangle B K E$ 为直角三角形, $S_{\triangle B K E}=6$.

	因为 $S_{\triangle B F K}=\frac{1}{4} S_{A D B K}=\frac{1}{4} S_{\triangle A B C}, S_{\triangle B E F}=\frac{1}{4} S_{\triangle A B C}, S_{\triangle E F K}=S_{\triangle A E F}=$ $\frac{1}{4} S_{\triangle A B C}$.

	从而 $S_{\triangle B K E}=\frac{3}{4} S_{\triangle A B C}, S_{\triangle A B C}=8$.
\end{solution}
\begin{note}
	可将问题一般化, 设 $\triangle A B C$ 三边上中线的长度分别为 $m_a, m_b$, $m_c$, 则有:
	$$
		S_{\triangle A B C}=\frac{4}{3} \sqrt{p\left(p-m_a\right)\left(p-m_b\right)\left(p-m_c\right)},
	$$

	其中 $p=\frac{1}{2}\left(m_a+m_b+m_c\right)$.
\end{note}
\begin{solution}[法二]
	利用重心的性质, 构造出三角形, 结合以 $A D ,  B E ,  C F$ 长为边的三角形与其相似,再求解面积.

	延长 $G D$ 至点 $H$, 使 $G D=H D$, 连结 $H C$.

	易证: $\triangle B D G \cong \triangle C D H$, 故 $B G=C H=\frac{2}{3} B E$.

	因为 $C G=\frac{2}{3} C F, G H=2 G D=\frac{2}{3} A D$, 故 $C G^2=H G^2+H C^2$, 因此 $S_{\triangle G H C}=\left(\frac{2}{3}\right)^2 \times 6=\frac{8}{3}$, 所以 $S_{\triangle C D G}=\frac{4}{3}$.

	又因为 $G$ 为 $\triangle A B C$ 的重心, 所以 $S_{\triangle A B C}=8$.
\end{solution}
\begin{note}
	由以上两个例题可知, 图形经过适当的平移可以使已知条件和结论中的图形元素得以延伸,再通过某些桥梁得以联系和统一.
\end{note}

\begin{question}
	如图, 在 $\triangle A B C$ 外作等腰 Rt $\triangle A B D$ 和等腰 Rt $\triangle A C E$, 且 $\angle B A D=\angle C A E=90^{\circ}, A M$ 为 $\triangle A B C$ 中 $B C$ 边上的中线, 连结 $D E$. 求证: $D E=2 A M$.


	\tikzset{every picture/.style={line width=0.75pt}} %set default line width to 0.75pt        

	\begin{centertikzpicture}[x=0.75pt,y=0.75pt,yscale=-1,xscale=1]
		%uncomment if require: \path (0,300); %set diagram left start at 0, and has height of 300

		%Shape: Right Triangle [id:dp1498948318987876] 
		\draw   (282.19,190.42) -- (322.24,66.21) -- (364.32,148.34) -- cycle ;
		%Shape: Right Triangle [id:dp4973273175953059] 
		\draw   (406.6,97.77) -- (414.89,190.62) -- (364.32,148.34) -- cycle ;
		%Straight Lines [id:da47439954537309914] 
		\draw    (414.89,190.62) -- (282.19,190.42) ;
		%Straight Lines [id:da14263887682479703] 
		\draw    (322.24,66.21) -- (406.6,97.77) ;
		%Straight Lines [id:da7282590777299278] 
		\draw    (364.32,148.34) -- (348.54,190.52) ;

		% Text Node
		\draw (369.29,136.04) node [anchor=north west][inner sep=0.75pt]    {$A$};
		% Text Node
		\draw (269.29,189.04) node [anchor=north west][inner sep=0.75pt]    {$B$};
		% Text Node
		\draw (415.79,188.04) node [anchor=north west][inner sep=0.75pt]    {$C$};
		% Text Node
		\draw (307.29,53.04) node [anchor=north west][inner sep=0.75pt]    {$D$};
		% Text Node
		\draw (410.79,84.04) node [anchor=north west][inner sep=0.75pt]    {$E$};
		% Text Node
		\draw (336.79,193.04) node [anchor=north west][inner sep=0.75pt]    {$M$};


	\end{centertikzpicture}\captionof{figure}{}

\end{question}
\begin{analysis}
	由 $M$ 为 $B C$ 中点, 可利用中线加倍得到 $A F=2 A M$, 再证 $A F=D E$. 注意到 $\triangle B M F$ 可通过 $\triangle C M A$ 旋转得到, 故可利用图形旋转相关性质思考:
\end{analysis}
\begin{proof}
	延长 $A M$ 至 $F$, 使 $A M=F M$, 连结 $B F$.
	因为 $B M=C M, A M=F M, \angle A M C=\angle F M B$,所以 $\triangle A M C \cong \triangle F M B$.
	从而 $\angle F=\angle M A C, B F=A C$, 所以
	$$
		\begin{aligned}
			\angle A B F & =180^{\circ}-\angle B A F-\angle F                                \\
			             & =180^{\circ}-\angle B A F-\angle M A C,                           \\
			             & =180^{\circ}-\left(180^{\circ}-\angle D A E\right)=\angle D A E .
		\end{aligned}
	$$
	又因为 $A D=A B, A E=A C$, 故 $A E=B F$.
	因而 $\triangle A D E \cong \triangle B A F, D E=A F=2 A M$.
\end{proof}
\begin{note}
	图形的旋转可通过构造全等来实现,旋转变换即全等变换的一种, 在几何证明题中, 常常以图形旋转的观点来研究, 是常用策略.
\end{note}

\begin{question}
	如图, 正方形 $A B C D$ 内一点 $E, E$ 到 $A,  B,  C$ 三点的距离之和的最小值为 $\sqrt{2}+\sqrt{6}$, 求此正方形的边长.


	\tikzset{every picture/.style={line width=0.75pt}} %set default line width to 0.75pt        

	\begin{centertikzpicture}[x=0.75pt,y=0.75pt,yscale=-1,xscale=1]
		%uncomment if require: \path (0,300); %set diagram left start at 0, and has height of 300

		%Shape: Square [id:dp9978137376670462] 
		\draw   (268,92) -- (378.43,92) -- (378.43,202.43) -- (268,202.43) -- cycle ;
		%Straight Lines [id:da30319941544272133] 
		\draw    (268,92) -- (299.43,161.86) ;
		%Straight Lines [id:da2229572263406452] 
		\draw    (299.43,161.86) -- (378.43,202.43) ;
		%Straight Lines [id:da03618190876208338] 
		\draw    (268,202.43) -- (299.43,161.86) ;

		% Text Node
		\draw (252.93,78.26) node [anchor=north west][inner sep=0.75pt]    {$A$};
		% Text Node
		\draw (256.93,202.76) node [anchor=north west][inner sep=0.75pt]    {$B$};
		% Text Node
		\draw (378.43,199.76) node [anchor=north west][inner sep=0.75pt]    {$C$};
		% Text Node
		\draw (378.93,78.26) node [anchor=north west][inner sep=0.75pt]    {$D$};
		% Text Node
		\draw (299.43,145.54) node [anchor=north west][inner sep=0.75pt]    {$E$};


	\end{centertikzpicture}\captionof{figure}{}

\end{question}
\begin{analysis}
	利用图形的旋转,将三条线段转换为首尾顺次连接,再利用两点间线段最短求最值问题.
\end{analysis}
\begin{solution}
	将 $\triangle A B E$ 绕 $A$ 点顺时针旋转 $60^{\circ}$ 到 $\triangle A M N$, 连结 $N E, M B$. 过 $M$ 作 $M P \perp B C$, 垂足为 $P$.

	由题意: $A E=A N, \angle N A E=60^{\circ}$, 从而 $\triangle A N E$ 为等边三角形. 故 $A E=N E$,所以
	$$
		A E+B E+C E=M N+N E+E C .
	$$

	当 $A E+B E+C E$ 最小时,折线 $M N E C$ 为线段, 且 $M C=\sqrt{2}+\sqrt{6}$;
	同理: $\triangle A M B$ 为等边三角形.
	所以 $A B=B C=B M, \angle M B A=60^{\circ}$.
	设 $A B=x$, 则 $P M=\frac{1}{2} x, P B=\frac{\sqrt{3}}{2} x$, 在 Rt $\triangle M P C$ 中:
	$$
		(\sqrt{2}+\sqrt{6})^2=\left(\frac{1}{2} x\right)^2+\left(\frac{\sqrt{3}}{2} x+x\right)^2,
	$$

	解得: $x=2$, 即 $A B=2$, 正方形边长为 2.
\end{solution}

\begin{question}
	如图, 在正 $\triangle A B C$ 内有一点 $P, P$ 到三个顶点 $A,  B,  C$ 的距离分别为 $a,  b,  c$, 求 $\triangle A B C$ 的面积.


	\tikzset{every picture/.style={line width=0.75pt}} %set default line width to 0.75pt        

	\begin{centertikzpicture}[x=0.75pt,y=0.75pt,yscale=-1,xscale=1]
		%uncomment if require: \path (0,300); %set diagram left start at 0, and has height of 300

		%Shape: Triangle [id:dp9383543874437772] 
		\draw   (349.48,96.34) -- (426.95,232.4) -- (272,232.4) -- cycle ;
		%Straight Lines [id:da05751437608273635] 
		\draw    (357.43,170.64) -- (272,232.4) ;
		%Straight Lines [id:da5548462089665349] 
		\draw    (349.48,96.34) -- (357.43,170.64) ;
		%Straight Lines [id:da3973572948994557] 
		\draw    (426.95,232.4) -- (357.43,170.64) ;

		% Text Node
		\draw (342.71,79.4) node [anchor=north west][inner sep=0.75pt]    {$A$};
		% Text Node
		\draw (258.71,229.4) node [anchor=north west][inner sep=0.75pt]    {$B$};
		% Text Node
		\draw (426.71,228.9) node [anchor=north west][inner sep=0.75pt]    {$C$};
		% Text Node
		\draw (351.43,179.04) node [anchor=north west][inner sep=0.75pt]    {$P$};


	\end{centertikzpicture}\captionof{figure}{}

\end{question}
\begin{analysis}
	由于 $A P,  B P,  C P$ 为已知, 故可将 $A P,  B P,  C P$ 移至一个三角形. 为此可分别旋转 $\triangle A B P,  \triangle B P C,  \triangle P C A$, 通过求六边形 $A F B D C E$的面积求解 $\triangle A B C$ 的面积.
\end{analysis}
\begin{solution}
	分别将 $\triangle A B P,  \triangle B C P,  \triangle C A P$ 绕着 $B$,  $C,  A$ 顺时针旋转 $60^{\circ}$, 得 到 $\triangle C B D,  \triangle A C E$,  $\triangle B A F$. 连结 $P D,  P E,  P F$.
	故 $S_{\text {六边形 } A F B D C E}=2 S_{\triangle A B C}$.

	因为 $A P=A F, \angle P A F=60^{\circ}$, 所以 $\triangle A P F$ 为等边三角形, 边长为 $a$. 从而 $P F=A P$.
	因为 $B F=P C$, 故 $\triangle B P F$ 的三边长分别为 $a, b, c$;
	同理: $\triangle B P D$ 是边长为 $b$ 的等边三角形, $\triangle C P E$ 是边长为 $c$ 的等边三角形,
	由于 $\triangle D P C$ 和 $\triangle E P A$ 的三边长都分别为 $a, b, c$.
	因此 $S_{\text {六边形} A F B D C E}=\frac{\sqrt{3}}{4}\left(a^2+b^2+c^2\right)+3 \sqrt{p(p-a)(p-b)(p-c)}$.
	所以 $S_{\triangle A B C}=\frac{\sqrt{3}}{8}\left(a^2+b^2+c^2\right)+\frac{3}{2} \sqrt{p(p-a)(p-b)(p-c)}$, 其中 $p=\frac{1}{2}(a+b+c)$.
\end{solution}
\begin{note}
	利用图形的旋转可构造等边三角形或等腰直角三角形, 实现线段和角度的转换, 使原来分散的线段和角集中起来或有序地排列起来, 得到新的图形以方便研究.
\end{note}

\begin{question}
	如图, 在 $\triangle A B C$ 中, $A D$ 是角平分线, $B E=C F$, 点 $M,  N$分别是 $B C$ 和 $E F$ 的中点. 求证: $M N \pll A D$.


	\tikzset{every picture/.style={line width=0.75pt}} %set default line width to 0.75pt        

	\begin{centertikzpicture}[x=0.75pt,y=0.75pt,yscale=-1,xscale=1]
		%uncomment if require: \path (0,300); %set diagram left start at 0, and has height of 300

		%Shape: Triangle [id:dp9755673279993666] 
		\draw   (299.3,88.1) -- (422.92,223.45) -- (268.94,223.45) -- cycle ;
		%Straight Lines [id:da08876761533695965] 
		\draw    (299.3,88.1) -- (332.61,223.28) ;
		%Straight Lines [id:da47111979754735467] 
		\draw    (281.68,168.47) -- (386.86,183.42) ;
		%Straight Lines [id:da25266754246056355] 
		\draw    (334.27,175.95) -- (346.45,223.28) ;

		% Text Node
		\draw (291.88,71.47) node [anchor=north west][inner sep=0.75pt]    {$A$};
		% Text Node
		\draw (256.62,217.61) node [anchor=north west][inner sep=0.75pt]    {$B$};
		% Text Node
		\draw (424.27,218.72) node [anchor=north west][inner sep=0.75pt]    {$C$};
		% Text Node
		\draw (323.75,224.81) node [anchor=north west][inner sep=0.75pt]    {$D$};
		% Text Node
		\draw (342.67,225.48) node [anchor=north west][inner sep=0.75pt]    {$M$};
		% Text Node
		\draw (339.17,179.69) node [anchor=north west][inner sep=0.75pt]    {$N$};
		% Text Node
		\draw (267.47,157.99) node [anchor=north west][inner sep=0.75pt]    {$E$};
		% Text Node
		\draw (392.15,173.05) node [anchor=north west][inner sep=0.75pt]    {$F$};


	\end{centertikzpicture}\captionof{figure}{}

\end{question}
\begin{analysis}
	利用等腰三角形轴对称性构造中点,再结合条件中的中点,构造平行四边形来论证平行关系.
\end{analysis}
\begin{proof}
	过 $E$ 作 $A D$ 的垂线,交 $A D$ 于点 $G$,交 $A C$ 于点 $S$. 过 $B$ 作 $A D$ 的垂线, 交 $A D$ 于点 $H$, 交 $A C$ 于点 $T$. 连结 $G N, H M$.
	因为 $A D$ 平分 $\angle B A C$, 所以 $\angle E A G=\angle S A G$.
	因为 $A D \perp E S$, 故 $\angle A G E=\angle A G S=90^{\circ}$.
	所以 $\triangle A E G \cong \triangle A S G$.

	故 $A E=A S, E G=S G$.
	同理: $A B=A T, B H=T H$.
	又因为 $N$ 为 $E F$ 的中点, 故 $G N \pxqdy \frac{1}{2} S F$; 同理: $M H \pxqdy \frac{1}{2} C T$.
	因为 $B E=F C$, 所以 $S T=F C$, 从而 $S F=C T$, 进而 $G N \pxqdy H M$.
	故四边形 $G N M H$ 为平行四边形, 所以 $M N / / A D$.
\end{proof}
\begin{note}
	本题是把角平分线作为翻折轴来进行解题的. 用角平分线作为翻折轴, 可以使翻折图形落至原来图形的另一侧, 而且对应点的连线被角平分线垂直平分. 这样不仅可以增加图形的直观性, 而且增强条件与结论间的逻辑联系.
\end{note}

\begin{question}
	如图, 在矩形 $A B C D$ 中, $A B=20, B C=10$, 若在 $A B$,  $A C$ 上各取一点 $N,  M$, 使得 $B M+M N$ 的值最小, 求这个最小值.


	\tikzset{every picture/.style={line width=0.75pt}} %set default line width to 0.75pt        

	\begin{centertikzpicture}[x=0.75pt,y=0.75pt,yscale=-1,xscale=1]
		%uncomment if require: \path (0,300); %set diagram left start at 0, and has height of 300

		%Shape: Rectangle [id:dp6131445314932089] 
		\draw   (255,113.14) -- (403.43,113.14) -- (403.43,214) -- (255,214) -- cycle ;
		%Straight Lines [id:da3439469140463751] 
		\draw    (403.43,113.14) -- (255,214) ;
		%Straight Lines [id:da6030236593719192] 
		\draw    (305.71,179.14) -- (403.43,214) ;
		%Straight Lines [id:da05143138056341101] 
		\draw    (305.71,179.14) -- (293.31,213.54) ;

		% Text Node
		\draw (239.14,210.11) node [anchor=north west][inner sep=0.75pt]    {$A$};
		% Text Node
		\draw (403.81,209.11) node [anchor=north west][inner sep=0.75pt]    {$B$};
		% Text Node
		\draw (405.81,101.78) node [anchor=north west][inner sep=0.75pt]    {$C$};
		% Text Node
		\draw (238.48,101.11) node [anchor=north west][inner sep=0.75pt]    {$D$};
		% Text Node
		\draw (285.81,164.78) node [anchor=north west][inner sep=0.75pt]    {$M$};
		% Text Node
		\draw (283.14,215.45) node [anchor=north west][inner sep=0.75pt]    {$N$};


	\end{centertikzpicture}\captionof{figure}{}

\end{question}
\begin{analysis}
	作 $B$ 关于 $A C$ 的对称点 $E$, 即求折线 $E M N$ 的最小值,这个最小值为点 $E$ 到 $A B$ 垂线段的距离.
\end{analysis}
\begin{solution}
	作 $B$ 关于直线 $A C$ 的对称点 $E$, 连结 $A E$,  $B E,  M E$. 过点 $E$ 作 $A B$ 的垂线, 交 $A B,  A C$ 于点 $F,  G$.

	由 $B M+M N=E M+M N \geqslant E F$, 所以 $B M+$ $M N$ 的最小值为 $E F$.

	因为 $2 S_{\triangle A B C}=A B \cdot B C=A C \cdot B H$, 故 $B H=4 \sqrt{5}, B E=2 B H=8 \sqrt{5}$.
	所以 $A H=\sqrt{A B^2-B H^2}=8 \sqrt{5}$.
	因为 $2 S_{\triangle A B E}=A B \cdot E F=B E \cdot A H$, 故 $E F=16$.
	所以 $B M+M N$ 的最小值为 16 , 此时 $N$ 和 $F$ 重合, $M$ 和 $G$ 重合.
\end{solution}

\begin{question}
	如图, 在 $\triangle A B C$ 中, $\angle A B C=90^{\circ}, A B=B C, P$ 为三角形内一点, 分别作 $P$ 关于 $B C,  C A,  A B$ 的对称点 $A^{\prime},  B^{\prime},  C^{\prime}$. 若所得 $\triangle A^{\prime} B^{\prime} C^{\prime}$ 中, $\angle B^{\prime} A^{\prime} C^{\prime}=90^{\circ}, A^{\prime} B^{\prime}=A^{\prime} C^{\prime}$. 求: $S_{\triangle A^{\prime} B^{\prime} C^{\prime}}: S_{\triangle A B C}$ 的值.


	\tikzset{every picture/.style={line width=0.75pt}} %set default line width to 0.75pt        

	\begin{centertikzpicture}[x=0.75pt,y=0.75pt,yscale=-1,xscale=1]
		%uncomment if require: \path (0,300); %set diagram left start at 0, and has height of 300

		%Shape: Right Triangle [id:dp516723719057484] 
		\draw   (269,108) -- (392.43,221) -- (269,221) -- cycle ;
		%Shape: Right Triangle [id:dp8912449963684967] 
		\draw   (221,198.46) -- (356.29,135.65) -- (307.74,239.15) -- cycle ;
		%Straight Lines [id:da45189573896432655] 
		\draw    (221,198.46) -- (307.36,198.66) ;
		%Straight Lines [id:da8084731051747136] 
		\draw    (307.36,198.66) -- (307.74,239.15) ;
		%Straight Lines [id:da05247831177520701] 
		\draw    (307.36,198.66) -- (356.29,135.65) ;

		% Text Node
		\draw (261.9,91.92) node [anchor=north west][inner sep=0.75pt]    {$A$};
		% Text Node
		\draw (257.9,220.59) node [anchor=north west][inner sep=0.75pt]    {$B$};
		% Text Node
		\draw (394.57,218.59) node [anchor=north west][inner sep=0.75pt]    {$C$};
		% Text Node
		\draw (297.24,239.59) node [anchor=north west][inner sep=0.75pt]    {$A'$};
		% Text Node
		\draw (358.57,120.26) node [anchor=north west][inner sep=0.75pt]    {$B'$};
		% Text Node
		\draw (206.57,186.59) node [anchor=north west][inner sep=0.75pt]    {$C'$};
		% Text Node
		\draw (296.57,179.59) node [anchor=north west][inner sep=0.75pt]    {$P$};


	\end{centertikzpicture}\captionof{figure}{}

\end{question}
\begin{analysis}
	由对称性可得的五边形面积是 $\triangle A B C$面积的两倍, 并且 $A C^{\prime} A^{\prime} B^{\prime}$ 是正方形. $\triangle A^{\prime} B^{\prime} C$ 为等腰直角三角形.
\end{analysis}
\begin{solution}
	连结 $P A,  P B,  P C,  A C^{\prime},  A B^{\prime},  B A^{\prime}, $ $B C^{\prime},  C A^{\prime},  C B^{\prime}$.
	由轴对称性知: $A C^{\prime}=A P=A B^{\prime}$.
	$\angle C^{\prime} A B=\angle P A B, \angle B^{\prime} A C=\angle P A C$.

	因为 $\angle A B C=90^{\circ}, A B=B C$, 所以 $\angle B A C=45^{\circ}$. 故 $\angle B^{\prime} A C^{\prime}=90^{\circ}$;同理: $\angle A^{\prime} C B^{\prime}=90^{\circ}, \angle A^{\prime} B C^{\prime}=180^{\circ}$, 故 $C^{\prime},  B,  A^{\prime}$ 三点共线.
	所以 $\triangle A B^{\prime} C^{\prime}$ 是等腰直角三角形.
	因为 $\triangle A^{\prime} B^{\prime} C^{\prime}$ 也是等腰直角三角形, 所以四边形 $A C^{\prime} A^{\prime} B^{\prime}$ 为正方形;同理: $\triangle A^{\prime} B^{\prime} C$ 为等腰直角三角形.
	设 $A^{\prime} B^{\prime}=a$, 故 $S_{A^{\prime} B^{\prime} A C^{\prime}}=a^2, S_{\triangle A^{\prime} B^{\prime} C}=\frac{1}{4} a^2$, 故 $S_{A^{\prime} C B^{\prime} A C^{\prime}}=\frac{5}{4} a^2$.
	由轴对称性质: $S_{A^{\prime} C B^{\prime} A C^{\prime}}=2 S_{\triangle A B C}$, 故 $S_{\triangle A B C}=\frac{5}{8} a^2$.
	故 $S_{\triangle A^{\prime} B^{\prime} C^{\prime}}: S_{\triangle A B C}=\frac{1}{2} a^2: \frac{5}{8} a^2=4: 5$.
\end{solution}

\subsection{练习题}
\begin{exercise}
	证明: 如果七条直线两两相交,那么所得的角中至少有一个角小于 $26^{\circ}$.
\end{exercise}
\begin{proof}
	在平面上任取一点 $P$, 过 $P$ 分别作给定的七条直线的平行线. 形成以 $P$ 为顶点的相邻的 14 个角. 如图: 设这 14 个角为 $\alpha_1, \alpha_2, \alpha_3, \cdots, \alpha_{13}, \alpha_{14}$. 这 14 个角分别等于七条直线两两相交所形成的 84 个角中的 14 个.
	如果每个 $\alpha_i \geqslant 26^{\circ}(i=1,2, \cdots, 14)$, 则
	$$
		\alpha_1+\alpha_2+\cdots+\alpha_{14} \geqslant 14 \times 26^{\circ}=364^{\circ}>360^{\circ},
	$$
	矛盾!
	故这 14 个角中至少有一个角小于 $26^{\circ}$, 原命题成立.
\end{proof}

\begin{exercise}
	如图, 在 “风车三角形” 中, $A A^{\prime}=B B^{\prime}=C C^{\prime}=2, \angle A O B^{\prime}=\angle B O C^{\prime}=$ $\angle C O A^{\prime}=60^{\circ}$. 求证: $S_{\triangle A O B^{\prime}}+S_{\triangle B O C^{\prime}}+S_{\triangle C O A^{\prime}}<\sqrt{3}$.


	\tikzset{every picture/.style={line width=0.75pt}} %set default line width to 0.75pt        

	\begin{centertikzpicture}[x=0.75pt,y=0.75pt,yscale=-1,xscale=1]
		%uncomment if require: \path (0,300); %set diagram left start at 0, and has height of 300

		%Straight Lines [id:da12467582263921573] 
		\draw    (256,177) -- (333.43,46.64) ;
		%Straight Lines [id:da9915168084381896] 
		\draw    (246,77) -- (324.43,205.64) ;
		%Straight Lines [id:da631099447911291] 
		\draw    (240.43,134.64) -- (402.43,135.86) ;
		%Straight Lines [id:da5501587979334464] 
		\draw    (246,77) -- (240.43,134.64) ;
		%Straight Lines [id:da590332993841753] 
		\draw    (333.43,46.64) -- (402.43,135.86) ;
		%Straight Lines [id:da8456949784719177] 
		\draw    (256,177) -- (324.43,205.64) ;

		% Text Node
		\draw (320.48,200.59) node [anchor=north west][inner sep=0.75pt]    {$A$};
		% Text Node
		\draw (227.14,126.59) node [anchor=north west][inner sep=0.75pt]    {$B$};
		% Text Node
		\draw (326.48,29.92) node [anchor=north west][inner sep=0.75pt]    {$C$};
		% Text Node
		\draw (237.14,167.92) node [anchor=north west][inner sep=0.75pt]    {$A'$};
		% Text Node
		\draw (400.48,128.92) node [anchor=north west][inner sep=0.75pt]    {$B'$};
		% Text Node
		\draw (236.48,58.92) node [anchor=north west][inner sep=0.75pt]    {$C'$};
		% Text Node
		\draw (274.48,108.26) node [anchor=north west][inner sep=0.75pt]    {$O$};


	\end{centertikzpicture}\captionof{figure}{}

\end{exercise}
\begin{proof}
	将 $\triangle B O C^{\prime}$ 沿 $B B^{\prime}$ 方向平移 2 个单位, 所移成的三角形记为 $\triangle B^{\prime} P R$;将 $\triangle A^{\prime} O C$ 沿 $A^{\prime} A$ 方向平移 2 个单位, 所移成的三角形记为 $\triangle A R Q$.
	因为
	$$
		\begin{aligned}
			 & O Q=O A+A Q=O A+O A^{\prime}=A A^{\prime}=2,                   \\
			 & O P=O B^{\prime}+B^{\prime} P=O B^{\prime}+O B=B B^{\prime}=2,
		\end{aligned}
	$$

	且 $\angle Q O P=60^{\circ}$, 所以 $\triangle Q O P$ 为正三角形.
	所以 $P Q=O Q=O P=2$.
	因为 $Q R+R P=O C+O C^{\prime}=C C^{\prime}=2$, 故 $Q ,  R ,  P$ 三点共线.
	所以 $S_{\triangle O P Q}=\frac{\sqrt{3}}{4} \times 2^2=\sqrt{3}$, 故 $S_{\triangle A O B^{\prime}}+S_{\triangle P R B^{\prime}}+S_{\triangle A Q R}<\sqrt{3}$, 所以 $S_{\triangle A O B^{\prime}}+S_{\triangle B O C^{\prime}}+S_{\triangle C O A^{\prime}}<\sqrt{3}$.
\end{proof}

\begin{exercise}
	如图, 在平行四边形 $A B C D$ 中, 由 $A$ 向另两边作垂线 $A P,  A Q$, 已知 $P Q=a, A C=b, H$ 为 $\triangle A P Q$ 的垂心. 求 $A H$ 的值.


	\tikzset{every picture/.style={line width=0.75pt}} %set default line width to 0.75pt        

	\begin{centertikzpicture}[x=0.75pt,y=0.75pt,yscale=-1,xscale=1]
		%uncomment if require: \path (0,300); %set diagram left start at 0, and has height of 300

		%Shape: Rectangle [id:dp5475311702784396] 
		\draw   (311.72,116.37) -- (450.71,116.37) -- (392.5,212.64) -- (253.51,212.64) -- cycle ;
		%Straight Lines [id:da38793245695055245] 
		\draw    (311.72,116.37) -- (312.19,211.98) ;
		%Straight Lines [id:da05366381726480629] 
		\draw    (311.72,116.37) -- (414.43,176.27) ;
		%Straight Lines [id:da39992472305633453] 
		\draw    (311.72,116.37) -- (392.5,212.64) ;
		%Straight Lines [id:da3076917283183087] 
		\draw    (312.19,211.98) -- (414.43,176.27) ;
		%Straight Lines [id:da9764842280121784] 
		\draw    (311.72,116.37) -- (337.38,202.93) ;
		%Straight Lines [id:da9808378369598041] 
		\draw    (351.27,140.02) -- (312.19,211.98) ;
		%Shape: Right Angle [id:dp8872898467261414] 
		\draw   (335.57,197.86) -- (340.49,196.35) -- (342,201.28) ;
		%Shape: Right Angle [id:dp24699786814757219] 
		\draw   (355.52,142.53) -- (353.09,147.07) -- (348.54,144.65) ;

		% Text Node
		\draw (294.19,105.5) node [anchor=north west][inner sep=0.75pt]    {$A$};
		% Text Node
		\draw (239.19,207.83) node [anchor=north west][inner sep=0.75pt]    {$B$};
		% Text Node
		\draw (393.52,208.83) node [anchor=north west][inner sep=0.75pt]    {$C$};
		% Text Node
		\draw (450.86,106.83) node [anchor=north west][inner sep=0.75pt]    {$D$};
		% Text Node
		\draw (306.19,212.83) node [anchor=north west][inner sep=0.75pt]    {$P$};
		% Text Node
		\draw (413.86,172.5) node [anchor=north west][inner sep=0.75pt]    {$Q$};
		% Text Node
		\draw (333.36,169.83) node [anchor=north west][inner sep=0.75pt]    {$H$};


	\end{centertikzpicture}\captionof{figure}{}

\end{exercise}
\begin{solution}
	取 $A C$ 的中点 $O$, 连结 $O P$.
	因为 $\triangle A P C$ 和 $\triangle A Q C$ 均为直角三角形, 所以
	$$
		O P=O A=O Q=\frac{1}{2} A C=\frac{1}{2} b .
	$$

	故 $O$ 为 $\triangle A P Q$ 的外心.
	过 $O$ 作 $O R \perp P Q, O T \perp A Q$, 连结 $H Q$, 过 $T$
	作 $T S / / A H$ 交 $H Q$ 于 $S$, 连结 $R S ,  O R$.

	易知 $R ,  T$ 分别为 $P Q ,  A Q$ 的中点.
	所以 $P R=\frac{1}{2} P Q=\frac{1}{2} a$.
	因为 $T S / / A H$, 故 $T S=\frac{1}{2} A H$ 且 $S$ 为 $H Q$ 的中点.
	因为 $A H \perp P Q, O R \perp P Q$, 所以 $A H / / O R$.
	故 $O R / / T S$.
	同理: $O T / / S R$.
	因为四边形 $O R S T$ 为平行四边形, 所以 $O R=T S=\frac{1}{2} A H$.
	在 Rt $\triangle P O R$ 中, $O R=\sqrt{\left(\frac{1}{2} b\right)^2-\left(\frac{1}{2} a\right)^2}=\frac{1}{2} \sqrt{b^2-a^2}$,

	所以 $A H=2 O R=\sqrt{b^2-a^2}$.
\end{solution}
%4 已知直角三角形 $A B C$ 中, 斜边 $A B$ 长为 $2, \angle A C B=90^{\circ}$, 三角形内一个动点到三个顶点的距离之和的最小值为 $\sqrt{7}$. 求这个直角三角形的两个锐角的大小.

%5. 如图, 在四边形 $A B C D$ 中, $\angle A B C=30^{\circ}, \angle A D C=60^{\circ}, A D=A C$. 证明: $B D^2=A B^2+B C^2$.

%6 如图, 已知正方形 $A B C D$ 的边长为 $1, P,  Q$ 是其内两点, 且 $\angle P A Q=$ $\angle P C Q=45^{\circ}$. 求 $S_{\triangle P A B}+S_{\triangle P C Q}+S_{\triangle Q A D}$ 的值.

\end{document}
