\chapter{因式分解技巧}
\section*{什么是因式分解}
在小学里, 我们学过整数的因数分解. 由乘法, 得
\begin{align*}
	3 \times 4=12
\end{align*}
反过来, 12 可以分解: $12=3 \times 4$.

当然,  4 还可以继续分解为 $2 \times 2$ . 于是得
\begin{align*}
	12=3 \times 2 \times 2
\end{align*}

这时 12 已经分解成质因数的乘积了.

同样地, 由整式乘法, 得
\begin{align*}
	(1+2 x)\left(1-x^{2}\right)=1+2 x-x^{2}-2 x^{3}
\end{align*}

反过来,  $1+2 x-x^{2}-2 x^{3}$ 可以分解为两个因式 $1+2 x$ 与 $1-x^{2}$ 的乘积, 即
\begin{align*}
	1+2 x-x^{2}-2 x^{3}=(1+2 x)\left(1-x^{2}\right)
\end{align*}

$1-x^{2}$ 还可以继续分解为 $(1+x)(1-x)$. 于是
\begin{align*}
	1+2 x-x^{2}-2 x^{3}=(1+2 x)(1+x)(1-x)
\end{align*}

这里 $x$ 的一次多项式 $1+2 x ,  1+x ,  1-x$ 都不能继续分解, 它们是不可约多项式, 也就是既约多项式.  所以,  $1+2 x-x^{2}-2 x^{3}$ 已经分解成质因式的乘积了.

把一个整式写成几个整式的乘积, 称为因式分解, 每一个乘式称为积的因式.

在因式分解中, 通常要求各个乘式(因式)都是既约多项式, 这样的因式称为质因式.

因式分解的方法, 我们将逐一介绍.

\section{提公因式}
学过因式分解的人爱说: “一提、二代、三分组.”

“提”是指“提取公因式”. 在因式分解时, 首先应当想到的是有没有公因式可提.

几个整式都含有的因式称为它们的公因式.

例如 $m a ,  m b , -m c$ 都含有因式 $m, m$ 就是它们的公因式.

由乘法分配律, 我们知道
\begin{align*}
	m(a+b-c)=m a+m b-m c,
\end{align*}
因此
\begin{align*}
	m a+m b-m c=m(a+b-c) .
\end{align*}

这表明上式左边三项的公因式 $m$ 可以提取出来, 作为整式 $m a+m b-$ $m c$ 的因式. $m a+m b-m c$ 的另一个因式 $a+b-c$ 仍由三项组成, 每一项等于 $m a+m b-m c$ 中对应的项除以公因式 $m$ :
\begin{align*}
	a=m a \div m, b=m b \div m, c=m c \div m
\end{align*}

% 例 1
\begin{example}[一次提净]\label{ex:提公因式-例1-一次提净}
	分解因式: $12 a^{2} x^{3}+6 a b x^{2} y-15 a c x^{2}$
\end{example}
\begin{solution}
	$12 a^{2} x^{3}+6 a b x^{2} y-15 a c x^{2}$ 由
	\begin{align*}
		12 a^{2} x^{3}, 6 a b x^{2} y,-15 a c x^{2}
	\end{align*}
	这三项组成, 它们的数系数 $12 ,  6 , -15$ 的最大公约数是 3 , 各项都含有因式 $a$和 $x^{2}$, 所以 $3 a x^{2}$ 是上述三项的公因式, 可以提取出来作为 $12 a^{2} x^{3}+6 a b x^{2} y-$ $15 a c x^{2}$ 的因式, 即有
	\begin{align*}
		  & 12 a^{2} x^{3}+6 a b x^{2} y-15 a c x^{2} \\
		= & 3 a x^{2}(4 a x+2 b y-5 c) .
	\end{align*}
\end{solution}
\begin{note}
	在\autoref{ex:提公因式-例1-一次提净}中, 如果只将因式 $3 a$ 或 $3 a x$ 提出, 那么留下的式子仍有公因式可以提取, 这增添了麻烦, 不如一次提净为好. 因此, 应当先检查数系数, 然后再一个个字母逐一检查, 将各项的公因式提出来, 使留下的式子没有公因式可以直接提取.

	还需注意原式如果由三项组成, 那么提取公因式后留下的式子仍由三项组成. 在例1 中, 这三项分别为 $12 a^{2} x^{3} ,  6 a b x^{2} y , -15 a c x^{2}$ 除以公因式 $3 a x^{2}$ 所得的商. 初学的同学为了防止产生错误, 可以采取两点措施:

	\begin{enumerate}
		\item 在提公因式前, 先将原式的三项都写成公因式 $3 a x^{2}$ 与另一个式子的积, 然后再提取公因式, 即
		      \begin{align*}
			        & 12 a^{2} x^{3}+6 a b x^{2} y-15 a c x^{2}                         \\
			      = & 3 a x^{2} \cdot 4 a x+3 a x^{2} \cdot 2 b y+3 a x^{2} \cdot(-5 c) \\
			      = & 3 a x^{2} \cdot(4 a x+2 b y-5 c) .
		      \end{align*}
		      在熟练之后应当省去中间过程, 直接写出结果.
		\item 用乘法分配律进行验算. 由乘法得出
		      \begin{align*}
			        & 3 a x^{2}(4 a x+2 b y-5 c)                  \\
			      = & 12 a^{2} x^{3}+6 a b x^{2} y-15 a c x^{2} .
		      \end{align*}
	\end{enumerate}
\end{note}

% 例 2
\begin{example}[视“多”为一]\label{ex:提公因式-例2-一次提净}
	分解因式: $2 a^{2} b(x+y)^{2}(b+c)-6 a^{3} b^{3}(x+y)(b+c)^{2}$
\end{example}
\begin{solution}
	原式由
	\begin{align*}
		2 a^{2} b(x+y)^{2}(b+c) , -6 a^{3} b^{3}(x+y)(b+c)^{2}
	\end{align*}
	这两项组成. 它们的数系数的最大公约数是 2 , 两项都含有因式 $a^{2}$ 和 $b$, 而且都含有因式 $x+y$ 与 $b+c$, 因此 $2 a^{2} b(x+y)(b+c)$ 是它们的公因式. 于是有
	\begin{align*}
		\begin{aligned}
			  & 2 a^{2} b(x+y)^{2}(b+c)-6 a^{3} b^{3}(x+y)(b+c)^{2}                     \\
			= & 2 a^{2} b(x+y)(b+c) \cdot(x+y)-2 a^{2} b(x+y)(b+c) \cdot 3 a b^{2}(b+c) \\
			= & 2 a^{2} b(x+y)(b+c)\left[(x+y)-3 a b^{2}(b+c)\right]                    \\
			= & 2 a^{2} b(x+y)(b+c)\left(x+y-3 a b^{3}-3 a b^{2} c\right) .
		\end{aligned}
	\end{align*}
	在本例中, 我们把多项式 $x+y ,  b+c$ 分别整个看成是一个字母, 这种观点在因式分解时是很有用的.
\end{solution}

% 例 3
\begin{example}[切勿漏 1]
	分解因式: $(2 x+y)^{3}-(2 x+y)^{2}+(2 x+y)$.
\end{example}
\begin{solution}
	我们把多项式 $2 x+y$ 看成是一个字母, 因此原式由
	\begin{align*}
		(2 x+y)^{3},-(2 x+y)^{2}, 2 x+y
	\end{align*}
	这三项组成,  $2 x+y$ 是这三项的公因式, 于是
	\begin{align*}
		\begin{aligned}
			  & (2 x+y)^{3}-(2 x+y)^{2}+(2 x+y)                               \\
			= & (2 x+y) \cdot(2 x+y)^{2}-(2 x+y) \cdot(2 x+y)+(2 x+y) \cdot 1 \\
			= & (2 x+y)\left[(2 x+y)^{2}-(2 x+y)+1\right] .
		\end{aligned}
	\end{align*}
	请注意, 中括号内的式子仍由三项组成, 千万不要忽略最后一项1. 在省去中间过程时, 尤需加倍留心.
\end{solution}

% 例 4
\begin{example}[注意符号]\label{ex:提公因式-例4-注意符号}
	分解因式: $-3 a b(2 x+3 y)^{4}+a c(2 x+3 y)^{3}-a(2 x+3 y)$.
\end{example}
\begin{solution}
	$\quad-3 a b(2 x+3 y)^{4}+a c(2 x+3 y)^{3}-a(2 x+3 y)$\\
	$=a(2 x+3 y) \cdot(-3 b) \cdot(2 x+3 y)^{3}+a(2 x+3 y) \cdot c(2 x+3 y)^{2}+$ $a(2 x+3 y) \cdot(-1)$
	\begin{align*}
		=a(2 x+3 y)\left[-3 b(2 x+3 y)^{3}+c(2 x+3 y)^{2}-1\right] .
	\end{align*}
\end{solution}
\begin{note}
	注意中括号内的最后一项是 -1 , 千万别漏掉. 本例中, 原式的第一项有个因数 -1 , 它也可以作为因数提取出来, 即
	\begin{align*}
		  & -3 a b(2 x+3 y)^{4}+a c(2 x+3 y)^{3}-a(2 x+3 y)                       \\
		= & -a(2 x+3 y) \cdot 3 b(2 x+3 y)^{3}-a(2 x+3 y) \cdot(-c)(2 x+3 y)^{2}- \\
		  & a(2 x+3 y) \cdot 1                                                    \\
		= & -a(2 x+3 y)\left[3 b(2 x+3 y)^{3}-c(2 x+3 y)^{2}+1\right] .
	\end{align*}
	这样做也是正确的. 但必须注意各项的符号, 提出因数 -1 后各项都应改变符号, 所以上式的中括号内三项的符号恰与原式中相应的三项相反.
\end{note}

% 例 5
\begin{example}[仔细观察]\label{ex:提公因式-例5-仔细观察}
	分解因式: $(2 x-3 y)(3 x-2 y)+(2 y-3 x)(2 x+3 y)$ .
\end{example}
\begin{solution}
	初看起来, 原式所含的第一项 $(2 x-3 y)(3 x-2 y)$ 与第二项 $(2 y-$ $3 x)(2 x+3 y)$ 没有公因式, 但进一步观察便会发现
	\begin{align*}
		2 y-3 x=-(3 x-2 y),
	\end{align*}
	因此 $3 x-2 y$ 是两项的公因式. 于是有
	\begin{align*}
		\begin{aligned}
			  & (2 x-3 y)(3 x-2 y)+(2 y-3 x)(2 x+3 y) \\
			= & (3 x-2 y)[(2 x-3 y)-(2 x+3 y)]        \\
			= & -6 y(3 x-2 y) .
		\end{aligned}
	\end{align*}
	提出公因式后, 留下的式子如果可以化简, 就应当化简.
\end{solution}

% 例 6
\begin{example}[化“分”为整]\label{ex:提公因式-例6-化分为整}
	分解因式: $3 a^{3} b^{2}-6 a^{2} b^{3}+\frac{27}{4} a b$.
\end{example}
\begin{solution}
	这里的第三项 $\frac{27}{4} a b$ 的系数是分数, 为了避免分数运算, 我们把 $\frac{1}{4}$ 先提取出来, 这时每项都除以 $\frac{1}{4}$ (也就是乘以 4 ), 即
	\begin{align*}
		\begin{aligned}
			  & 3 a^{3} b^{2}-6 a^{2} b^{3}+\frac{27}{4} a b                 \\
			= & \frac{1}{4}\left(12 a^{3} b^{2}-24 a^{2} b^{3}+27 a b\right) \\
			= & \frac{3}{4} a b\left(4 a^{2} b-8 a b^{2}+9\right) .
		\end{aligned}
	\end{align*}
	熟练以后可以将以上两步并作一步, “一次提净”.

	在提出一个分数因数(它的分母是各项系数的公分母)后, 我们总可以使各项系数都化为整数(这个过程实质上就是通分). 并且, 还可以假定第一项系数是正整数, 否则可用前面说过的方法, 把 -1 作为公因数提出, 使第一项系数成为正整数.
\end{solution}
\begin{note}
	提公因式是因式分解的基本方法之一. 在因式分解时, 首先应该想到是否有公因式可提. 在与其他方法配合时, 即使开始已经提出公因式, 但是经过分组或应用公式后还有可能再出现公因式. 凡有公因式应立即提净. 提公因式时, 应注意各项的符号, 千万不要漏掉一项.
\end{note}

\subsection*{习题 1}
将以下各式分解因式:
\begin{enumerate}
	\item $5 x^{2} y-10 x y z+5 x y$.
	\item $2 a(x-a)+b(a-x)-(x-a)$.
	\item $3-2 x(x+1)+a(x+1)+(x+1)$.
	\item $\frac{3}{2} b^{3 n-1}+\frac{1}{6} b^{2 n-1}$ ($n$ 是正整数).
	\item $2(p-1)^{2}-4 q(p-1)$.
	\item $m n\left(m^{2}+n^{2}\right)-n^{2}\left(m^{2}+n^{2}\right)$.
	\item $(5 a-2 b)(2 m+3 p)-(2 a-7 b)(2 m+3 p)$.
	\item $2(x+y)+6(x+y)^{2}-4(x+y)^{3}$.
	\item $(x+y)^{2}(b+c)-(x+y)(b+c)^{2}$.
	\item $6 p(x-1)^{3}-8 p^{2}(x-1)^{2}-2 p(1-x)^{2}$.
\end{enumerate}

\section{应用公式}
将乘法公式反过来写就得到因式分解中所用的公式, 常见的有七个公式:
\begin{enumerate}
	\item $a^{2}-b^{2}=(a+b)(a-b)$.
	\item $a^{3}+b^{3}=(a+b)\left(a^{2}-a b+b^{2}\right)$.
	\item $a^{3}-b^{3}=(a-b)\left(a^{2}+a b+b^{2}\right)$.
	\item $a^{2}+2 a b+b^{2}=(a+b)^{2}$.
	\item $a^{2}-2 a b+b^{2}=(a-b)^{2}$.
	\item $a^{3}+3 a^{2} b+3 a b^{2}+b^{3}=(a+b)^{3}$.
	\item $a^{3}-3 a^{2} b+3 a b^{2}-b^{3}=(a-b)^{3}$.
\end{enumerate}
以上公式必须熟记, 牢牢掌握各自的特点.

\subsection{平方差}
七个公式中, 平方差公式应用得最多.
% 例 1
\begin{example}
	分解因式: $9(m-n)^{2}-4(m+n)^{2}$ .
\end{example}
\begin{solution}
	原式由两项组成, 这两项符号相反, 并且
	\begin{align*}
		 & 9(m-n)^{2}=[3(m-n)]^{2}, \\
		 & 4(m+n)^{2}=[2(m+n)]^{2},
	\end{align*}
	因此可以应用平方差公式, 得
	\begin{align*}
		  & 9(m-n)^{2}-4(m+n)^{2}            \\
		= & {[3(m-n)]^{2}-[2(m+n)]^{2}}      \\
		= & {[3(m-n)+2(m+n)][3(m-n)-2(m+n)]} \\
		= & (5 m-n)(m-5 n) .
	\end{align*}
\end{solution}

% ·例 2
\begin{example}
	分解因式: $75 x^{6} y-12 x^{2} y^{5}$.
\end{example}
\begin{solution}
	\begin{align*}
		75 x^{6} y - 12 x^{2} y^{5}  = & 3 x^{2} y \left(25 x^{4} - 4 y^{4}\right)                                  \\
		=                              & 3 x^{2} y \left[\left(5 x^{2}\right)^{2} - \left(2 y^{2}\right)^{2}\right] \\
		=                              & 3 x^{2} y \left(5 x^{2} + 2 y^{2}\right) \left(5 x^{2} - 2 y^{2}\right)
	\end{align*}
\end{solution}

% 例 3
\begin{example}\label{ex:应用公式-例3}
	分解因式: $-\left(3 a^{2}-5 b^{2}\right)^{2}+\left(5 a^{2}-3 b^{2}\right)^{2}$.
\end{example}
\begin{solution}
	\begin{align*}
		  & -\left(3 a^{2}-5 b^{2}\right)^{2}+\left(5 a^{2}-3 b^{2}\right)^{2}                                                                           \\
		= & \left(5 a^{2}-3 b^{2}\right)^{2}-\left(3 a^{2}-5 b^{2}\right)^{2}                                                                            \\
		= & \left[\left(5 a^{2}-3 b^{2}\right)+\left(3 a^{2}-5 b^{2}\right)\right]\left[\left(5 a^{2}-3 b^{2}\right)-\left(3 a^{2}-5 b^{2}\right)\right] \\
		= & \left(8 a^{2}-8 b^{2}\right)\left(2 a^{2}+2 b^{2}\right)                                                                                     \\
		= & 16\left(a^{2}-b^{2}\right)\left(a^{2}+b^{2}\right)                                                                                           \\
		= & 16(a+b)(a-b)\left(a^{2}+b^{2}\right)
	\end{align*}
\end{solution}
\begin{note}
	\autoref{ex:应用公式-例3}表明在因式公解中可能需要多次应用公式或提公因式, 直到不能继续分解为止.
\end{note}

\subsection{立方和与立方差}
% 例 4
\begin{example}
	分解因式: $9 x^{5}-72 x^{2} y^{3}$ .
\end{example}
\begin{solution}
	\begin{align*}
		9 x^{5}-72 x^{2} y^{3} & = 9 x^{2}\left(x^{3}-8 y^{3}\right)              \\
		                       & = 9 x^{2}\left[x^{3}-(2 y)^{3}\right]            \\
		                       & = 9 x^{2}(x-2 y)\left(x^{2}+2 x y+4 y^{2}\right)
	\end{align*}
\end{solution}

% 例 5
\begin{example}
	分解因式: $a^{6}+b^{6}$ .
\end{example}
\begin{solution}
	\begin{align*}
		a^{6}+b^{6} & = \left(a^{2}\right)^{3}+\left(b^{2}\right)^{3}                                                  \\
		            & = \left(a^{2}+b^{2}\right)\left[\left(a^{2}\right)^{2}-a^{2} b^{2}+\left(b^{2}\right)^{2}\right] \\
		            & = \left(a^{2}+b^{2}\right)\left(a^{4}-a^{2} b^{2}+b^{4}\right)
	\end{align*}
\end{solution}

\subsection{完全平方}
% 例 6
\begin{example}
	分解因式: $9 x^{2}-24 x y+16 y^{2}$ .
\end{example}
\begin{solution}
	原式由三项组成, 第一项 $9 x^{2}=(3 x)^{2}$ , 第三项 $16 y^{2}=(4 y)^{2}$ , 而
	\begin{align*}
		2 \cdot 3 x \cdot 4 y=24 x y
	\end{align*}
	与中间一项只差一个符号, 因此可以利用(完全)平方式, 得
	\begin{align*}
		\begin{aligned}
			  & 9 x^{2}-24 x y+16 y^{2} \\
			= & (3 x-4 y)^{2} .
		\end{aligned}
	\end{align*}
	不是平方式的二次三项式, 通常用十字相乘法分解(后面会讲).
\end{solution}

% 例 7
\begin{example}
	分解因式: $8 a-4 a^{2}-4$ .
\end{example}
\begin{solution}
	首先把原式“理顺”, 也就是将它的各项按字母 $a$ 降幂 (或升幂)排列,从而有
	\begin{align*}
		\begin{aligned}
			  & 8 a-4 a^{2}-4              \\
			= & -4 a^{2}+8 a-4             \\
			= & -4\left(a^{2}-2 a+1\right) \\
			= & -4(a-1)^{2} .
		\end{aligned}
	\end{align*}
\end{solution}
\begin{note}
	按某个字母降幂排列是一个简单而有用的措施(简单的往往是有用的), 值得注意.
\end{note}

% 例 8
\begin{example}\label{ex:应用公式-例8}
	分解因式: $4 a^{2}+9 b^{2}+9 c^{2}-18 b c-12 c a+12 a b$ .
\end{example}
\begin{solution}
	我们需要引入一个公式.由乘法可得
	\begin{align*}
		(a+b+c)^{2}=a^{2}+b^{2}+c^{2}+2 a b+2 b c+2 c a,
	\end{align*}
	即若干项的和的平方等于各项的平方与每两项乘积的 2 倍的和.
	上面的式子可写成
	\begin{align*}
		  & a^{2}+b^{2}+c^{2}+2 a b+2 b c+2 c a \\
		= & (a+b+c)^{2} .
	\end{align*}
	这也是一个因式分解的公式.

	联系到\autoref{ex:应用公式-例8}就有
	\begin{align*}
		  & 4 a^{2}+9 b^{2}+9 c^{2}-18 b c-12 c a+12 a b                         \\
		= & (2 a)^{2}+(3 b)^{2}+(-3 c)^{2}+2(3 b)(-3 c)+2(2 a)(-3 c)+2(2 a)(3 b) \\
		= & (2 a+3 b-3 c)^{2} .
	\end{align*}
\end{solution}

\subsection{完全立方}
% 例 9
\begin{example}
	分解因式: $8 x^{3}+27 y^{3}+36 x^{2} y+54 x y^{2}$ .
\end{example}
\begin{solution}
	\begin{align*}
		  & 8 x^{3}+27 y^{3}+36 x^{2} y+54 x y^{2}                \\
		= & 8 x^{3}+36 x^{2} y+54 x y^{2}+27 y^{3}                \\
		= & (2 x)^{3}+3(2 x)^{2}(3 y)+3(2 x)(3 y)^{2}+(3 y)^{3} x \\
		= & (2 x+3 y)^{3} .                                       \\
	\end{align*}
\end{solution}

% 例 10
\begin{example}
	分解因式: $729 a^{6}-243 a^{4}+27 a^{2}-1$.
\end{example}
\begin{solution}
	\begin{align*}
		  & 729 a^{6}-243 a^{4}+27 a^{2}-1                                                                                 \\
		= & \left(9 a^{2}\right)^{3}-3 \cdot\left(9 a^{2}\right)^{2} \cdot 1+3 \cdot\left(9 a^{2}\right) \cdot 1^{2}-1^{3} \\
		= & \left(9 a^{2}-1\right)^{3}                                                                                     \\
		= & (3 a+1)^{3}(3 a-1)^{3}
	\end{align*}
\end{solution}

% 例 11
\begin{example}
	分解因式: $a^{6}-b^{6}$ .
\end{example}
\begin{solution}
	$a^{6}$ 可以看成平方:
	\begin{align*}
		a^{6}=\left(a^{3}\right)^{2},
	\end{align*}
	也可以看成立方:
	\begin{align*}
		a^{6}=\left(a^{2}\right)^{3},
	\end{align*}
	于是 $a^{6}-b^{6}$ 的分解就有两条路可走. \\
	第一条路是先应用平方差公式:
	\begin{align*}
		a^{6}-b^{6} & = \left(a^{3}\right)^{2}-\left(b^{3}\right)^{2}                      \\
		            & = \left(a^{3}+b^{3}\right)\left(a^{3}-b^{3}\right)                   \\
		            & = (a+b)\left(a^{2}-a b+b^{2}\right)(a-b)\left(a^{2}+a b+b^{2}\right)
	\end{align*}
	第二条路是从立方差公式入手:
	\begin{align*}
		a^{6}-b^{6} & = \left(a^{2}\right)^{3}-\left(b^{2}\right)^{3}                \\
		            & = \left(a^{2}-b^{2}\right)\left(a^{4}+a^{2} b^{2}+b^{4}\right) \\
		            & = (a+b)(a-b)\left(a^{4}+a^{2} b^{2}+b^{4}\right)
	\end{align*}
\end{solution}
\begin{note}
	采用两种方法分解, 获得的结果应当相同. 因此比较
	\begin{align*}
		(a+b)\left(a^{2}-a b+b^{2}\right)(a-b)\left(a^{2}+a b+b^{2}\right)
	\end{align*}
	与
	\begin{align*}
		(a+b)(a-b)\left(a^{4}+a^{2} b^{2}+b^{4}\right),
	\end{align*}
	我们知道 $a^{4}+a^{2} b^{2}+b^{4}$ 不是既约多项式, 并且有
	\begin{align}\label{eq:应用公式-例11-1}
		a^{4}+a^{2} b^{2}+b^{4}=\left(a^{2}+a b+b^{2}\right)\left(a^{2}-a b+b^{2}\right)
	\end{align}
	及
	\begin{align}\label{eq:应用公式-例11-2}
		a^{6}-b^{6}=(a+b)(a-b)\left(a^{2}+a b+b^{2}\right)\left(a^{2}-a b+b^{2}\right) .
	\end{align}
	于是, 从 $a^{6}-b^{6}$ 的分解出发, 不但得到\ref{eq:应用公式-例11-2}式, 而且知道 $a^{4}+a^{2} b^{2}+b^{4}$ 不是既约多项式, 导出了\ref{eq:应用公式-例11-1}式, 可谓问一知三.

	后面我们还要介绍导出\ref{eq:应用公式-例11-1}式的另一种方法.
\end{note}

\subsection{$2^{1984}+1$ 不是质数}

% 例 12
\begin{example}
	求证 $2^{1984}+1$ 不是质数.
\end{example}
\begin{solution}
	为了将 $2^{1984}+1$ 分解因数, 我们需要知道一个新的公式, 即在 $n$ 为正奇数时
	\begin{align*}
		a^{n}+b^{n}=(a+b)\left(a^{n-1}-a^{n-2} b+a^{n-3} b^{2}-\cdots-a b^{n-2}+b^{n-1}\right) .
	\end{align*}
	上式不难用乘法验证, 将右边的两个因式相乘便得到 $a^{n}+b^{n}$ . 现在我们有
	\begin{align*}
		2^{1984}+1 & = \left(2^{64}\right)^{31}+1^{31}                                                       \\
		           & = \left(2^{64}+1\right)\left(2^{64 \times 30}-2^{64 \times 29}+\cdots-2^{64}+1\right) .
	\end{align*}
	$2^{64}+1$ 是 $2^{1984}+1$ 的真因数, 它大于 1 , 小于 $2^{1984}+1$ , 所以 $2^{1984}+1$ 不是质数. 用这个方法可以证明:当 $n$ 有大于 1 的奇数因数时,  $2^{n}+1$ 不是质数.
\end{solution}
\begin{note}
	类似地, 由乘法可以得到在 $n$ 为正整数时
	\begin{align*}
		a^{n}-b^{n}=(a-b)\left(a^{n-1}+a^{n-2} b+a^{n-3} b^{2}+\cdots+a b^{n-2}+b^{n-1}\right) . \tag{12}
	\end{align*}
	这也是一个有用的公式.
\end{note}

% 例 13
\begin{example}
	分解因式: $x^{5}-1$.
\end{example}
\begin{solution}
	\begin{align*}
		x^{5}-1 & = (x-1)\left(x^{4}+x^{3}+x^{2}+x+1\right)
	\end{align*}
\end{solution}

\subsection*{习题 2}
将以下各式分解因式:
\begin{enumerate}
	\item $16- (3 a+2 b)^{2}$.
	\item $4 y^{2}-(2 z-x)^{2}$.
	\item $a^{4}-b^{4}$.
	\item $-81 a^{4} b^{4}+16 c^{4}$.
	\item $20 a^{3} x^{3}-45 a x y^{2}$.
	\item $\left(3 a^{2}-b^{2}\right)^{2}-\left(a^{2}-3 b^{2}\right)^{2}$.
	\item $ x^{8}-y^{8}$.
	\item $16 x^{5}-x$.
	\item $\left(5 x^{2}+2 x-3\right)^{2}-\left(x^{2}-2 x-3\right)^{2}$.
	\item $32 a^{3} b^{3}-4 b^{9}$.
	\item $8 a^{3} b^{3} c^{3}-1$.
	\item $64 x^{6} y^{3}+y^{15}$.
	\item $x^{2}(a+b)^{2}-2 x y\left(a^{2}-b^{2}\right)+y^{2}(a-b)^{2}$.
	\item $a^{n+2}+8 a^{n}+16 a^{n-2}$.
	\item $9 a^{2}+x^{2 n}+6 a+2 x^{n}+6 a x^{n}+1$.
	\item $a^{2}+b^{2}+c^{2}+2 a b-2 a c-2 b c$.
	\item $x^{2}+9 y^{2}+4 z^{2}-6 x y+4 x z-12 y z$.
	\item $(p+q)^{3}-3(p+q)^{2}(p-q)+3(p+q)(p-q)^{2}-(p-q)^{3}$.
	\item $4 a^{2} b^{2}-\left(a^{2}+b^{2}\right)^{2}$.
	\item $(a+x)^{4}-(a-x)^{4}$.
\end{enumerate}


\section{分组分解}
% 例 1
\begin{example}[分组分解三部曲]
	分解因式: $a x-b y-b x+a y .$
\end{example}
\begin{solution}
	\begin{align*}
		  & a x-b y-b x+a y     \\
		= & (a x-b x)+(a y-b y) \\
		= & x(a-b)+y(a-b)       \\
		= & (x+y)(a-b) .
	\end{align*}
	分组的方法并不是唯一的,对于上面的整式 $a x-b y-b x+a y$ ,也可以采用下面的做法:
	\begin{align*}
		  & a x-b y-b x+a y     \\
		= & (a x+a y)-(b x+b y) \\
		= & a(x+y)-b(x+y)       \\
		= & (x+y)(a-b)
	\end{align*}
	两种做法的效果是一样的,殊途同归!可以说,一种是按照 $x$ 与 $y$ 来分组(含 $x$ 的项在一组,含 $y$ 的项在另一组);另一种是按 $a$ 与 $b$ 来分组。
\end{solution}
一般地,分组分解大致分为三步:
\begin{enumerate}
	\item 将原式的项适当分组;
	\item 对每一组进行处理(“提”或“代”);
	\item 将经过处理后的每一组当作一项,再采用“提”或“代”进行分解。
\end{enumerate}
一位高明的棋手,在下棋时,决不会只看一步。同样,在进行分组时,不仅要看到第二步,而且要看到第三步
一个整式的项有许多种分组的方法,初学者往往需要经过尝试才能找到适当的分组方法,但是只要努力实践,多加练习,就会成为有经验的“行家”.

% 例 2
\begin{example}[殊途同归]
	分解因式: $x^{2}+a x^{2}+x+a x-1-a$.
\end{example}
\begin{solution}
	\textbf{解法一}:按字母 $x$ 的幂来分组。
	\begin{align*}
		\begin{aligned}
			  & x^{2}+a x^{2}+x+a x-1-a                  \\
			= & \left(x^{2}+a x^{2}\right)+(x+a x)-(1+a) \\
			= & x^{2}(1+a)+x(1+a)-(1+a)                  \\
			= & (1+a)\left(x^{2}+x-1\right) .
		\end{aligned}
	\end{align*}

	\textbf{解法二}:按字母 $a$ 的幂来分组.
	\begin{align*}
		\begin{aligned}
			  & x^{2}+a x^{2}+x+a x-1-a                           \\
			= & \left(a x^{2}+a x-a\right)+\left(x^{2}+x-1\right) \\
			= & a\left(x^{2}+x-1\right)+\left(x^{2}+x-1\right)    \\
			= & (a+1)\left(x^{2}+x-1\right) .
		\end{aligned}
	\end{align*}
\end{solution}

% 例 5
\begin{example}
	分解因式: $-1-2 x-x^{2}+y^{2}$.
\end{example}
\begin{solution}
	\begin{align*}
		  & -1-2 x-x^{2}+y^{2}             \\
		= & y^{2}-\left(x^{2}+2 x+1\right) \\
		= & y^{2}-(x+1)^{2}                \\
		= & (y+x+1)(y-x-1)
	\end{align*}
\end{solution}

% 例 6
\begin{example}
	分解因式: $a x^{3}+x+a+1$.
\end{example}
\begin{solution}
	\begin{align*}
		  & a x^{3}+x+a+1                      \\
		= & \left(a x^{3}+a\right)+(x+1)       \\
		= & a(x+1)\left(x^{2}-x+1\right)+(x+1) \\
		= & (x+1)\left(a x^{2}-a x+a+1\right)
	\end{align*}
\end{solution}

% 例 10
\begin{example}
	分解因式: $a b\left(c^{2}-d^{2}\right)-\left(a^{2}-d^{2}\right) c d.$
\end{example}
\begin{solution}
	此式无法直接进行分解,必须先用乘法分配律将原式变为四项,再进行分组.
	\begin{align*}
		  & a b\left(c^{2}-d^{2}\right)-\left(a^{2}-b^{2}\right) c d          \\
		= & a b c^{2}-a b d^{2}-a^{2} c d+b^{2} c d                           \\
		= & \left(a b c^{2}-a^{2} c d\right)+\left(b^{2} c d-a b d^{2}\right) \\
		= & a c(b c-a d)+b d(b c-a d)                                         \\
		= & (a c+b d)(b c-a d) .
	\end{align*}
	从这个例子可以看出,错误的分组还不如不分组。聪明的人并不是不犯错误的人,而是善于改正错误的人。
\end{solution}
如果"一提、二代"都不能奏效,就应当采用分组分解。分组分解应依照前面所说的三步进行。这三步是密切联系的,不仅要看到第二步,而且要看到第三步。在第二步与第三步都是提取公因式时,各组的项数相等 (平均分配). 否则, 应当瞄准公式来进行分组. 应当注意, 分组需要尝试,失败了,从零开始。只要反复实践,就能掌握分组的技巧,运用自如。

\subsection*{习题 3}
将以下各式分解因式(对应书本第14$\sim$24题):
\begin{enumerate}
	\item $ x^{3}+b x^{2}+a x+a b$.
	\item $ a c x^{3}+b c x^{2}+a d x+b d$.
	\item $ a^{4}+a^{3} b-a b^{3}-b^{4}$.
	\item $ a^{4}-a^{3} b-a b^{3}+b^{4}$.
	\item $ a^{2} b^{2}-a^{2}-b^{2}+1$.
	\item $ x^{2} y^{2}-x^{2} z^{2}-y^{2} z^{2}+z^{4}$.
	\item $ x^{2} y^{2} z^{2}-x^{2} z-y^{2} z+1$.
	\item $ x^{4}+x^{3} y+x z^{3}+y z^{3}$.
	\item $(a+b)^{2}+(a+c)^{2}-(c+d)^{2}-(b+d)^{2}$.
	\item $ a x\left(y^{3}+b^{3}\right)+b y\left(b x^{2}+a^{2} y\right)$.
	\item $(a+b)^{3}+(b+c)^{3}+(c+a)^{3}+a^{3}+b^{3}+c^{3}$.
\end{enumerate}

\section{十字相乘}
\section{多项式的因式分解}
