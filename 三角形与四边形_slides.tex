\documentclass[aspectratio=169]{ctexbeamer}
%\usetheme{Madrid}
\usetheme{Boadilla}
%\usetheme{CambridgeUS}
%\usecolortheme{beaver}
%\usecolortheme{wolverine}
\usepackage{amsmath} 
\usepackage{amssymb} 
\usepackage{amsfonts} 
\usepackage{graphicx}
\usepackage{comment}
\usepackage{pgfplots}
\pgfplotsset{compat=1.18}
\usefonttheme[onlymath]{serif} % 衬线数学字体

%\setbeamertemplate{theorem}[ams style]
\setbeamertemplate{theorems}[numbered]

\theoremstyle{definition}
\newtheorem{question}{问题}[section]
\newtheorem{exercise}{练习}[section]
\newtheorem{formula}{公式}[section]
\newtheorem{proposition}{命题}[section]
\newtheorem{property}{性质}[section]

\let\oldtikzpicture\tikzpicture
\let\oldendtikzpicture\endtikzpicture
\renewenvironment{tikzpicture}
    {\begin{flushright}\oldtikzpicture}
    {\oldendtikzpicture\end{flushright}}
    
\newcommand{\pll}{\kern 0.56em/\kern -0.8em /\kern 0.56em}

\title[三角形与四边形]{平面几何}
\subtitle{三角形与四边形}
\author[珠海一中创美营]{珠海一中创美营(数学)}
\date[\today]{\today}
\AtBeginSection[]
{
	\begin{frame}
		\frametitle{目录}
		\tableofcontents[currentsection]
	\end{frame}
}
\begin{document}
\frame{\titlepage}
\frame{\frametitle{目录}\tableofcontents}
\section{三角形的基本概念和性质}
\begin{frame}{三角形的基本概念和性质}
\begin{theorem}
    三角形顶角的平分线与底边上的高所夹的角等于两底角差的一半。   
\end{theorem}


\tikzset{every picture/.style={line width=0.75pt}} %set default line width to 0.75pt        

\begin{tikzpicture}[x=0.75pt,y=0.75pt,yscale=-1,xscale=1]
%uncomment if require: \path (0,300); %set diagram left start at 0, and has height of 300

%Shape: Triangle [id:dp24488112638922765] 
\draw   (358.17,92) -- (409,193.56) -- (224.67,193.56) -- cycle ;
%Straight Lines [id:da9931344856531952] 
\draw    (358.17,92) -- (358,194.22) ;
%Straight Lines [id:da8638599185890368] 
\draw    (358.17,92) -- (327,192.92) ;
%Shape: Arc [id:dp5410570934824912] 
\draw  [draw opacity=0] (348.9,122.2) .. controls (350.97,123.04) and (353.26,123.5) .. (355.67,123.5) .. controls (356.66,123.5) and (357.63,123.42) .. (358.58,123.27) -- (355.67,107.52) -- cycle ; \draw   (348.9,122.2) .. controls (350.97,123.04) and (353.26,123.5) .. (355.67,123.5) .. controls (356.66,123.5) and (357.63,123.42) .. (358.58,123.27) ;  

% Text Node
\draw (351.17,73.9) node [anchor=north west][inner sep=0.75pt]    {$A$};
% Text Node
\draw (216.17,195.4) node [anchor=north west][inner sep=0.75pt]    {$B$};
% Text Node
\draw (402.67,194.9) node [anchor=north west][inner sep=0.75pt]    {$C$};
% Text Node
\draw (318.67,196.4) node [anchor=north west][inner sep=0.75pt]    {$T$};
% Text Node
\draw (351.17,196.4) node [anchor=north west][inner sep=0.75pt]    {$H$};
% Text Node
\draw (345.83,126.4) node [anchor=north west][inner sep=0.75pt]    {$\theta $};


\end{tikzpicture}


\end{frame}
\end{document}