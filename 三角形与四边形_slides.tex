\documentclass[aspectratio=169]{ctexbeamer}
%\usetheme{Madrid}
\usetheme{Boadilla}
%\usetheme{CambridgeUS}
%\usecolortheme{beaver}
%\usecolortheme{wolverine}
\usepackage{amsmath} 
\usepackage{amssymb} 
\usepackage{amsfonts} 
\usepackage{graphicx}
\usepackage{comment}
\usepackage{pgfplots}
\pgfplotsset{compat=1.18}
\usefonttheme[onlymath]{serif} % 衬线数学字体

%\setbeamertemplate{theorem}[ams style]
\setbeamertemplate{theorems}[numbered]

\theoremstyle{definition}
\newtheorem{question}{问题}[section]
\newtheorem{exercise}{练习}[section]
\newtheorem{formula}{公式}[section]
\newtheorem{proposition}{命题}[section]
\newtheorem{property}{性质}[section]

\let\oldtikzpicture\tikzpicture
\let\oldendtikzpicture\endtikzpicture
\renewenvironment{tikzpicture}
    {\begin{flushright}\oldtikzpicture}
    {\oldendtikzpicture\end{flushright}}
    
\newcommand{\pll}{\kern 0.56em/\kern -0.8em /\kern 0.56em}

\title[三角形与四边形]{平面几何}
\subtitle{三角形与四边形}
\author[珠海一中创美营]{珠海一中创美营 (数学) }
\date[\today]{\today}
\AtBeginSection[]
{
	\begin{frame}
		\frametitle{目录}
		\tableofcontents[currentsection]
	\end{frame}
}
\begin{document}
\frame{\titlepage}
\frame{\frametitle{目录}\tableofcontents}
\section{三角形的基本概念和性质}
\begin{frame}{三角形的基本概念和性质}
	\begin{enumerate}
		\item 边与边之间的关系:两边之和大于第三边,两边之差小于第三边.
		\item 角与角之间的关系:三个内角的和等于 $180^{\circ}$,即在 $\triangle A B C$ 中有 $\angle A+\angle B+\angle C=180^{\circ}$. 由此即知三角形的一个外角等于与它不相邻的两个内角之和.
		\item 边与角之间的关系:在同一三角形中,边长与对角成正比,即大边对大角,小边对小角.
	\end{enumerate}
\end{frame}

\begin{frame}{三角形的基本概念和性质}
	\begin{itemize}
		\item {\color{blue!50!black}三角形的角平分线:} 三角形一个角的平分线与这个角的对边相交,这个角的顶点和交点之间的线段叫做三角形的角平分线.
		\item {\color{blue!50!black}三角形的中线:} 在三角形中,连结一个顶点和它的对边中点的线段叫做三角形的中线.
		\item {\color{blue!50!black}三角形的高:} 从三角形一个顶点向它的对边所在直线画垂线,顶点和垂足间的线段叫做三角形的高线,简称三角形的高.
		\item {\color{blue!50!black}三角形的中位线:} 连结三角形两边中点的线段叫做三角形的中位线。中位线平行于第三边且等于第三边的一半.
		\item {\color{blue!50!black}三角形的外角平分线:} 三角形一个内角的邻补角的平分线与这个角的对边的延长线相交,这个角的顶点和交点之间的线段叫做三角形的外角平分线.
	\end{itemize}
\end{frame}

\begin{frame}{三角形的基本概念和性质}
	\begin{theorem}
		三角形顶角的平分线与底边上的高所夹的角等于两底角差的一半。   
	\end{theorem}
	
	
	\tikzset{every picture/.style={line width=0.75pt}} %set default line width to 0.75pt        
	
	\begin{tikzpicture}[x=0.75pt,y=0.75pt,yscale=-1,xscale=1]
		%uncomment if require: \path (0,300); %set diagram left start at 0, and has height of 300
		
		%Shape: Triangle [id:dp24488112638922765] 
		\draw   (358.17,92) -- (409,193.56) -- (224.67,193.56) -- cycle ;
		%Straight Lines [id:da9931344856531952] 
		\draw    (358.17,92) -- (358,194.22) ;
		%Straight Lines [id:da8638599185890368] 
		\draw    (358.17,92) -- (327,192.92) ;
		%Shape: Arc [id:dp5410570934824912] 
		\draw  [draw opacity=0] (348.9,122.2) .. controls (350.97,123.04) and (353.26,123.5) .. (355.67,123.5) .. controls (356.66,123.5) and (357.63,123.42) .. (358.58,123.27) -- (355.67,107.52) -- cycle ; \draw   (348.9,122.2) .. controls (350.97,123.04) and (353.26,123.5) .. (355.67,123.5) .. controls (356.66,123.5) and (357.63,123.42) .. (358.58,123.27) ;  
		
		% Text Node
		\draw (351.17,73.9) node [anchor=north west][inner sep=0.75pt]    {$A$};
		% Text Node
		\draw (216.17,195.4) node [anchor=north west][inner sep=0.75pt]    {$B$};
		% Text Node
		\draw (402.67,194.9) node [anchor=north west][inner sep=0.75pt]    {$C$};
		% Text Node
		\draw (318.67,196.4) node [anchor=north west][inner sep=0.75pt]    {$T$};
		% Text Node
		\draw (351.17,196.4) node [anchor=north west][inner sep=0.75pt]    {$H$};
		% Text Node
		\draw (345.83,126.4) node [anchor=north west][inner sep=0.75pt]    {$\theta $};
		
		
	\end{tikzpicture}
	
	
\end{frame}

\setcounter{theorem}{0}
% 例 1
\begin{frame}[t]
	\begin{example}
		设$P$是边长为 1 的正三角形$ABC$内一点,求证:
		\begin{align}
			\frac{3}{2}< PA+PB+PC < 2.
		\end{align}
	\end{example}
	
	
	\tikzset{every picture/.style={line width=0.75pt}} %set default line width to 0.75pt        
	
	\begin{tikzpicture}[x=0.75pt,y=0.75pt,yscale=-1,xscale=1]
		%uncomment if require: \path (0,300); %set diagram left start at 0, and has height of 300
		
		%Shape: Triangle [id:dp48456512171565747] 
		\draw   (356.67,82) -- (435.33,212.36) -- (278,212.36) -- cycle ;
		%Straight Lines [id:da5764022312498003] 
		\draw    (339.33,166) -- (278,212.36) ;
		%Straight Lines [id:da12406244425770074] 
		\draw    (356.67,82) -- (339.33,166) ;
		%Straight Lines [id:da7051815211987567] 
		\draw    (339.33,166) -- (435.33,212.36) ;
		
		% Text Node
		\draw (336,169.4) node [anchor=north west][inner sep=0.75pt]    {$P$};
		% Text Node
		\draw (349.33,62.07) node [anchor=north west][inner sep=0.75pt]    {$A$};
		% Text Node
		\draw (264,207.73) node [anchor=north west][inner sep=0.75pt]    {$B$};
		% Text Node
		\draw (436.67,208.4) node [anchor=north west][inner sep=0.75pt]    {$C$};
		
		
	\end{tikzpicture}
	
\end{frame}

% 例 8
\begin{frame}[t]
	\begin{example}
		如图,$A D$ 是 $\triangle A B C$ 的中线,$E$ 是 $A D$ 上的一点,且 $A E=$ $\frac{1}{3} A D, C E$ 交 $A B$ 于点 $F$. 若 $A F=1.2 \mathrm{~cm}$, 求$AB$ 长。	
	\end{example}
	
	
	\tikzset{every picture/.style={line width=0.75pt}} %set default line width to 0.75pt        
	
	\begin{tikzpicture}[x=0.75pt,y=0.75pt,yscale=-1,xscale=1]
		%uncomment if require: \path (0,300); %set diagram left start at 0, and has height of 300
		
		%Shape: Triangle [id:dp8705716247243724] 
		\draw   (374.82,97.67) -- (421.79,212) -- (274.77,212) -- cycle ;
		%Straight Lines [id:da3620881129003368] 
		\draw    (374.82,97.67) -- (345,211.67) ;
		%Straight Lines [id:da45857981996077] 
		\draw    (421.79,212) -- (349,127) ;
		
		% Text Node
		\draw (369.33,79.4) node [anchor=north west][inner sep=0.75pt]    {$A$};
		% Text Node
		\draw (258.67,206.07) node [anchor=north west][inner sep=0.75pt]    {$B$};
		% Text Node
		\draw (424,206.07) node [anchor=north west][inner sep=0.75pt]    {$C$};
		% Text Node
		\draw (340,110.73) node [anchor=north west][inner sep=0.75pt]    {$F$};
		% Text Node
		\draw (337.33,213.07) node [anchor=north west][inner sep=0.75pt]    {$D$};
		% Text Node
		\draw (366,129.07) node [anchor=north west][inner sep=0.75pt]    {$E$};
		
		
	\end{tikzpicture}
	
\end{frame}

% 例 9
\begin{frame}[t]
	\begin{example}
		在 $\triangle A B C$ 中,$P、Q$ 分别是边 $A B$ 和 $A C$ 上的点,中线 $A M$ 与 $P Q$交于 $N$ . 若 $A B: A P=5: 2, A C: A Q=4: 3$, 求$A M: A N$. 
	\end{example}
	
	
	\tikzset{every picture/.style={line width=0.75pt}} %set default line width to 0.75pt        
	
	\begin{tikzpicture}[x=0.75pt,y=0.75pt,yscale=-1,xscale=1]
		%uncomment if require: \path (0,300); %set diagram left start at 0, and has height of 300
		
		%Shape: Triangle [id:dp45951190115695617] 
		\draw   (385.24,95.5) -- (404.52,218.5) -- (239,218.5) -- cycle ;
		%Straight Lines [id:da2856725481898308] 
		\draw    (385.24,95.5) -- (324,218.5) ;
		%Straight Lines [id:da23553510426671842] 
		\draw    (336,136.67) -- (399.33,184.67) ;
		
		% Text Node
		\draw (380,77.4) node [anchor=north west][inner sep=0.75pt]    {$A$};
		% Text Node
		\draw (224,213.4) node [anchor=north west][inner sep=0.75pt]    {$B$};
		% Text Node
		\draw (404,214.4) node [anchor=north west][inner sep=0.75pt]    {$C$};
		% Text Node
		\draw (314,221.4) node [anchor=north west][inner sep=0.75pt]    {$M$};
		% Text Node
		\draw (324,123.73) node [anchor=north west][inner sep=0.75pt]    {$P$};
		% Text Node
		\draw (401,174.73) node [anchor=north west][inner sep=0.75pt]    {$Q$};
		% Text Node
		\draw (362,141.4) node [anchor=north west][inner sep=0.75pt]    {$N$};
		
		
	\end{tikzpicture}
	
\end{frame}

\section{三角形的面积、边角间关系定理}
\begin{frame}[t]
	\begin{theorem}[三角形的角平分线性质定理]
		$\triangle A B C$ 中,若 $A P$ 是 $\angle A$ 的平分线,则
		\begin{align*}
			\frac{B P}{P C}=\frac{A B}{A C}
		\end{align*}
	\end{theorem}
	\begin{theorem}[三角形的外角平分线性质定理]
		$\triangle A B C$ 中,若 $A Q$ 是 $\angle A$ 的外角平分线,则
		\begin{align*}
			\frac{B Q}{Q C}=\frac{A B}{A C} \tag{2-5}
		\end{align*}
	\end{theorem}
\end{frame}

\begin{frame}{调和点列与调和线束}
	% 插入 Geogebra 链接
	\href{http://geogebra.org/m/hpysfjx6}{角平分线性质定理.ggb}
	\tikzset{every picture/.style={line width=0.75pt}} %set default line width to 0.75pt        
	
	\begin{tikzpicture}[x=0.75pt,y=0.75pt,yscale=-1,xscale=1]
		%uncomment if require: \path (0,300); %set diagram left start at 0, and has height of 300
		
		%Shape: Triangle [id:dp155496310519041] 
		\draw   (339.97,132) -- (336.9,207.67) -- (201.32,207.67) -- cycle ;
		%Straight Lines [id:da6060151359191379] 
		\draw    (339.97,132) -- (290,208) ;
		%Straight Lines [id:da6090544304480308] 
		\draw  [dash pattern={on 4.5pt off 4.5pt}]  (227.32,207.67) -- (465.9,207.67) ;
		%Straight Lines [id:da9744149572486258] 
		\draw  [dash pattern={on 4.5pt off 4.5pt}]  (339.97,132) -- (378,110.67) ;
		%Straight Lines [id:da9315879151470301] 
		\draw    (339.97,132) -- (465.9,207.67) ;
		
		% Text Node
		\draw (325,117.07) node [anchor=north west][inner sep=0.75pt]    {$A$};
		% Text Node
		\draw (189.67,200.4) node [anchor=north west][inner sep=0.75pt]    {$B$};
		% Text Node
		\draw (329.67,209.07) node [anchor=north west][inner sep=0.75pt]    {$C$};
		% Text Node
		\draw (375.67,103.73) node [anchor=north west][inner sep=0.75pt]    {$T$};
		% Text Node
		\draw (465,203.73) node [anchor=north west][inner sep=0.75pt]    {$Q$};
		% Text Node
		\draw (283,208.73) node [anchor=north west][inner sep=0.75pt]    {$P$};	
	\end{tikzpicture}
\end{frame}

% 正弦定理和余弦定理
\begin{frame}[t]
	\begin{theorem}[正弦定理]
		在 $\triangle A B C$ 中,角 $A、B、C$ 所对的边长分别为 $a、b、c$, 则
		\begin{align*}
			\frac{a}{\sin  A}=\frac{b}{\sin  B}=\frac{c}{\sin  C}=\frac{a b c}{2 S_{\triangle A B C}}= 2 R.
		\end{align*}
		其中 $R$ 为的$\triangle A B C$ 外接圆半径。
	\end{theorem}
	\begin{itemize}
		\item 大边对大角
		\item 三角形面积公式 $S=\frac{1}{2} a b \sin C =\frac{1}{2} b c \sin A =\frac{1}{2} c a \sin B$
	\end{itemize}
\end{frame}

\begin{frame}[t]
	\begin{theorem}[余弦定理]
		在 $\triangle A B C$ 中,角 $A,B,C$ 所对的边长分别为 $a、b、c$, 则	
		\begin{align*}
			 & c^{2}=a^{2}+b^{2}-2 a b \cdot \cos  C,           \\
			 & b^{2}=c^{2}+a^{2}-2 c a \cdot \cos  B  \tag{2-7} \\
			 & a^{2}=b^{2}+c^{2}-2 b c \cdot \cos  A .
		\end{align*}
	\end{theorem}
	% 插入 Geogebra 链接
	\href{https://www.geogebra.org/m/vmpcfg29}{等面积法 - 余弦定理.ggb}
\end{frame}

% 勾股定理 
\begin{frame}
	\begin{corollary}[勾股定理]
		在 $\triangle A B C$ 中,$\angle C=90^{\circ}$, 则
		\begin{align*}
			c^{2}=a^{2}+b^{2}
		\end{align*}
	\end{corollary}
	\begin{alertblock}{广勾股定理}
		对直角三角形 $\triangle ABC$ 而言,$\angle ADB = 90^{\circ}$, 点 $C$ 为直角边 $BC$ 所在直线上一点,则有
		\begin{align*}
			AB^2 = CA^2 + CB^2 \mp 2CD \cdot CB . 
		\end{align*}
	\end{alertblock}
\end{frame}

% 张角定理
\begin{frame}[t]
	\begin{theorem}[张角定理]
		设 $P$ 为 $\triangle A B C$ 的边 $B C$ 上一点, $\angle B A P=\alpha, \angle C A P=$ $\beta$, 则
		\begin{align*}
			\frac{\sin (\alpha+\beta)}{A P}=\frac{\sin \alpha}{A C}+\frac{\sin \beta}{A B} . 
		\end{align*}
	\end{theorem}
	\tikzset{every picture/.style={line width=0.75pt}} %set default line width to 0.75pt        
	
	\begin{tikzpicture}[x=0.75pt,y=0.75pt,yscale=-1,xscale=1]
		%uncomment if require: \path (0,300); %set diagram left start at 0, and has height of 300
		
		%Shape: Triangle [id:dp1235498259297576] 
		\draw   (376,91) -- (432.53,235.5) -- (260.52,235.5) -- cycle ;
		%Straight Lines [id:da7624772267523616] 
		\draw    (376,91) -- (352,235.5) ;
		
		% Text Node
		\draw (370,70.4) node [anchor=north west][inner sep=0.75pt]    {$A$};
		% Text Node
		\draw (248.52,235.9) node [anchor=north west][inner sep=0.75pt]    {$B$};
		% Text Node
		\draw (434.53,238.9) node [anchor=north west][inner sep=0.75pt]    {$C$};
		% Text Node
		\draw (346,240.4) node [anchor=north west][inner sep=0.75pt]    {$P$};
	\end{tikzpicture}
\end{frame}

% 张角定理的逆定理
\begin{frame}[t]
	\begin{theorem}[张角定理的逆定理]
		设 $B、P、C$ 依次是平面内从一点 $A$ 所引三条射线 $A B、A P、A C$ 上的点 ( $A P$ 在 $A B、A C$ 之间) , $\angle B A P=\alpha, \angle C A P=$ $\beta$, 且 $\alpha+\beta<180^{\circ}$, 若有
		\begin{align*}
			\frac{\sin (\alpha+\beta)}{A P}=\frac{\sin \alpha}{A C}+\frac{\sin \beta}{A B},
		\end{align*}
		则三点 $B, P, C$ 在一条直线上.
	\end{theorem}
\end{frame}

% 斯特瓦尔特定理
\begin{frame}[t]
	\begin{theorem}[斯特瓦尔特定理]
		设 $P$ 为 $\triangle A B C$ 的边 $B C$ 上一点,则
		\begin{align*}
			A P^{2}=A B^{2} \cdot \frac{P C}{B C}+A C^{2} \cdot \frac{B P}{B C}-B C^{2} \cdot \frac{P C}{B C} \cdot \frac{B P}{B C}. 
		\end{align*}
	\end{theorem}
\end{frame}

\begin{frame}[t]{斯特瓦尔特定理的推论}
	特别地,当 $A P$ 为三角形中的重要线段时,有以下结果。
	
	(1) 当 $A P$ 为边 $B C$ 上的中线时,则
	\begin{align*}
		A P^{2}=\frac{1}{2} A B^{2}+\frac{1}{2} A C^{2}-\frac{1}{4} B C^{2} .
	\end{align*}
	(2) 当 $A P$ 为角 $A$ 的平分线时,则
	\begin{align*}
		A P^{2}=A B \cdot A C-B P \cdot P C
	\end{align*}
	(3) 当 $A P$ 为角 $A$ 的外角平分线时,则
	\begin{align*}
		A P^{2}=-A B \cdot A C+B P \cdot P C
	\end{align*}
	(4) 当 $\triangle A B C$ 为等腰三角形,即 $A B=A C$ 时,则
	\begin{align*}
		A P^{2}=A B^{2}-B P \cdot P C
	\end{align*}
	(5) 若 $P$ 分线段 $B C$ 满足 $\frac{B P}{B C}=\lambda$ 时,则
	\begin{align*}
		A P^{2}=\lambda(\lambda-1) B C^{2}+(1-\lambda) \cdot A B^{2}+\lambda \cdot A C^{2}
	\end{align*}
\end{frame}

\setcounter{theorem}{0}
% 例 1
\begin{frame}[t]
	\begin{example}
		在 $\triangle A B C$ 中,已知 $B D$ 和 $C E$ 分别是两边上的中线,并且 $B D \perp C E, B D=4, C E=6$. 那么,$\triangle A B C$ 的面积等于:(\quad) \\
		A. 12 \quad B. 14 \quad C. 16 \quad D. 18
	\end{example}
	\tikzset{every picture/.style={line width=0.75pt}} %set default line width to 0.75pt        
	\begin{tikzpicture}[x=0.75pt,y=0.75pt,yscale=-1,xscale=1]
		%uncomment if require: \path (0,300); %set diagram left start at 0, and has height of 300
		
		%Shape: Triangle [id:dp14872560919008038] 
		\draw   (323.12,101) -- (410.42,228.5) -- (333.33,228.5) -- cycle ;
		%Straight Lines [id:da11373845104470193] 
		\draw    (370,170.33) -- (333.33,228.5) ;
		%Straight Lines [id:da4913852778448766] 
		\draw    (328.67,169) -- (410.42,228.5) ;
		
		% Text Node
		\draw (308.33,93.07) node [anchor=north west][inner sep=0.75pt]    {$A$};
		% Text Node
		\draw (321.67,226.07) node [anchor=north west][inner sep=0.75pt]    {$B$};
		% Text Node
		\draw (410.33,224.4) node [anchor=north west][inner sep=0.75pt]    {$C$};
		% Text Node
		\draw (375,160.07) node [anchor=north west][inner sep=0.75pt]    {$D$};
		% Text Node
		\draw (313.67,160.73) node [anchor=north west][inner sep=0.75pt]    {$E$};
		
		
	\end{tikzpicture}
\end{frame}

% 例 2
\begin{frame}[t]
	\begin{example}
		如图,在 $\triangle A B C$ 中,$E F / / B C$, $S_{\triangle A E F}=S_{\triangle B C E}$ . 若 $S_{\triangle A B C}=1$ ,则 $S_{\triangle C E F}$ 等于 (\quad) . \\
		A. $\frac{1}{4}$ \quad B. $\frac{1}{5}$ \quad C. $\sqrt{5}-2$ \quad D. $\sqrt{3}-\frac{3}{2}$\\
	\end{example}
	\tikzset{every picture/.style={line width=0.75pt}} %set default line width to 0.75pt        
	\begin{tikzpicture}[x=0.75pt,y=0.75pt,yscale=-1,xscale=1]
		%uncomment if require: \path (0,300); %set diagram left start at 0, and has height of 300
		
		%Shape: Triangle [id:dp19523833295070525] 
		\draw   (354.5,118) -- (396,197.5) -- (273,197.5) -- cycle ;
		%Straight Lines [id:da9727740916904974] 
		\draw    (308,163.33) -- (396,197.5) ;
		%Straight Lines [id:da508850838012195] 
		\draw    (308,163.33) -- (378.33,163) ;
		%Straight Lines [id:da08560857314864534] 
		\draw  [dash pattern={on 4.5pt off 4.5pt}]  (273,197.5) -- (378.33,163) ;
		
		% Text Node
		\draw (348,101.73) node [anchor=north west][inner sep=0.75pt]    {$A$};
		% Text Node
		\draw (260,194.4) node [anchor=north west][inner sep=0.75pt]    {$B$};
		% Text Node
		\draw (396,193.07) node [anchor=north west][inner sep=0.75pt]    {$C$};
		% Text Node
		\draw (291.33,151.4) node [anchor=north west][inner sep=0.75pt]    {$E$};
		% Text Node
		\draw (382,151.4) node [anchor=north west][inner sep=0.75pt]    {$F$};
		
		
	\end{tikzpicture}
\end{frame}

% 例 3
\begin{frame}[t]
	\begin{example}
		如图,在 $\triangle A B C$ 中,$D、E$ 分别是 $A C、B C$ 的中点,$B F=$ $\frac{1}{3} A B, B D$ 与 $F C$ 相交于 $G$, 连结 $E G$ . \\
		(1) 求证:$G E / / A C$ ; (2) 求 $\frac{S_{\triangle B F G}}{S_{\triangle B E G}}$ 的值。
	\end{example}
	
	
	\tikzset{every picture/.style={line width=0.75pt}} %set default line width to 0.75pt        
	
	\begin{tikzpicture}[x=0.75pt,y=0.75pt,yscale=-1,xscale=1]
		%uncomment if require: \path (0,300); %set diagram left start at 0, and has height of 300
		
		%Shape: Triangle [id:dp2754705579845924] 
		\draw   (360.5,103) -- (396,202.5) -- (281,202.5) -- cycle ;
		%Straight Lines [id:da5422044594572788] 
		\draw    (281,202.5) -- (378,151.75) ;
		%Straight Lines [id:da3534898199329508] 
		\draw    (308.5,168.25) -- (396,202.5) ;
		%Straight Lines [id:da6391932599504748] 
		\draw    (329.5,177.13) -- (342,202.75) ;
		
		% Text Node
		\draw (353.5,85.9) node [anchor=north west][inner sep=0.75pt]    {$A$};
		% Text Node
		\draw (269,198.4) node [anchor=north west][inner sep=0.75pt]    {$B$};
		% Text Node
		\draw (394,200.4) node [anchor=north west][inner sep=0.75pt]    {$C$};
		% Text Node
		\draw (380.5,140.9) node [anchor=north west][inner sep=0.75pt]    {$D$};
		% Text Node
		\draw (335,203.9) node [anchor=north west][inner sep=0.75pt]    {$E$};
		% Text Node
		\draw (296.5,156.4) node [anchor=north west][inner sep=0.75pt]    {$F$};
		% Text Node
		\draw (325,158.4) node [anchor=north west][inner sep=0.75pt]    {$G$};
		
		
	\end{tikzpicture}
	
\end{frame}

% 例 4
\begin{frame}[t]
	\begin{example}
		如图,$P$ 是 $\triangle A B C$ 内的一点,连结 $A P$ 、 $B P、C P$ 并延长,分别与 $B C、A C、A B$ 交于 $D、E、F$ ,已知: $A P=6, B P=9, P D=6, P E=3, C F=20$ . 求 $\triangle A B C$ 的面积。
	\end{example}
	
	\tikzset{every picture/.style={line width=0.75pt}} %set default line width to 0.75pt        
	
	\begin{tikzpicture}[x=0.75pt,y=0.75pt,yscale=-1,xscale=1]
		%uncomment if require: \path (0,300); %set diagram left start at 0, and has height of 300
		
		%Shape: Triangle [id:dp27354999774492117] 
		\draw   (334,89) -- (399,231.5) -- (329,231.5) -- cycle ;
		%Straight Lines [id:da018715325349367706] 
		\draw    (329,231.5) -- (371,170.25) ;
		%Straight Lines [id:da2765292141653657] 
		\draw    (331,187.25) -- (399,231.5) ;
		%Straight Lines [id:da01926455412965833] 
		\draw    (334,89) -- (356,231.75) ;
		
		% Text Node
		\draw (315.5,231.4) node [anchor=north west][inner sep=0.75pt]    {$A$};
		% Text Node
		\draw (398.5,231.9) node [anchor=north west][inner sep=0.75pt]    {$B$};
		% Text Node
		\draw (328.5,73.4) node [anchor=north west][inner sep=0.75pt]    {$C$};
		% Text Node
		\draw (316.8,179.1) node [anchor=north west][inner sep=0.75pt]    {$E$};
		% Text Node
		\draw (351,231.4) node [anchor=north west][inner sep=0.75pt]    {$F$};
		% Text Node
		\draw (334.8,196.3) node [anchor=north west][inner sep=0.75pt]    {$P$};
		% Text Node
		\draw (371.5,156.4) node [anchor=north west][inner sep=0.75pt]    {$D$};
		
		
	\end{tikzpicture}
\end{frame}

% 例 5
\begin{frame}[t]
	\begin{example}
		如图,$O$ 是凸五边形 $A B C D E$ 内一点,且 $\angle 1=\angle 2, \angle 3=$ $\angle 4, \angle 5=\angle 6, \angle 7=\angle 8$ . 求证: $\angle 9$ 与 $\angle 10$ 相等或互补。
	\end{example}
	
	
	\tikzset{every picture/.style={line width=0.75pt}} %set default line width to 0.75pt        
	
	\begin{tikzpicture}[x=0.75pt,y=0.75pt,yscale=-1,xscale=1]
		%uncomment if require: \path (0,300); %set diagram left start at 0, and has height of 300
		
		%Shape: Regular Polygon [id:dp6162259040636933] 
		\draw   (387.08,211.34) -- (309.49,211.34) -- (285.51,137.54) -- (348.29,91.93) -- (411.06,137.54) -- cycle ;
		%Straight Lines [id:da18111982336114418] 
		\draw    (348.29,91.93) -- (347.5,153.5) ;
		%Straight Lines [id:da7631887076370052] 
		\draw    (347.5,153.5) -- (285.51,137.54) ;
		%Straight Lines [id:da8160979388854117] 
		\draw    (347.5,153.5) -- (309.49,211.34) ;
		%Straight Lines [id:da6330557522358231] 
		\draw    (347.5,153.5) -- (411.06,137.54) ;
		%Straight Lines [id:da6774226670985486] 
		\draw    (347.5,153.5) -- (387.08,211.34) ;
		
		% Text Node
		\draw (333,134.9) node [anchor=north west][inner sep=0.75pt]    {$O$};
		% Text Node
		\draw (340.5,76.4) node [anchor=north west][inner sep=0.75pt]    {$A$};
		% Text Node
		\draw (273,130.9) node [anchor=north west][inner sep=0.75pt]    {$B$};
		% Text Node
		\draw (298.5,210.9) node [anchor=north west][inner sep=0.75pt]    {$C$};
		% Text Node
		\draw (381.5,211.4) node [anchor=north west][inner sep=0.75pt]    {$D$};
		% Text Node
		\draw (410.5,129.9) node [anchor=north west][inner sep=0.75pt]    {$E$};
		
		
	\end{tikzpicture}
\end{frame}

% 例 6
\begin{frame}[t]
	\begin{example}
		已知 $A D、A E$ 分别是 $\triangle A B C$ 的角 $A$ 的内、外角平分线,点 $D$ 在边 $B C$ 上,点 $E$ 在边 $B C$ 的延长线上。求证:$\frac{1}{B E}+\frac{1}{C E}=\frac{2}{D E}$.
	\end{example}
	
	\tikzset{every picture/.style={line width=0.75pt}} %set default line width to 0.75pt        
	
	\begin{tikzpicture}[x=0.75pt,y=0.75pt,yscale=-1,xscale=1]
		%uncomment if require: \path (0,300); %set diagram left start at 0, and has height of 300
		
		%Shape: Triangle [id:dp8372383726649295] 
		\draw   (376.56,131.36) -- (348.81,187.03) -- (249.08,187.03) -- cycle ;
		%Straight Lines [id:da34458523369863703] 
		\draw    (376.32,131.36) -- (315.89,186.78) ;
		%Straight Lines [id:da08131356290560587] 
		\draw  [dash pattern={on 4.5pt off 4.5pt}]  (293.45,187.03) -- (444.95,187.03) ;
		%Straight Lines [id:da13193993881712252] 
		\draw  [dash pattern={on 4.5pt off 4.5pt}]  (376.32,131.36) -- (404.3,115.67) ;
		%Straight Lines [id:da2520301889511636] 
		\draw    (376.32,131.36) -- (444.95,187.03) ;
		%Shape: Arc [id:dp6560204859146406] 
		\draw  [draw opacity=0] (358.61,147.23) .. controls (356.62,145.85) and (355.16,143.55) .. (354.7,140.85) -- (363.05,138.99) -- cycle ; \draw   (358.61,147.23) .. controls (356.62,145.85) and (355.16,143.55) .. (354.7,140.85) ;  
		%Shape: Arc [id:dp3308895194486201] 
		\draw  [draw opacity=0] (365.67,152.79) .. controls (362.54,152.52) and (359.88,150.32) .. (358.61,147.23) -- (366.33,143.16) -- cycle ; \draw   (365.67,152.79) .. controls (362.54,152.52) and (359.88,150.32) .. (358.61,147.23) ;  
		
		% Text Node
		\draw (364.53,115.17) node [anchor=north west][inner sep=0.75pt]    {$A$};
		% Text Node
		\draw (241.12,188.25) node [anchor=north west][inner sep=0.75pt]    {$B$};
		% Text Node
		\draw (342.95,187.28) node [anchor=north west][inner sep=0.75pt]    {$C$};
		% Text Node
		\draw (444.44,184.43) node [anchor=north west][inner sep=0.75pt]    {$E$};
		% Text Node
		\draw (308.36,188.45) node [anchor=north west][inner sep=0.75pt]    {$D$};
		% Text Node
		\draw (427.04,156.16) node [anchor=north west][inner sep=0.75pt]    {$b$};
		% Text Node
		\draw (316.25,160.27) node [anchor=north west][inner sep=0.75pt]    {$a$};
		% Text Node
		\draw (339.88,144.68) node [anchor=north west][inner sep=0.75pt]    {$\alpha $};
		% Text Node
		\draw (348.85,153.28) node [anchor=north west][inner sep=0.75pt]    {$\alpha $};
		
		
	\end{tikzpicture}
	
\end{frame}

% 例 7
\begin{frame}[t]
	\begin{example}
		在 $\triangle A B C$ 中,$A B=2 \sqrt{2}, A C=\sqrt{2}, B C=2$, 设 $P$ 为边 $B C$ 上任一点,则 (\quad) . \\
		A. $P A^{2}<P B \cdot P C$\\
		B. $P A^{2}=P B \cdot P C$\\
		C. $P A^{2}>P B \cdot P C$\\
		D. $P A^{2}$ 与 $P B \cdot P C$ 的大小关系不确定\\
	\end{example}
\end{frame}

% 例 8
\begin{frame}[t]
	\begin{example}
		在 $\triangle A B C$ 中,$A B=A C=2$, 边 $B C$ 上有 100 个不同的点 $P_{1}$ 、 $P_{2}, \cdots, P_{100}$ . 记 $m_{i}=A P_{i}^{2}+B P_{i} \cdot P_{i} C(i=1,2, \cdots, 100)$ ,则 $m_{1}+m_{2}+\cdots$ $+m_{100}=$?\\
	\end{example}
\end{frame}


\section{全等三角形}
\section{相似三角形}
\section{三角形中与比例线段有关的几个定理}
\section{三角形的四心}
\end{document}