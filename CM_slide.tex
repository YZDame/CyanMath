\documentclass[aspectratio=169]{ctexbeamer}
%\usetheme{Madrid}
%\usetheme{Boadilla}
%\usecolortheme{beaver}
\usepackage{amsmath} 
\usepackage{amssymb} 
\usepackage{amsfonts} 
\usepackage{graphicx}
\usepackage{pgfplots}
\pgfplotsset{compat=1.18}
\usefonttheme[onlymath]{serif} % 衬线数学字体

%\setbeamertemplate{theorem}[ams style]
\setbeamertemplate{theorems}[numbered]

\theoremstyle{definition}
\newtheorem{question}{问题}[section]
\newtheorem{exercise}{练习}[section]
\newtheorem{formula}{公式}[section]
\newtheorem{proposition}{命题}[section]
\newtheorem{property}{性质}[section]
	
\title[初等数论]{初等数论}
\subtitle{整除, 同余和不定方程}
\author{LeyuDame}
\date[\today]{\today}
%\AtBeginSection[]
%{
%	\begin{frame}
%		\frametitle{目录}
%		\tableofcontents[currentsection]
%	\end{frame}
%}
\begin{document}
\frame{\titlepage}
%\frame{\frametitle{目录}\tableofcontents}
\section{整除}
\subsection{整除的概念与基本性质}
\begin{frame}{整除的概念与基本性质}
			对任给的两个整数 $a ,  b(a \neq 0)$, 如果存在整数 $q$, 使得 $b=a q$,那么称 $b$ 能被 $a$ 整除(或称 $a$ 能整除 $b$ ), 记作 $a \mid b$. 否则, 称 $b$ 不能被 $a$ 整除, 记作 $a \nmid b$ .

		如果 $a \mid b$, 那么称 $a$ 为 $b$ 的因数, $b$ 为 $a$ 的倍数.
\end{frame}

\begin{frame}{整除的概念与基本性质}
	\begin{property}
		如果 $a \mid b$, 那么 $a \mid(-b)$ ,反过来也成立; 进一步,如果 $a \mid b$, 那么 $(-a) \mid b$ ,反过来也成立.
	\end{property}
	\begin{property}
		如果 $a|b, b| c$, 那么 $a \mid c$ . (传递性)
	\end{property}
	\begin{property}
		若 $a|b, a| c$, 则对任意整数 $x ,  y$, 都有 $a \mid b x+c y$ . (即 $a$ 能整除 $b ,  c$ 的任意一个“线性组合”)
	\end{property}
\end{frame}

% 例 1
\begin{frame}[t]
	\begin{example}
		若 $a|n, b| n$, 且存在整数 $x ,  y$, 使得 $a x+b y=1$, 证明: $a b \mid n$.
	\end{example}
\end{frame}

% 例 2
\begin{frame}[t]
	\begin{example}
		证明:无论在数 12008 的两个 0 之间添加多少个 3 ,所得的数都是 19 的倍数.
	\end{example}
\end{frame}

% 例 3
\begin{frame}[t]
	\begin{example}
		已知一个 1000 位正整数的任意连续 10 个数码形成的 10 位数是 $2^{10}$ 的倍数. 证明:该正整数为 $2^{1000}$ 的倍数.
	\end{example}
\end{frame}

%例 4 
\begin{frame}[t]
	\begin{example}
		设 $m$ 是一个大于 2 的正整数,证明:对任意正整数 $n$ ,都有 $2^{m}-1 \nmid$ $2^{n}+1$.
	\end{example}
\end{frame}

\subsection{素数与合数}
\begin{frame}{素数与合数}
	\begin{property}
		设 $n$ 为大于 1 的正整数, $p$ 是 $n$ 的大于 1 的因数中最小的正整数, 则 $p$ 为素数.
	\end{property}
	\begin{property}
		如果对任意 $1$ 到 $\sqrt{n}$ 之间的素数 $p$, 都有 $p \nmid n$, 那么 $n$ 为素数. 这里 $n(>1)$ 为正整数.
	\end{property}
	\begin{proof}
		事实上, 若 $n$ 为合数, 则可写 $n=p q, 2 \leqslant p \leqslant q$. 因此 $p^{2} \leqslant n$, 即 $p \leqslant \sqrt{n}$ .

		这表明 $p$ 的素因子 $\leqslant \sqrt{n}$ , 且它是 $n$ 的因数, 与条件矛盾. 因此 $n$ 为素数.
	\end{proof}
\end{frame}

\begin{frame}{素数与合数}
	\begin{property}
		素数有无穷多个.
	\end{property}
	\begin{proof}
		若只有有限个素数, 设它们是 $p_{1}<p_{2}<\cdots<p_{n}$. 考虑数
		\begin{align*}
			x=p_{1} p_{2} \cdots p_{n}+1
		\end{align*}
		其最小的大于 1 的因数 $p$, 它是一个素数, 因此, $p$ 应为 $p_{1}, p_{2}, \cdots, p_{n}$ 中的某个数. 设 $p=p_{i}, 1 \leqslant i \leqslant n$, 并且 $x=p_{i} y$, 则 $p_{1} p_{2} \cdots p_{n}+1=p_{i} y$, 即
		\begin{align*}
			p_{i}(y-\left.p_{1} p_{2} \cdots p_{i-1} p_{i+1} \cdots p_{n}\right)=1.
		\end{align*}
		这导致 $p_{i} \mid 1$. 矛盾.

		所以, 素数有无穷多个.
	\end{proof}
\end{frame}

% 例 1
\begin{frame}[t]
	\begin{example}
		设 $n$ 为大于 1 的正整数. 证明: 数 $n^{5}+n^{4}+1$ 不是素数.
	\end{example}
\end{frame}

% 例 2
\begin{frame}[t]
	\begin{example}
		考察下面的数列:
		\begin{align*}
			101,10101,1010101, \cdots
		\end{align*}
		问:该数列中有多少个素数?
	\end{example}
\end{frame}

% 例 3
\begin{frame}[t]
	\begin{example}
		求所有的正整数 $n$, 使得 $\frac{n(n+1)}{2}-1$ 是一个素数.
	\end{example}
\end{frame}

% 例 4
\begin{frame}[t]
	\begin{example}
		对任意正整数 $n$ , 证明: 存在连续 $n$ 个正整数, 它们都是合数.
	\end{example}
\end{frame}

% 例 5
\begin{frame}[t]
	\begin{example}
		设 $n$ 为大于 2 的正整数. 证明: 存在一个素数 $p$ , 满足 $n<p<n!$ .
	\end{example}
\end{frame}

% 例 6
\begin{frame}[t]
	\begin{example}
		设 $a ,  b ,  c ,  d ,  e ,  f$ 都是正整数,  $S=a+b+c+d+e+f$ 是 $a b c+$ $d e f$ 和 $a b+b c+c a-d e-e f-e d$ 的因数. 证明: $S$ 为合数.
	\end{example}
\end{frame}
\subsection{最大公因数与最小公倍数}
\begin{frame}{最大公因数与最小公倍数}
	\begin{block}{带余数除法}
		设 $a ,  b$ 是两个整数, $a \neq 0$, 则存在唯一的一对整数 $q$ 和 $r$,满足
		\begin{align*}
			b=a q+r, 0 \leqslant r<|b|
		\end{align*}
		其中 $q$ 称为 $b$ 除以 $a$ 所得的商, $r$ 称为 $b$ 除以 $a$ 所得的余数.\\
	\end{block}
\end{frame}

\begin{frame}
	\begin{property}[贝祖(Bezout)定理]
		设 $d=(a, b)$ , 则存在整数 $x ,  y$ , 使得
		\begin{align*}
			a x+b y=d
		\end{align*}
	\end{property}
	\begin{property}\label{prop:如果d是a,b公因数则d|(a,b)}
		设 $d$ 为 $a ,  b$ 的公因数, 则 $d \mid(a, b)$ .
	\end{property}\begin{property}\label{prop:a,b互素的充要条件}
		设 $a ,  b$ 是不全为零的整数, 则 $a$ 与 $b$ 互素的充要条件是存在整数 $x ,  y$ 满足
		\begin{align*}
			a x+b y=1
		\end{align*}
	\end{property}
\end{frame}

\begin{frame}
	\begin{property}
		设 $a|c, b| c$ , 且 $(a, b)=1$ , 则 $a b \mid c$ .
	\end{property}
	\begin{property}
		设 $a \mid b c$ , 且 $(a, b)=1$ , 则 $a \mid c$ .
	\end{property}
	\begin{property}\label{prop:如果p是ab的公因数则p是a或b的公因数}
		设 $p$ 为素数,  $p \mid a b$ , 则 $p \mid a$ 或 $p \mid b$ .
	\end{property}
\end{frame}

\begin{frame}{公倍数}
	设 $a ,  b$ 都是不等于零的整数, 如果整数 $c$ 满足 $a \mid c$ 且 $b \mid c$ , 那么称 $c$ 为 $a ,  b$ 的公倍数. 
	
	在 $a ,  b$ 的所有正的公倍数中, 最小的那个称为 $a ,  b$ 的最小公倍数, 记作 $[a, b]$ .
\end{frame}

\begin{frame}
	\begin{property}
		设 $a ,  b$ 为非零整数, $d ,  c$ 分别是 $a ,  b$ 的一个公因数与公倍数,则 $d|(a, b),[a, b]| c$ .
	\end{property}
	\begin{property}\label{prop:最大公因数与最小公倍数的关系}
		设 $a ,  b$ 都是正整数,则 $[a, b]=\frac{a b}{(a, b)}$.
	\end{property}
	\begin{property}
		$\left(a_{1}, a_{2}, a_{3}, \cdots, a_{n}\right)=\left(\left(a_{1}, a_{2}\right), a_{3}, \cdots, a_{n}\right)$ ;

		而 $\left[a_{1}, a_{2}\right.$, $\left.a_{3}, \cdots, a_{n}\right]=\left[\left[a_{1}, a_{2}\right], a_{3}, \cdots, a_{n}\right]$.
	\end{property}
\end{frame}

\begin{frame}
	\begin{property}
		存在整数 $x_{1}, x_{2}, \cdots, x_{n}$ , 使得
		\begin{align*}
			a_{1} x_{1}+a_{2} x_{2}+\cdots+a_{n} x_{n}=\left(a_{1}, a_{2}, \cdots, a_{n}\right)
		\end{align*}
	\end{property}
	\begin{property}\label{prop:最大公因数与最小公倍数的可乘性}
		设 $m$ 为正整数, 则
		\begin{align}
			 & \left(m a_{1}, m a_{2}, \cdots, m a_{n}\right)=m\left(a_{1}, a_{2}, \cdots, a_{n}\right),   \\
			 & {\left[m a_{1}, m a_{2}, \cdots, m a_{n}\right]=m\left[a_{1}, a_{2}, \cdots, a_{n}\right]}.
		\end{align}
	\end{property}
\end{frame}

% 例 1
\begin{frame}[t]
	\begin{example}
		设 $a ,  b$ 为正整数, 且 $\frac{a b}{a+b}$ 也是正整数. 证明:  $(a, b)>1$ .
	\end{example}
\end{frame}

% 例 2
\begin{frame}[t]
	\begin{example}
		设正整数 $a ,  b ,  c$ 满足 $b^{2}=a c$ . 证明: $(a, b)^{2}=a(a, c)$.
	\end{example}
\end{frame}

% 例 3
\begin{frame}[t]
	\begin{example}
		求所有的正整数 $a ,  b(a \leqslant b)$ , 使得
		\begin{align}\label{最大公因数与最小公倍数-例3}
			a b=300+7[a, b]+5(a, b).
		\end{align}
	\end{example}
\end{frame}

% 例 4
\begin{frame}[t]
	\begin{example}
		求所有的正整数 $a ,  b$, 使得
		\begin{align}\label{eq:最大公因数与最小公倍数-例4-1}
			(a, b)+9[a, b]+9(a+b)=7 a b.
		\end{align}
	\end{example}
\end{frame}

% 例 5
\begin{frame}[t]
	\begin{example}
		Fibonacci 数列定义如下:  $F_{1}=F_{2}=1, F_{n+2}=F_{n+1}+F_{n}, n=1$ ,  $2, \cdots$ . 证明: 对任意正整数 $m ,  n$ , 都有 $\left(F_{m}, F_{n}\right)=F_{(m, n)}$.
	\end{example}
\end{frame}

% 例 6
\begin{frame}[t]
	\begin{example}
		设 $n$ 为大于 1 的正整数. 证明: 存在从小到大排列后成等差数列 (即从第二项起, 每一项与它前面那项的差为常数的数列) 的 $n$ 个正整数, 它们中任意两项互素.
	\end{example}
\end{frame}

\subsection{算术基本定理}

\subsection{习题 1}

\section{同余}
\subsection{同余的概念与基本性质}
\end{document}