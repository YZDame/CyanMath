\documentclass[aspectratio=169]{ctexbeamer}
%\usetheme{Madrid}
%\usetheme{Boadilla}
%\usecolortheme{beaver}
\usepackage{amsmath} 
\usepackage{amssymb} 
\usepackage{amsfonts} 
\usepackage{graphicx}
\usepackage{pgfplots}
\pgfplotsset{compat=1.18}
\usefonttheme[onlymath]{serif} % 衬线数学字体

\theoremstyle{definition}
\newtheorem{question}{问题}[section]
\newtheorem{exercise}{练习}[section]
\newtheorem{formula}{公式}[section]
\newtheorem{proposition}{命题}[section]
\newtheorem{property}{性质}[section]

\theoremstyle{plain}
\newtheorem*{analysis}{分析}

\title[初等数论]{初等数论}
\subtitle{整除, 同余和不定方程}
\author{LeyuDame}
\date[\today]{\today}
%\AtBeginSection[]
%{
%	\begin{frame}
%		\frametitle{目录}
%		\tableofcontents[currentsection]
%	\end{frame}
%}
\begin{document}
\frame{\titlepage}
%\frame{\frametitle{目录}\tableofcontents}
\section{同余}
\begin{frame}
	\begin{definition}
		对任给的两个整数 $a ,  b(a \neq 0)$, 如果存在整数 $q$, 使得 $b=a q$,那么称 $b$ 能被 $a$ 整除(或称 $a$ 能整除 $b$ ), 记作 $a \mid b$. 否则, 称 $b$ 不能被 $a$ 整除, 记作 $a \nmid b$ .

		如果 $a \mid b$, 那么称 $a$ 为 $b$ 的因数, $b$ 为 $a$ 的倍数.
	\end{definition}
\end{frame}

\begin{frame}
	\begin{property}
		如果 $a \mid b$, 那么 $a \mid(-b)$ ,反过来也成立; 进一步,如果 $a \mid b$, 那么 $(-a) \mid b$ ,反过来也成立.
	\end{property}
	\begin{property}
		如果 $a|b, b| c$, 那么 $a \mid c$ . 这表明整除具有传递性.
	\end{property}

	\begin{property}
		若 $a|b, a| c$, 则对任意整数 $x ,  y$, 都有 $a \mid b x+c y$ . (即 $a$ 能整除 $b ,  c$ 的任意一个“线性组合”)
	\end{property}
\end{frame}

\begin{frame}
	\begin{example}
		若 $a|n, b| n$, 且存在整数 $x ,  y$, 使得 $a x+b y=1$, 证明: $a b \mid n$.
	\end{example}
\end{frame}

\begin{frame}
	\begin{proof}
		由条件, 可设 $n=a u, n=b v, u ,  v$ 为整数. 于是\\
		\begin{align}
			n & =n(a x+b y)      \\
			  & =n a x+n b y     \\
			  & =a b v x+a b u y \\
			  & =a b(v x+u y)
		\end{align}

		因此
		\begin{align*}
			a b \mid n
		\end{align*}
	\end{proof}
\end{frame}
\end{document}