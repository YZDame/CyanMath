\documentclass[aspectratio=169]{ctexbeamer}
%\usetheme{Madrid}
\usetheme{Boadilla}
%\usecolortheme{beaver}
\usepackage{amsmath} 
\usepackage{amssymb} 
\usepackage{amsfonts} 
\usepackage{multicol}
\usepackage{graphicx}
\usepackage{pgfplots}
\pgfplotsset{compat=1.18}
\usefonttheme[onlymath]{serif} % 衬线数学字体

%\setbeamertemplate{theorem}[ams style]
\setbeamertemplate{theorems}[numbered]

\theoremstyle{definition}
\newtheorem{question}{问题}[section]
\newtheorem{exercise}{练习}[section]
\newtheorem{formula}{公式}[section]
\newtheorem{proposition}{命题}[section]
\newtheorem{property}{性质}[section]
	
\title[因式分解II]{因式分解II}
\subtitle{}
\author[珠海一中创美营]{珠海一中创美营(数学)}
\date[\today]{\today}
\AtBeginSection[]
{
	\begin{frame}
		\frametitle{目录}
		\tableofcontents[currentsection]
	\end{frame}
}
\begin{document}
\frame{\titlepage}
\begin{frame}
	\frametitle{目录}
	\tableofcontents
\end{frame}

\section{分组分解}
% 例 1
\begin{frame}[t]
	\begin{example}[分组分解三部曲]
		分解因式: $a x-b y-b x+a y .$
	\end{example}
\end{frame}

\begin{frame}{分组分解三部曲}
	一般地, 分组分解大致分为三步:
	\begin{enumerate}
		\item 将原式的项适当分组;
		\item 对每一组进行处理(“提”或“代”);
		\item 将经过处理后的每一组当作一项, 再采用“提”或“代”进行分解.
	\end{enumerate}
	一位高明的棋手, 在下棋时, 决不会只看一步. 同样, 在进行分组时, 不仅要看到第二步, 而且要看到第三步.

	一个整式的项有许多种分组的方法, 初学者往往需要经过尝试才能找到适当的分组方法, 但是只要努力实践, 多加练习, 就会成为有经验的“行家”.
\end{frame}

% 例 2
\begin{frame}[t]
	\begin{example}[殊途同归]
		分解因式: $x^{2}+a x^{2}+x+a x-1-a$.
	\end{example}
\end{frame}

% 例 5
\begin{frame}[t]
	\begin{example}[瞄准公式]
		分解因式: $-1-2 x-x^{2}+y^{2}$.
	\end{example}
\end{frame}

% 例 6
\begin{frame}[t]
	\begin{example}[瞄准公式]
		分解因式: $a x^{3}+x+a+1$.
	\end{example}
\end{frame}

% 例 10
\begin{frame}[t]
	\begin{example}[从零开始]
		分解因式: $a b\left(c^{2}-d^{2}\right)-\left(a^{2}-d^{2}\right) c d.$
	\end{example}
\end{frame}

\section{十字相乘}
\subsection{二次三项式}
% 例 2
\begin{frame}[t]
	\begin{example}
		分解因式: $x^{2}-7 x+6.$
	\end{example}
\end{frame}

% 例 7
\begin{frame}[t]
	\begin{example}
		分解因式: $x^{2}+7 x-8$.
	\end{example}
\end{frame}

% 例 9
\begin{frame}[t]
	\begin{example}
		分解因式: $x+12-x^{2}$.
	\end{example}
\end{frame}

% 例 10
\begin{frame}[t]
	\begin{example}[二次项系数不为 1]
		分解因式: $6 x^{2}-7 x+2$.
	\end{example}
\end{frame}

\subsection{二次齐次式}
\begin{frame}{二次齐次式}
	形如 $a x^{2}+b x y+c y^{2}$ 的多项式, 每一项都是 $x$ 与 $y$ 的二次式( $x y$ 中 $x$ 与 $y$ 的次数都是 1 , 所以 $x y$ 的次数是 $1+1=2$ ), 称为 $x$ 与 $y$ 的二次齐次式. 它的分解与 $x$ 的二次三项式一样, 采用十字相乘.
\end{frame}

% 例 13
\begin{frame}[t]
	\begin{example}
		分解因式: $6 x^{2}-7 x y+2 y^{2}$ .
	\end{example}
\end{frame}

\subsection{系数和为零}
\begin{frame}{系数和为零}
	如果二次三项式 $a x^{2}+b x+c$ 的系数和
	\begin{align*}
		a+b+c=0,
	\end{align*}
	那么
	\begin{align*}
		a x^{2}+b x+c=(x-1)(a x-c) .
	\end{align*}
	记住这个结论, 下面的例题就能迎刃而解了.
\end{frame}

% 例 15
\begin{frame}[t]
	\begin{example}
		分解因式: $3 x^{2}+5 x-8$.
	\end{example}
\end{frame}

% 例 16
\begin{frame}[t]
	\begin{example}
		分解因式: $12 x^{2}-19 x y+7 y^{2}$.
	\end{example}
\end{frame}

\subsection{综合运用}
% 例 1
\begin{frame}[t]
	\begin{example}[换元]
		分解因式: $x^{6}-28 x^{3}+27.$
	\end{example}
\end{frame}

% 例 2
\begin{frame}[t]
	\begin{example}
		分解因式: $\left(x^{2}+4 x+8\right)^{2}+3 x\left(x^{2}+4 x+8\right)+2 x^{2}$.
	\end{example}
\end{frame}

% 例 3
\begin{frame}[t]
	\begin{example}
		证明:四个连续整数的乘积加 1 是整数的平方.
	\end{example}
\end{frame}

% 例 4
\begin{frame}[t]
	\begin{example}
		分解因式: $4(x+5)(x+6)(x+10)(x+12)-3 x^{2}.$
	\end{example}
\end{frame}

\section{多项式的因式分解}
\subsection{余数定理}
\begin{frame}{多项式的一次因式}
	设 $a_{n} x^{n}+a_{n-1} x^{n-1}+\cdots+a_{1} x+a_{0}$ 为 $x$ 的 $n$ 次多项式, 本节介绍求它的一次因式的方法.

	我们用 $f(x)$ 表示多项式 $a_{n} x^{n}+a_{n-1} x^{n-1}+\cdots+a_{1} x+a_{0}$ , 用 $f(a)$ 表示这个多项式在 $x=a$ 时的值. 例如, 在 $f(x)=x^{3}+6 x^{2}+11 x+6$ 时, 可得
	\begin{align*}
		 & f(1)=1+6+11+6=24   \\
		 & f(-1)=-1+6-11+6=0  \\
		 & f(+2)=8+24+22+6=60
	\end{align*}
\end{frame}

\begin{frame}
	如果我们用一次多项式 $x-c$ 作除式去除多项式 $f(x)$ , 那么余式是一个数. 设这时商式为多项式 $g(x)$ , 余式(余数)为 $r$ , 则
	\begin{align}\label{eq:多项式除法}
		f(x)=(x-c) Q(x)+r,
	\end{align}
	即被除式等于除式乘以商式再加余式.

	在\ref{eq:多项式除法}式中令 $x=c$ , 便得到
	\begin{align*}
		f(c)=0+r=r,
	\end{align*}
\end{frame}

\begin{frame}{余数定理}
	因此, 我们有
	\begin{center}
		$x-c$ 除 $f(x)$ 时, 所得的余数为 $f(c)$ .
	\end{center}
	这个结论称为\textbf{余数定理}.
	\begin{block}{}
		如果 $f(c)=0$, 那么 $x-c$ 是 $f(x)$ 的因式. 反过来, 如果 $x-c$ 是 $f(x)$ 的因式, 那么 $f(c)=0$ .
	\end{block}
\end{frame}

\begin{frame}[t]
	\begin{example}
		分解因式: $f(x)=x^{3}+6 x^{2}+11 x+6$.
	\end{example}
\end{frame}

\begin{frame}[t]
	\begin{example}
		设 $f(x)=2 x^{3}-5 x^{2}+5 x-3$, 计算 $f(1), f(-1), f\left(\frac{3}{2}\right)$, 并把 $f(x)$ 分解.
	\end{example}
\end{frame}

\subsection{有理根的求法}
\begin{frame}
	如果 $f(c)=0$ , 那么就说 $c$ 是多项式 $f(x)$ 的根. 因此, 在 $c$ 是 $f(x)$ 的根时,  $x-c$ 是 $f(x)$ 的因式. 问题是怎样求出 $f(x)$ 的根?

	我们假定 $f(x)=a_{n} x^{n}+a_{n-1} x^{n-1}+\cdots+a_{1} x+a_{0}$ 是整系数多项式, 也就是说 $a_{n}, a_{n-1}, \cdots, a_{1}, a_{0}$ 都是整数. 又设有理数 $c=\frac{p}{q}$ 是 $f(x)$ 的根, 这里 $p 、$ $q$ 是两个互质的整数.

	由于 $f(c)=0$ , 即
	\begin{align*}
		a_{n}\left(\frac{p}{q}\right)^{n}+a_{n-1}\left(\frac{p}{q}\right)^{n-1}+\cdots+a_{1}\left(\frac{p}{q}\right)+a_{0}=0
	\end{align*}
	两边同乘 $q^{n}$ 得
	\begin{align}\label{eq:有理根}
		a_{n} p^{n}+a_{n-1} p^{n-1} q+\cdots+a_{1} p q^{n-1}+a_{0} q^{n}=0
	\end{align}
\end{frame}

\begin{frame}
	\ref{eq:有理根}式右边被 $p$ 整除( 0 被任何一个不等于 0 的数整除), 所以它的左边也被 $p$ 整除. 显然, 左边的前 $n$ 项都被 $p$ 整除, 所以最后一项 $a_{0} q^{n}$ 也被 $p$ 整除, 但 $p$ 与 $q$ 互质, 所以 $p$ 整除 $a_{0}$ , 即 $p$ 是 $a_{0}$ 的因数(约数). 同样地,  $q$ 应当整除 $a_{n} p^{n}$ , 从而 $q$ 是 $a_{n}$ 的因数(约数). 于是, 可得
	\begin{block}
		有理根 $c=\frac{p}{q}$ 的分子 $p$ 是常数项 $a_{0}$ 的因数, 分母 $q$ 是首项系数 $a_{n}$ 的因数.
	\end{block}
\end{frame}

\begin{frame}[t]
	\begin{example}
		分解因式: $f(x)=2 x^{3}-x^{2}-5 x-2$ . 
	\end{example}
\end{frame}

\begin{frame}[t]
	\begin{example}
		分解因式: $f(x)=3 x^{3}+x^{2}+x-2$ . 
	\end{example}
\end{frame}

\begin{frame}[t]
	\begin{example}
		分解因式: $f(x)=6 x^{4}+5 x^{3}+3 x^{2}-3 x-2$.
	\end{example}
\end{frame}

\subsection{首1多项式}
\begin{frame}[t]{首1多项式}
	对于首项系数为 1 的整系数多项式 $f(x)$ , 问题更加简单. 这时 $q=1$ , 有理根都是整数根. 
	\begin{example}
		分解因式: $x^{6}+2 x^{5}+3 x^{4}+4 x^{3}+3 x^{2}+2 x+1$ . 
	\end{example}
\end{frame}

\begin{frame}[t]
	\begin{example}
		分解因式: $x^{3}-9 x^{2}+26 x-24$.
	\end{example}
\end{frame}

\begin{frame}[t]
	\begin{example}
		分解因式: $x^{3}-9 x^{2} y+26 x y^{2}-24 y^{3}$ . 
	\end{example}
\end{frame}

\begin{frame}[t]
	\begin{example}
		分解因式: $-24 y^{3}+26 y^{2}-9 y+1$ . 
	\end{example}
\end{frame}

\begin{frame}[t]
	\begin{example}
		分解因式: $x^{3}-\frac{5}{3} x^{2}-\frac{11}{3} x-1$.
	\end{example}
\end{frame}

\section*{未完待续}
\begin{frame}
	\Huge
	$$
		\mathcal{TO}	\quad
		\mathcal{BE}  	\quad
		\mathcal{CONTINUED}
		\ldots
	$$
\end{frame}
\end{document}