\documentclass[aspectratio=169]{ctexbeamer}
%\usetheme{Madrid}
\usetheme{Boadilla}
%\usetheme{CambridgeUS}
%\usecolortheme{beaver}
%\usecolortheme{wolverine}
\usepackage{amsmath} 
\usepackage{amssymb} 
\usepackage{amsfonts} 
\usepackage{graphicx}
\usepackage{pgfplots}
\pgfplotsset{compat=1.18}
\usefonttheme[onlymath]{serif} % 衬线数学字体

%\setbeamertemplate{theorem}[ams style]
\setbeamertemplate{theorems}[numbered]

\theoremstyle{definition}
\newtheorem{question}{问题}[section]
\newtheorem{exercise}{练习}[section]
\newtheorem{formula}{公式}[section]
\newtheorem{proposition}{命题}[section]
\newtheorem{property}{性质}[section]
	
\title[整除, 同余和不定方程]{初等数论}
\subtitle{整除, 同余和不定方程}
\author[珠海一中创美营]{珠海一中创美营(数学)}
\date[\today]{\today}
\AtBeginSection[]
{
	\begin{frame}
		\frametitle{目录}
		\tableofcontents[currentsection]
	\end{frame}
}
\begin{document}
\frame{\titlepage}
\frame{\frametitle{目录}\tableofcontents}
\section{整除}
\subsection{整除的概念与基本性质}\setcounter{theorem}{0}
\begin{frame}{整除的概念与基本性质}
	对任给的两个整数 $a ,  b(a \neq 0)$, 如果存在整数 $q$, 使得 $b=a q$,那么称 $b$ 能被 $a$ 整除(或称 $a$ 能整除 $b$ ), 记作 $a \mid b$. 否则, 称 $b$ 不能被 $a$ 整除, 记作 $a \nmid b$ .

	如果 $a \mid b$, 那么称 $a$ 为 $b$ 的因数, $b$ 为 $a$ 的倍数.
\end{frame}

\begin{frame}{整除的概念与基本性质}
	\begin{property}
		如果 $a \mid b$, 那么 $a \mid(-b)$ ,反过来也成立; 进一步,如果 $a \mid b$, 那么 $(-a) \mid b$ ,反过来也成立.
	\end{property}
	\pause
	\begin{property}
		如果 $a|b, b| c$, 那么 $a \mid c$ . (传递性)
	\end{property}
	\pause
	\begin{property}
		若 $a|b, a| c$, 则对任意整数 $x ,  y$, 都有 $a \mid b x+c y$ . (即 $a$ 能整除 $b ,  c$ 的任意一个“线性组合”)
	\end{property}
\end{frame}

% 例 1
\begin{frame}[t]
	\begin{example}
		若 $a|n, b| n$, 且存在整数 $x ,  y$, 使得 $a x+b y=1$, 证明: $a b \mid n$.
	\end{example}
\end{frame}

% 例 2
\begin{frame}[t]
	\begin{example}
		证明:无论在数 12008 的两个 0 之间添加多少个 3 ,所得的数都是 19 的倍数.
	\end{example}
\end{frame}

% 例 3
\begin{frame}[t]
	\begin{example}
		已知一个 1000 位正整数的任意连续 10 个数码形成的 10 位数是 $2^{10}$ 的倍数. 证明:该正整数为 $2^{1000}$ 的倍数.
	\end{example}
\end{frame}

%例 4 
\begin{frame}[t]
	\begin{example}
		设 $m$ 是一个大于 2 的正整数,证明:对任意正整数 $n$ ,都有 $2^{m}-1 \nmid$ $2^{n}+1$.
	\end{example}
\end{frame}

\subsection{素数与合数}\setcounter{theorem}{0}
\begin{frame}{素数与合数}
	\begin{property}
		设 $n$ 为大于 1 的正整数, $p$ 是 $n$ 的大于 1 的因数中最小的正整数, 则 $p$ 为素数.
	\end{property}
	\pause
	\begin{property}
		如果对任意 $1$ 到 $\sqrt{n}$ 之间的素数 $p$, 都有 $p \nmid n$, 那么 $n$ 为素数. 这里 $n(>1)$ 为正整数.
	\end{property}
	\pause
	\begin{proof}
		事实上, 若 $n$ 为合数, 则可写 $n=p q, 2 \leqslant p \leqslant q$. 因此 $p^{2} \leqslant n$, 即 $p \leqslant \sqrt{n}$ .

		这表明 $p$ 的素因子 $\leqslant \sqrt{n}$ , 且它是 $n$ 的因数, 与条件矛盾. 因此 $n$ 为素数.
	\end{proof}
\end{frame}

\begin{frame}{素数与合数}
	\begin{property}
		素数有无穷多个.
	\end{property}
	\pause
	\begin{proof}
		若只有有限个素数, 设它们是 $p_{1}<p_{2}<\cdots<p_{n}$. 考虑数
		\begin{align*}
			x=p_{1} p_{2} \cdots p_{n}+1
		\end{align*}
		其最小的大于 1 的因数 $p$, 它是一个素数, 因此, $p$ 应为 $p_{1}, p_{2}, \cdots, p_{n}$ 中的某个数. 设 $p=p_{i}, 1 \leqslant i \leqslant n$, 并且 $x=p_{i} y$, 则 $p_{1} p_{2} \cdots p_{n}+1=p_{i} y$, 即
		\begin{align*}
			p_{i}(y-\left.p_{1} p_{2} \cdots p_{i-1} p_{i+1} \cdots p_{n}\right)=1.
		\end{align*}
		这导致 $p_{i} \mid 1$. 矛盾.

		所以, 素数有无穷多个.
	\end{proof}
\end{frame}

% 例 1
\begin{frame}[t]
	\begin{example}
		设 $n$ 为大于 1 的正整数. 证明: 数 $n^{5}+n^{4}+1$ 不是素数.
	\end{example}
\end{frame}

% 例 2
\begin{frame}[t]
	\begin{example}
		考察下面的数列:
		\begin{align*}
			101,10101,1010101, \cdots
		\end{align*}
		问:该数列中有多少个素数?
	\end{example}
\end{frame}

% 例 3
\begin{frame}[t]
	\begin{example}
		求所有的正整数 $n$, 使得 $\frac{n(n+1)}{2}-1$ 是一个素数.
	\end{example}
\end{frame}

% 例 4
\begin{frame}[t]
	\begin{example}
		对任意正整数 $n$ , 证明: 存在连续 $n$ 个正整数, 它们都是合数.
	\end{example}
\end{frame}

% 例 5
\begin{frame}[t]
	\begin{example}
		设 $n$ 为大于 2 的正整数. 证明: 存在一个素数 $p$ , 满足 $n<p<n!$ .
	\end{example}
\end{frame}

% 例 6
\begin{frame}[t]
	\begin{example}
		设 $a ,  b ,  c ,  d ,  e ,  f$ 都是正整数,  $S=a+b+c+d+e+f$ 是 $a b c+$ $d e f$ 和 $a b+b c+c a-d e-e f-e d$ 的因数. 证明: $S$ 为合数.
	\end{example}
\end{frame}

\subsection{最大公因数与最小公倍数}\setcounter{theorem}{0}
\begin{frame}{最大公因数与最小公倍数}
	\begin{block}{带余数除法}
		设 $a ,  b$ 是两个整数, $a \neq 0$, 则存在唯一的一对整数 $q$ 和 $r$,满足
		\begin{align*}
			b=a q+r, 0 \leqslant r<|b|
		\end{align*}
		其中 $q$ 称为 $b$ 除以 $a$ 所得的商, $r$ 称为 $b$ 除以 $a$ 所得的余数.\\
	\end{block}
\end{frame}

\begin{frame}
	\begin{property}[贝祖(Bezout)定理]
		设 $d=(a, b)$ , 则存在整数 $x ,  y$ , 使得
		\begin{align*}
			a x+b y=d
		\end{align*}
	\end{property}
	\begin{property}\label{prop:如果d是a,b公因数则d|(a,b)}
		设 $d$ 为 $a ,  b$ 的公因数, 则 $d \mid(a, b)$ .
	\end{property}\begin{property}\label{prop:a,b互素的充要条件}
		设 $a ,  b$ 是不全为零的整数, 则 $a$ 与 $b$ 互素的充要条件是存在整数 $x ,  y$ 满足
		\begin{align*}
			a x+b y=1
		\end{align*}
	\end{property}
\end{frame}

\begin{frame}
	\begin{property}
		设 $a|c, b| c$ , 且 $(a, b)=1$ , 则 $a b \mid c$ .
	\end{property}
	\begin{property}
		设 $a \mid b c$ , 且 $(a, b)=1$ , 则 $a \mid c$ .
	\end{property}
	\begin{property}\label{prop:如果p是ab的公因数则p是a或b的公因数}
		设 $p$ 为素数,  $p \mid a b$ , 则 $p \mid a$ 或 $p \mid b$ .
	\end{property}
\end{frame}

\begin{frame}{公倍数}
	设 $a ,  b$ 都是不等于零的整数, 如果整数 $c$ 满足 $a \mid c$ 且 $b \mid c$ , 那么称 $c$ 为 $a ,  b$ 的公倍数.

	在 $a ,  b$ 的所有正的公倍数中, 最小的那个称为 $a ,  b$ 的最小公倍数, 记作 $[a, b]$ .
\end{frame}

\begin{frame}
	\begin{property}
		设 $a ,  b$ 为非零整数, $d ,  c$ 分别是 $a ,  b$ 的一个公因数与公倍数,则 $d|(a, b),[a, b]| c$ .
	\end{property}
	\begin{property}\label{prop:最大公因数与最小公倍数的关系}
		设 $a ,  b$ 都是正整数,则 $[a, b]=\frac{a b}{(a, b)}$.
	\end{property}
	\begin{property}
		$\left(a_{1}, a_{2}, a_{3}, \cdots, a_{n}\right)=\left(\left(a_{1}, a_{2}\right), a_{3}, \cdots, a_{n}\right)$ ;

		而 $\left[a_{1}, a_{2}\right.$, $\left.a_{3}, \cdots, a_{n}\right]=\left[\left[a_{1}, a_{2}\right], a_{3}, \cdots, a_{n}\right]$.
	\end{property}
\end{frame}

\begin{frame}
	\begin{property}
		存在整数 $x_{1}, x_{2}, \cdots, x_{n}$ , 使得
		\begin{align*}
			a_{1} x_{1}+a_{2} x_{2}+\cdots+a_{n} x_{n}=\left(a_{1}, a_{2}, \cdots, a_{n}\right)
		\end{align*}
	\end{property}
	\begin{property}\label{prop:最大公因数与最小公倍数的可乘性}
		设 $m$ 为正整数, 则
		\begin{align}
			 & \left(m a_{1}, m a_{2}, \cdots, m a_{n}\right)=m\left(a_{1}, a_{2}, \cdots, a_{n}\right),   \\
			 & {\left[m a_{1}, m a_{2}, \cdots, m a_{n}\right]=m\left[a_{1}, a_{2}, \cdots, a_{n}\right]}.
		\end{align}
	\end{property}
\end{frame}

% 例 1
\begin{frame}[t]
	\begin{example}
		设 $a ,  b$ 为正整数, 且 $\frac{a b}{a+b}$ 也是正整数. 证明:  $(a, b)>1$ .
	\end{example}
\end{frame}

% 例 2
\begin{frame}[t]
	\begin{example}
		设正整数 $a ,  b ,  c$ 满足 $b^{2}=a c$ . 证明: $(a, b)^{2}=a(a, c)$.
	\end{example}
\end{frame}

% 例 3
\begin{frame}[t]
	\begin{example}
		求所有的正整数 $a ,  b(a \leqslant b)$ , 使得
		\begin{align}\label{最大公因数与最小公倍数-例3}
			a b=300+7[a, b]+5(a, b).
		\end{align}
	\end{example}
\end{frame}

% 例 4
\begin{frame}[t]
	\begin{example}
		求所有的正整数 $a ,  b$, 使得
		\begin{align}\label{eq:最大公因数与最小公倍数-例4-1}
			(a, b)+9[a, b]+9(a+b)=7 a b.
		\end{align}
	\end{example}
\end{frame}

% 例 5
\begin{frame}[t]
	\begin{example}
		Fibonacci 数列定义如下:  $F_{1}=F_{2}=1, F_{n+2}=F_{n+1}+F_{n}, n=1$ ,  $2, \cdots$ . 证明: 对任意正整数 $m ,  n$ , 都有 $\left(F_{m}, F_{n}\right)=F_{(m, n)}$.
	\end{example}
\end{frame}

% 例 6
\begin{frame}[t]
	\begin{example}
		设 $n$ 为大于 1 的正整数. 证明: 存在从小到大排列后成等差数列 (即从第二项起, 每一项与它前面那项的差为常数的数列) 的 $n$ 个正整数, 它们中任意两项互素.
	\end{example}
\end{frame}

\subsection{算术基本定理}\setcounter{theorem}{0}
\begin{frame}{算术基本定理}
	\begin{theorem}[算术基本定理]
		设 $n$ 是大于 1 的正整数, 则 $n$ 可以分解成若干个素数的乘积的形式, 并且在不考虑这些素数相乘时的前后次序时, 这种分解是唯一的. 即对任意大于 1 的正整数 $n$,都存在唯一的一种素因数分解形式:
		\begin{align*}
			n=p_{1}^{\alpha_{1}} p_{2}^{\alpha_{2}} \cdots p_{k}^{\alpha_{k}}
		\end{align*}

		这里 $p_{1}<p_{2}<\cdots<p_{k}$ 为素数,  $\alpha_{1}, \alpha_{2}, \cdots, \alpha_{k}$ 为正整数.
	\end{theorem}
\end{frame}

% 3 个推论
\begin{frame}
	\begin{corollary}
		设 $n$ 的所有正因数(包括 1 和 $n$ )的个数为 $d(n)$ , 那么
		\begin{align*}
			d(n)=\left(\alpha_{1}+1\right)\left(\alpha_{2}+1\right) \cdots\left(\alpha_{k}+1\right)
		\end{align*}
	\end{corollary}
	\begin{corollary}
		设 $n$ 的所有正因数之和为 $\sigma(n)$ , 那么
		\begin{align*}
			\sigma(n)=\left(1+p_{1}+\cdots+p_{1}^{\alpha_{1}}\right)\left(1+p_{2}+\cdots+p_{2}^{\alpha_{2}}\right) \cdots\left(1+p_{k}+\cdots+p_{k}^{\alpha_{k}}\right)
		\end{align*}
	\end{corollary}
\end{frame}
\begin{frame}
	\begin{corollary}
		设 $n ,  m$ 的素因数分解分别为
		\begin{align*}
			n=p_{1}^{\alpha_{1}} p_{2}^{\alpha_{2}} \cdots p_{k}^{\alpha_{k}}, m=p_{1}^{\beta_{1}} p_{2}^{\beta_{2}} \cdots p_{k}^{\beta_{k}},
		\end{align*}
		这里 $p_{1}<p_{2}<\cdots<p_{k}$ , 都为素数,  $\alpha_{i} ,  \beta_{i}$ 都是非负整数, 并且对每个 $1 \leqslant i \leqslant$ $k, \alpha_{i}$ 与 $\beta_{i}$ 不全为零, 那么, 我们有 $(m, n)=p_{1}^{\gamma_{1}} p_{2}^{\gamma_{2}} \cdots p_{k}^{\gamma_{k}}$ ;  $[m, n]=$ $p_{1}^{\delta_{1}} p_{2}^{\delta_{2}} \cdots p_{k}^{\delta_{k}}$ , 其中 $\gamma_{i}=\min \left\{\alpha_{i}, \beta_{i}\right\}, \delta_{i}=\max \left\{\alpha_{i}, \beta_{i}\right\}, 1 \leqslant i \leqslant k$ .
	\end{corollary}
\end{frame}

\setcounter{theorem}{16}
% 例 1
\begin{frame}[t]
	\begin{example}
		在一个走廊上依次排列着编号为 $1,2, \cdots, 2012$ 的灯共 2012 盏, 最初每盏灯的状态都是开着的. 一个好动的学生做了下面的 2012 次操作: 对 $1 \leqslant k \leqslant 2012$ , 该学生第 $k$ 次操作时, 将所有编号是 $k$ 的倍数的灯的开关都拉了一下. 问:最后还有多少盏灯是开着的?(提示: $44^2=1936, 45^2=2025$)
	\end{example}
\end{frame}

% 例 2 
\begin{frame}[t]
	\begin{example}
		求所有的正整数 $n$, 使得 $n=d(n)^{2}$ .
	\end{example}
\end{frame}


% 例 3
\begin{frame}[t]
	\begin{example}
		设 $n$ 为正整数. 证明:数 $2^{2^{n}}+2^{2^{n-1}}+1$ 至少有 $n$ 个不同的素因子.
	\end{example}
\end{frame}

% 例 4
\begin{frame}[t]
	\begin{example}
		设 $m ,  n$ 是正整数, 且 $m$ 的所有正因数之积等于 $n$ 的所有正因数之积. 问: $m$ 与 $n$ 是否必须相等?
	\end{example}
\end{frame}

% 例 5
\begin{frame}[t]
	\begin{example}
		求所有的正整数 $x, y$, 使得
		\begin{align*}
			y^{x}=x^{50}
		\end{align*}
	\end{example}
\end{frame}

% 例 6
\begin{frame}[t]
	\begin{example}
		给定正整数 $n>1$ , 设 $d_{1}, d_{2}, \cdots, d_{n}$ 都是正整数, 满足:  $\left(d_{1}, d_{2}, \cdots\right.$ ,  $\left.d_{n}\right)=1$ , 且对 $j=1,2, \cdots, n$ 都有 $d_{j} \mid \sum_{i=1}^{n} d_{i}\left(\right.$ 这里 $\sum_{i=1}^{n} d_{i}=d_{1}+d_{2}+\cdots+$ $\left.d_{n}\right)$.

		(1)证明: $d_{1} d_{2} \cdots d_{n} \mid\left(\sum_{i=1}^{n} d_{i}\right)^{n-2}$ ;

		(2)举例说明:  $n>2$ 时, 上式右边的幂次不能减小.
	\end{example}
\end{frame}

\section{同余}
\subsection{同余的概念与基本性质}\setcounter{theorem}{0}
\begin{frame}{同余的概念与基本性质}
	同余是由大数学家高斯引入的一个概念. 我们可以将它理解为"余同", 即余数相同. 正如奇数与偶数是依能否被 2 整除而得到的关于整数的分类一样, 考虑除以 $m(\geqslant 2)$ 所得余数的不同, 可以将整数分为 $m$ 类. 两个属于同一类中的数相对于"参照物" $m$ 而言, 具有"余数相同"这个性质. 这种为对比两个整数的性质, 引入一个参照物的思想是同余理论的一个基本出发点.
\end{frame}

\begin{frame}[t]
	\begin{definition}
		如果 $a ,  b$ 除以 $m(\geqslant 1)$ 所得的余数相同, 那么称 $a ,  b$ 对模 $m$ 同余, 记作 $a \equiv b(\bmod m)$ . 否则, 称 $a ,  b$ 对模 $m$ 不同余, 记作 $a \neq b(\bmod m)$ .
	\end{definition}
	\begin{property}
		$a \equiv b(\bmod m)$ 的充要条件是 $m \mid a-b$ .
	\end{property}
\end{frame}

\begin{frame}[t]
	\begin{property}
		若 $a \equiv b(\bmod m), c \equiv d(\bmod m)$ , 则 $a+c \equiv b+d(\bmod m)$ ,  $a-c \equiv b-d(\bmod m), a c \equiv b d(\bmod m)$.
	\end{property}
	\pause
	\begin{proof}
		这些结论与等式的一些相关结论极其相似, 它们都容易证明. 我们只给出第 3 个式子的证明.

		只需证明:  $m \mid a c-b d$ .

		因为
		\begin{align}
			a c-b d & =a c-b c+b c-b d \\
			        & =(a-b) c+b(c-d)
		\end{align}
		由条件 $m|a-b, m| c-d$ , 知 $m \mid a c-b d$ .
	\end{proof}
\end{frame}

\begin{frame}[t]
	\begin{property}
		若 $a \equiv b(\bmod m), n$ 为正整数, 则 $a^{n} \equiv b^{n}(\bmod m)$ .
	\end{property}
	\begin{property}
		若 $a \equiv b\left(\bmod m_{1}\right), a \equiv b\left(\bmod m_{2}\right)$ , 则 $a \equiv b\left(\bmod \left[m_{1}, m_{2}\right]\right)$ .
	\end{property}
	\begin{property}
		若 $a b \equiv a c(\bmod m)$ , 则 $b \equiv c\left(\bmod \frac{m}{(a, m)}\right)$ .
	\end{property}
\end{frame}

\begin{frame}[t]
	\begin{property}
		如果 $(a, m)=1$ , 那么存在整数 $b$ , 使得 $a b \equiv 1(\bmod m)$ . 这个 $b$称 $a$ 对模 $m$ 的数论倒数, 记为 $a^{-1}(\bmod m)$ , 在不会引起误解时常常简记为 $a^{-1}$ .
	\end{property}
	\begin{proof}
		利用贝祖定理, 可知存在整数 $x ,  y$ 使得
		\begin{align*}
			a x+m y=1
		\end{align*}
		于是,  $m \mid a x-1$ , 即 $a x \equiv 1(\bmod m)$ , 故存在符合条件的 $b$ .
	\end{proof}
\end{frame}

% 例 1
\begin{frame}[t]
	\begin{example}
		求所有的素数 $p ,  q ,  r(p \leqslant q \leqslant r)$ , 使得
		\begin{align*}
			p q+r, p q+r^{2}, q r+p, q r+p^{2}, r p+q, r p+q^{2}
		\end{align*}
		都是素数.
	\end{example}
\end{frame}

% 例 2
\begin{frame}[t]
	\begin{example}
		设 $n$ 为大于 1 的正整数, 且 $1!, 2!, \cdots, n$ !中任意两个数除以 $n$所得的余数不同. 证明: $n$ 是一个素数.
	\end{example}
\end{frame}

% 例 3
\begin{frame}[t]
	\begin{example}
		设整数 $x ,  y ,  z$ 满足
		\begin{align}\label{eq:同余的概念与基本性质-例3}
			(x-y)(y-z)(z-x)=x+y+z .
		\end{align}
		证明: $x+y+z$ 是 27 的倍数.
	\end{example}
\end{frame}

% 例 4
\begin{frame}[t]
	\begin{example}
		是否存在 19 个不同的正整数, 使得在十进制表示下, 它们的数码和相同, 并且这 19 个数之和为 1999 ?
	\end{example}
\end{frame}

% 例 5
\begin{frame}[t]
	\begin{example}
		设 $m ,  n ,  k$ 为正整数,  $n \geqslant m+2, k$ 为大于 1 的奇数, 并且 $p=k \times$ $2^{n}+1$ 为素数,  $p \mid 2^{2^{m}}+1$ . 证明:  $k^{2^{n-1}} \equiv 1(\bmod p)$ .
	\end{example}
\end{frame}

\subsection{剩余系及其应用}\setcounter{theorem}{0}
\begin{frame}{剩余系及其应用}
对任意正整数 $m$ 而言, 一个整数除以 $m$ 所得的余数只能是 $0,1,2, \cdots$ ,  $m-1$ 中的某一个, 依此可将整数分为 $m$ 个类(例如 $m=2$ 时, 就是奇数或偶数), 从每一类中各取一个数所组成的集合就称为模 $m$ 的一个完全剩余系, 简称为模 $m$ 的完系. 依此定义, 可以容易地得到下面的两个性质.
\end{frame}

\begin{frame}
\begin{property}
	若整数 $a_{1}, a_{2}, \cdots, a_{m}$ 对模 $m$ 两两不同余, 则 $a_{1}, a_{2}, \cdots, a_{n}$ 构成模 $m$ 的一个完系.
\end{property}

\begin{property}
	任意连续 $m$ 个整数构成模 $m$ 的一个完系, 其中必有一个数为 $m$的倍数.
\end{property}
引入完系的概念, 蕴含了"整体处理"的思想, 在用同余方法处理数论问题时, 我们常常需要选择不同的完系来达到目的, 做出恰当地分析.
\end{frame}

% 例 1
\begin{frame}[t]
\begin{example}
	证明: 在十进制表示下, 任意 39 个连续正整数中, 必有一个数的数码和是 11 的倍数.
\end{example}
\end{frame}

% 例 2
\begin{frame}[t]
\begin{example}
	设 $n$ 为正奇数. 证明: 数
	\begin{align*}
		2-1,2^{2}-1, \cdots, 2^{n-1}-1
	\end{align*}
	中必有一个数是 $n$ 的倍数.
\end{example}
\end{frame}

% 例 3
\begin{frame}[t]
\begin{example}
	设 $m ,  n$ 为正整数, $m$ 为奇数, 且 $\left(m, 2^{n}-1\right)=1$. 证明: 数 $1^{n}+$ $2^{n}+\cdots+m^{n}$ 是 $m$ 的倍数.
\end{example}
\end{frame}

% 例 4
\begin{frame}[t]
\begin{example}
	(1) 证明: 存在无穷多组整数 $(x, a, b, c)$ , 使得
	\begin{align*}
		x^{2}+a^{2}=(x+1)^{2}+b^{2}=(x+2)^{2}+c^{2}
	\end{align*}\\
	(2) 问: 是否存在整数组 $(x, a, b, c, d)$ , 使得
	\begin{align*}
		x^{2}+a^{2}=(x+1)^{2}+b^{2}=(x+2)^{2}+c^{2}=(x+3)^{2}+d^{2} ?
	\end{align*}
\end{example}
\end{frame}

% 例 5
\begin{frame}[t]
\begin{example}
	设 $n$ 为正整数. 证明: 存在一个各数码都是奇数的正整数, 它是 $5^{n}$的倍数.
\end{example}
\end{frame}

\subsection{费马小定理及其应用}\setcounter{theorem}{0}
\begin{frame}
\begin{theorem}[Fermat 小定理]
	设 $p$ 为素数, $a$ 为整数,则 $a^{p} \equiv a(\bmod p)$ . 特别地,若 $p \nmid a$ , 则 $a^{p-1} \equiv 1(\bmod p)$ .
\end{theorem}
\pause
\begin{proof}
	当 $p \mid a$ 时, 结论显然成立.

	当 $p \nmid a$ 时, 设 $x_{1}, x_{2}, \cdots, x_{p-1}$ 是 $1,2, \cdots, p-1$ 的一个排列, 我们先证:  $a x_{1}, a x_{2}, \cdots, a x_{p-1}$ 中任意两个数对模 $p$ 不同余.

	事实上, 若存在 $1 \leqslant i<j \leqslant p-1$ , 使得 $a x_{i} \equiv a x_{j}(\bmod p)$ , 则 $p \mid a\left(x_{i}-x_{j}\right)$ , 而 $p \nmid a$ , 故 $p \mid x_{i}-x_{j}$ (注意, 这里用到 $p$ 为素数), 但 $x_{i}$ 与 $x_{j}$对模 $p$ 不同余, 矛盾.

	又 $a x_{1}, a x_{2}, \cdots, a x_{p-1}$ 中显然没有一个数为 $p$ 的倍数, 因此,  $a x_{1}$ ,  $a x_{2}, \cdots, a x_{p-1}$ 除以 $p$ 所得的余数是 $1,2, \cdots, p-1$ 的一个排列, 利用同余的性质, 知
	\begin{align*}
		\left(a x_{1}\right)\left(a x_{2}\right) \cdots\left(a x_{p-1}\right) \equiv x_{1} x_{2} \cdots x_{p-1}(\bmod p)
	\end{align*}
	再由 $x_{1} x_{2} \cdots x_{p-1}=(p-1)$ , , 它不是 $p$ 的倍数(注意, 这里再次用到 $p$ 为素数), 所以,  $a^{p-1} \equiv 1(\bmod p)$ .
\end{proof}
\end{frame}

% 例 1
\begin{frame}[t]
\begin{example}
	设 $n$ 为正整数. 证明: $7 \mid 3^{n}+n^{3}$ 的充要条件是 $7 \mid 3^{n} n^{3}+1$ .
\end{example}
\end{frame}

% 例 2
\begin{frame}[t]
\begin{example}
	设 $x$ 为整数,  $p$ 是 $x^{2}+1$ 的奇素因数, 证明:  $p \equiv 1(\bmod 4)$ .
\end{example}
\end{frame}

% 例 3
\begin{frame}[t]
\begin{example}
	设 $x$ 为整数,  $p$ 是数 $x^{6}+x^{5}+\cdots+1$ 的素因数. 证明:  $p=7$ 或 $p \equiv$ $1(\bmod 7)$.
\end{example}
\end{frame}

% 例 4
\begin{frame}[t]
\begin{example}
	设 $p$ 为素数. 证明: 存在无穷多个正整数 $n$ , 使得 $p \mid 2^{n}-n$ .
\end{example}
\end{frame}

% 例 5
\begin{frame}[t]
\begin{example}
	由 Fermat 小定理知, 对任意奇素数 $p$ , 都有 $2^{p-1} \equiv 1(\bmod p)$ . 问: 是否存在合数 $n$ , 使得 $2^{n-1} \equiv 1(\bmod n)$ 成立?
\end{example}
\end{frame}

% 例 6
\begin{frame}[t]
\begin{example}
	求所有的素数 $p$, 使得 $\frac{2^{p-1}-1}{p}$ 是一个完全平方数.
\end{example}
\end{frame}

\subsection{完全平方数}\setcounter{theorem}{0}
\begin{frame}
\begin{property}\label{prop:完全平方数-1}
	完全平方数 $\equiv 0$ 或 $1(\bmod 4)$ , 奇数的平方 $\equiv 1(\bmod 8)$ .
\end{property}
\begin{property}
	相邻两个完全平方数之间没有一个正整数是完全平方数. (这个性质经常用来证明某一类数不是完全平方数)
\end{property}
\begin{property}
	若两个互素的正整数之积是完全平方数, 则这两个数都是完全平方数.
\end{property}
\end{frame}


% 例 1
\begin{frame}[t]
\begin{example}
	设素数从小到大依次排列为 $p_{1}, p_{2}, \cdots$ . 证明: 对任意大于 1 的正整数 $n$ , 数 $p_{1} p_{2} \cdots p_{n}-1$ 和 $p_{1} p_{2} \cdots p_{n}+1$ 都不是完全平方数.
\end{example}
\end{frame}

% 例 2
\begin{frame}[t]
\begin{example}
	已知正整数 $a ,  b$ 满足关系式
	\begin{align*}
		2 a^{2}+a=3 b^{2}+b
	\end{align*}
	证明: $a-b$ 和 $2 a+2 b+1$ 都是完全平方数.
\end{example}
\end{frame}

% 例 3
\begin{frame}[t]
\begin{example}
	设正整数 $x ,  y ,  z$ 满足 $(x, y, z)=1$ , 并且 $\frac{1}{x}+\frac{1}{y}=\frac{1}{z}$. 证明:  $x+y ,  x-z ,  y-z$ 都是完全平方数.
\end{example}
\end{frame}

% 例 4
\begin{frame}[t]
\begin{example}
	求所有的素数 $p$ , 使得 $p^{3}-4 p+9$ 是一个完全平方数.
\end{example}
\end{frame}

% 例 5
\begin{frame}[t]
\begin{example}
	已知 $n$ 为正整数, 且 $2 n+1$ 与 $3 n+1$ 都是完全平方数. 证明:  $40 \mid n$ .
\end{example}
\end{frame}

% 例 6
\begin{frame}[t]
\begin{example}
	若 $a ,  b$ 是使得 $a b+1$ 为完全平方数的正整数, 则记 $a \sim b$. 证明: 若 $a \sim b$ , 则存在正整数 $c$ , 使得 $a \sim c, b \sim c$ .
\end{example}
\end{frame}

% 例 7
\begin{frame}[t]
\begin{example}
	求所有的正整数数对 $(a, b)$, 使得
	\begin{align*}
		a^{3}+6 a b+1, b^{3}+6 a b+1
	\end{align*}
	都是完全立方数.
\end{example}
\end{frame}

% 例 8
\begin{frame}[t]
\begin{example}
	求最小的正整数 $n$, 使得存在整数 $x_{1}, x_{2}, \cdots, x_{n}$ , 满足
	\begin{align*}
		x_{1}^{4}+x_{2}^{4}+\cdots+x_{n}^{4}=1599
	\end{align*}
\end{example}
\end{frame}


\section{不定方程}
\end{document}






















































